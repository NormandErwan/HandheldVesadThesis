%%%%%%%%%%%%%%%%%%%%%%%%%%%%%%%%%%%%%%%%%%%%%%%%%%%
% DECLARATION DE LA CLASSE DU DOCUMENT
%%%%%%%%%%%%%%%%%%%%%%%%%%%%%%%%%%%%%%%%%%%%%%%%%%%
%% Utiliser le format suivant:
% \documentclass[type de papier% ("letterpaper" demandé)
% , simple ou recto verso% ("oneside" ou "twoside")%
% , taille de la police% ("12pt" demandé)%
% , type de document% ("these", "memoire" ou "thesepararticles")%
% , langue du document ("francais" ou "english")%
% , options supplémentaires% ("creativecommons" pour préciser que le document est soumis à la licence créative commomns, "hyperref", "withAlgo2e" pour utiliser le package algorithm2e avec la mise en page correcte de la liste des algorithmes)
%]{thETS}

%% Exemple pour une thèse creative commons, utilisant le package hyperref 
\documentclass[letterpaper%
, twoside%
, 12pt%
, memoire%
, francais
, creativecommons, hyperref, withAlgo2e%
]{thETS} 

%%%%%%%%%%%%%%%%%%%%%%%%%%%%%%%%%%%%%%%%%%%%%%%%%%%
% IMPORTANT: NOTES D'IMPRESSION POUR RESPECTER LES MARGES
%%%%%%%%%%%%%%%%%%%%%%%%%%%%%%%%%%%%%%%%%%%%%%%%%%%
%%  Si vous créez un fichier PDF directement avec PDFLatex, et vous utilisez Acrobat Reader
%% pour faire l'impression, n'oubliez-pas de changer l'option <<Mise à l'échelle>> pour la valeur
%% <<Aucune>> pour que les marges soient imprimés correctement.
%%%%%%%%%%%%%%%%%%%%%%%%%%%%%%%%%%%%%%%%%%%%%%%%%%%

%%%%%%%%%%%%%%%%%%%%%%%%%%%%%%%%%%%%%%%%%%%%%%%%%%%
% DECLARATION DES INFORMATIONS DE LA PAGE TITRE
%%%%%%%%%%%%%%%%%%%%%%%%%%%%%%%%%%%%%%%%%%%%%%%%%%%

\title{Titre du document}

\author{Prénom NOM DE FAMILLE}
\authorcopyright{Prénom Nom}

\datesoutenance{``Date de soutenance''}

\datedepot{``Date du dépôt au Bureau des cycles supérieurs''}

\directeur{M.}{Prénom Nom}{Nom du département et institution}

%\codirecteurB{M.}{Prénom Nom}{département et institution}

\president{M.}{Prénom Nom}{département et institution}

\examinexterne{M.}{Prénom Nom}{département et institution}{}

%%%%%%%%%%%%%%%%%%%%%%%%%%%%%%%%%%%%%%%%%%%%%%%%%%%
% CHANGEMENT DE L'INTITULÉ DU DIPLOME
%%%%%%%%%%%%%%%%%%%%%%%%%%%%%%%%%%%%%%%%%%%%%%%%%%%
%% Il est possible de modifier l'intitulé du diplome en redéfinissant la commande
% \lediplome, comme dans l'exemple suivant:

%\renewcommand{\lediplome}{
%    DE LA\\MAÎTRISE EN GÉNIE ÉLECTRIQUE\\M.Sc.A.
%    }

\listfiles

%%%%%%%%%%%%%%%%%%%%%%%%%%%%%%%%%%%%%%%%%%%%%%%%%%%
% CORPS DU DOCUMENT
%%%%%%%%%%%%%%%%%%%%%%%%%%%%%%%%%%%%%%%%%%%%%%%%%%%
\begin{document}

\pagenumbering{Roman}

%%- Affichage de la page titre -%%
\maketitle

%%- Affichage de la présentation du jury -%%
\presentjury

%%%- Avant propos -%%
\begin{avantpropos}
	\lipsum[1]
\end{avantpropos}


%%- Remerciements -%%
\begin{remerciements}
	\lipsum[1]
\end{remerciements}


%%- Sommaire -%%
\begin{sommaire}{mot-clé1, mot-clé2}
	\lipsum[1]
\end{sommaire}


%%- Abstract -%%
\begin{abstract}{Titre en anglais}{keyword1, keyword2}
	\lipsum[1]
\end{abstract}


%%- Affichage de la table des matières -%%
\tableofcontents


%%- Affichage de la liste des tableaux -%%
\listoftables


%%- Affichage de la liste des Figures -%%
\listoffigures


%%- Déclaration et affichage de la liste des abbréviations -%%
\begin{listofabbr}[3cm]
	\item [ETS] École de Technologie Supérieure
\item [ASC] Agence Spatiale Canadienne
\end{listofabbr}


%%- Déclaration et affichage de la liste des symboles -%%
\begin{listofsymbols}[3cm]
	\item [$\textbf{a}$] Première lettre de l'alphabet
\item [$\textbf{A}$] Première lettre de l'alphabet en majuscule
\end{listofsymbols}


\cleardoublepage

\pagenumbering{arabic}

% Marginpar à gauche du document
\reversemarginpar


%%- Corps du document -%%
\begin{introduction}
	\lipsum[1]
\end{introduction}

\begin{revuedelitterature}
	\lipsum[1]
\end{revuedelitterature}

%%- Premier chapitre de démonstration -%%
\chapter{Test de long titre de Chapitre, avec retour à la ligne. Lorem ipsum dolor sit amet, consectetur adipiscing elit. Pellentesque justo justo, porta sagittis feugiat eget, ornare rhoncus ligula. Nunc non odio sed lacus rutrum rhoncus.}


\section{Tests de mise en page}

Dans cette section, différents environnements de mise en page sont présentés. 

\subsection{Test des listings}

Présentation des principaux listings: les énumations et les listes.


\subsubsection{Énumérations: environement enum}

Test de l'environment enum:
\begin{enumerate}
 \item test 1
 \item test 2
\end{enumerate}

\subsubsection{Listes: environement itemize}

Test de l'environement itemize 
\begin{itemize}
 \item test 1
 \item test 2
\end{itemize}

\subsection{Test des équations}

Mise en page des équations

\begin{equation}
   \beta = 8
\end{equation}

\begin{equation}
   \bm{\gamma} = \alpha \times 3
\end{equation}

\section{Seconde section}

Exemple de seconde section pour illustrer la mise en page de la table des matières


%%- Deuxiemme chapitre de démonstration -%%
\chapter{Ajout d'un second chapitre}

\section{Test de mise en page d'un tableau}

Les tableaux sont soumis aux mêmes contraintes que les figures, en dehors de la position de la légende qui doit être au dessus.


\begin{table}
		\parbox{0.65\textwidth}{\caption{Test de longue légende pour un tableau, avec retour à la ligne.}} % Contrainte manuelle de la largeur de la légende
		\begin{tabular}{|c|c|c|c|c|c|c|c|}
		\hline
			{\bf titre} & {\bf titre} & {\bf titre} & {\bf titre} & {\bf titre} & {\bf titre} & {\bf titre} & {\bf titre} \\
	  \hline
			blá & blá & blá & blá & blá & blá & blá & blá \\
	  \hline
			blá & blá & blá & blá & blá & blá & blá & blá \\
	  \hline
			blá & blá & blá & blá & blá & blá & blá & blá \\
	  \hline
			blá & blá & blá & blá & blá & blá & blá & blá \\
	  \hline
			blá & blá & blá & blá & blá & blá & blá & blá \\
	  \hline
			blá & blá & blá & blá & blá & blá & blá & blá \\
	  \hline
		\end{tabular}
\end{table}


\section{Test des références}

\subsection{Références à la bibliographie}

Citation d'une référence de la bibliographie \cite{Arica2002}.

\subsection{Références à un label du document}

Référence à une Figure associée à un label: Figure \ref{fig:vueEts}.

\subsection{Références à des adresses}

\subsubsection{Test de href}

Utilisation de href, pour intégrer un lien dans une portion de texte:
\href{http://www.etsmtl.ca/Etudiants-actuels/Cycles-sup/Realisation-etudes/Guides-gabarits}{Lien vers la page des gabarits de l'ÉTS}.

\subsubsection{Test de url}

Utilisation de url pour citer un lien cliquable:
\url{http://www.etsmtl.ca/Etudiants-actuels/Cycles-sup/Realisation-etudes/Guides-gabarits}.

\begin{conclusion}
	\lipsum[1]
\end{conclusion}


%%- Annexes -%%
\appendix

%% Lorsqu'on a plus qu'une annexe.
\multiannexe

\lipsum[1]


%%- Bibliographie -%%
\newpage % Interligne sinmple pour la bibliographie
\begin{spacing}{1}  
	\nocite{*} % Utiliser la commande nocite pour afficher des références qui n'ont pas été citées dans le document. '*' permet de toutes les afficher.
	\bibliographystyle{bibETS} % Utilisation du style bibliographique de l'ETS
	\addcontentsline{toc}{chapter}{BIBLIOGRAPHIE} % Ajout de la bibliographie à la table des matières
	
	\bibliography{./bibliography/bibliography} % Liste des fichiers bib de bibliographie, biblio.bib est un exemple
\end{spacing}

\end{document}
