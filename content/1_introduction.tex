La réalité augmentée (RA) est une \textquote{technique [\dots] consistant à superposer en temps réel des images virtuelles [\dots] à des images issues du monde réel [\dots]}. \citep{OQLFRA2017}, c'est-à-dire générer des objets virtuels en trois dimensions (3D) combinés avec l'environnement réel d'un utilisateur, donnant l'illusion que le virtuel coexiste avec le réel. Ainsi la RA permet d'\emph{augmenter la perception} du réel et permet d'\emph{augmenter les interactions} possibles d'un utilisateur avec son environnement \citep{Azuma1997}. La RA peut toucher tous les sens humains, mais est principalement utilisée pour le sens visuel.

\figureLayoutETS{HoloLens}{%
  \subfigureETS[0.19]{HoloLens_1}{Affichage d'un modèle 3D dans la pièce.}%
  \figurehspace%
  \subfigureETS[0.19]{HoloLens_2}{Les applications prennent généralement la forme de fenêtres virtuelles que l'utilisateur place sur un mur.}%
}{
  Illustrations publicitaires de démonstration du visiocasque de RA Microsoft HoloLens.\\
  Tiré de \cite{Microsoft2018}.
}

Les deux précédentes années ont été très importantes pour la RA. Microsoft a provoqué beaucoup d'intérêt en proposant en 2016 avec le HoloLens (\url{https://www.microsoft.com/hololens}) un visiocasque de RA portable aux performances bien supérieures à celle de ses concurrents \figrefp{HoloLens}. Puis, en 2017, Google et Apple ont permis de concevoir des applications de RA sur les milliards de téléphones intelligents Android et iOS dans le monde avec leurs plateformes de développement ARCore (\url{https://developers.google.com/ar/}) et ARKit (\url{https://developer.apple.com/arkit/}). Couplé aux annonces pour 2018 des visiocasques Magic Leap (\url{https://www.magicleap.com/}), North Star (\url{https://developer.leapmotion.com/northstar/}) et Meta 2 (\url{https://www.metavision.com/}), le marché de la RA est promis à devenir très important dans les années à venir.

\figureETS[0.9]{GartnerHypeCycle2017}{
  Courbe d'intérêt pour les nouvelles technologies, présenté en fonction du temps, en cinq phases. La RA est dans la phase trois du \textquote{gouffre des désillusions}, la RV dans la phase quatre de la \textquote{pente de l'illumination}.\\
  Adapté de \cite{GartnerHypeCurve2017}.
}

La RA reste cependant une technologie encore peu connue et techniquement moins mature que la réalité virtuelle (RV). Elle est tout de même particulièrement prometteuse et va probablement profondément transformer nos quotidiens personnels et professionnels dans les cinq à dix prochaines années \figrefp{GartnerHypeCycle2017}, comme ont pu le faire les téléphones intelligents à la fin des années 2000 \citep{Chaffey2018}. La RA nous accompagnera en construisant des interfaces humain-machines (IHM) naturelles avec nos ordinateurs, téléphones et objets connectés. Cette vision est également partagée par des entreprises importantes dans les secteurs de la RV et la RA, comme Unity : \textquote{We believe [augmented reality] should be human and object centric, and use the world as a device} \citep{UnityFutureMRPartIII2017} qui ont proposé d'utiliser les visiocasques de RA pour étendre des cartes de visite \figrefp{UnityFutureMRPartII2017}, ou totalement remplacer l'ordinateur de bureau \figrefp{UnityFutureMRPartIII2017}.

\figureLayoutETS{UnityFutureMR2017}{%
  \subfigureETS[0.19]{UnityFutureMRPartII2017}{Extension d'une carte de visite.}%
  \figurehspace%
  \subfigureETS[0.19]{UnityFutureMRPartIII2017}{Bureau de travail en RA.}%
}{
  Concepts d'IHMs de RA dans un environnement de bureaux.\\
  Tiré de a) \cite{UnityFutureMRPartII2017} et b) \cite{UnityFutureMRPartIII2017}.
}

Ainsi, comme pour les interfaces graphiques des ordinateurs dans les années 1990 ou celles des téléphones intelligents dans les années 2000, nous nous interrogeons sur la forme qu'aurait une IHM d'un système d'exploitation utilisant de la RA. En ce sens, nous pensons alors intéressant de combiner les téléphones intelligents avec des visiocasques de RA. Ces deux technologies sont en effet mobiles mais présentent des caractéristiques d'IHM différentes qui semblent complémentaires :
\begin{enumerate}
  \item Les téléphones intelligents sont de plus en plus puissants et possèdent des écrans haute définition mais ont atteint une limite de taille physique d'écran pour pouvoir être tenus dans la main. Leurs utilisateurs pourraient bénéficier d'un plus grand écran pour d'utiliser plusieurs applications simultanément, de naviguer parmi de nombreux fichiers ou photos, ou encore de visualiser de grandes cartes ou des photos haute-résolution. De plus, le doigt est stabilisé sur la surface de l'écran, rendant les interactions tactiles précises, et demande de petits mouvements faciles à exécuter \citep{Argelaguet2013}.
  \item Les visiocasques de RA permettent d'entourer un utilisateur de nombreux et grands écrans virtuels \citep{Ens2014}. Il n'est pourtant pas encore clair quelles interfaces et techniques d'interactions sont les plus adaptées RA \citep{Piumsomboon2013, Billinghurst2015}. Le pointage en 3D avec la main, par exemple, permet de pointer directement sur le contenu virtuel de manière similaire aux interactions avec des objets réels. Cependant, le manque de retour haptique de ces interfaces intangibles \citep{Chan2010}, les mouvements complexes demandés au bras \citep{Argelaguet2013} ou encore les problèmes d'occlusion avec le contenu virtuel \cite{Piumsomboon2014} peuvent rendre cette technique difficile.
\end{enumerate}
\medskip

\figureETS{HandheldVESADMidAirInArOut}{
  Photomontage d'une vue à la troisième personne de notre tâche expérimentale, dans la condition \condition{VESAD} : une grille est affichée sur l'écran physique du téléphone et sur une fenêtre virtuelle située autour formant ainsi un unique écran étendu.
}

Nous proposons alors de concevoir un prototype de téléphone intelligent dont l'écran est étendu par RA. Nous appelons l'IHM de ce prototype \texten{Virtuality Extended Screen-Aligned Display} (VESAD) : une fenêtre virtuelle est affichée en RA, placée autour de l'écran du téléphone, coplanaire, synchronisé avec lui et le suivant dans ses mouvements. Cela donne à l'utilisateur le sentiment d'un seul grand écran étendu tenu en main, permettant d'afficher beaucoup plus d'information en même temps \figrefp{HandheldVESADMidAirInArOut}. Les interactions pourraient se faire sur l'écran tactile qui sont précises, stables et connues ou dans l'espace autour qui est plus direct et possiblement plus intuitif. On peut s'interroger alors si une technique d'interaction est à préférer ? Comment placer le contenu virtuel sur l'écran étendu ? Ainsi, nous évaluons expérimentalement, dans ce travail de recherche, un VESAD appliqué à un téléphone intelligent comparé à un téléphone non étendu, ainsi que ces deux techniques d'interactions.

Après notre revue de littérature et la définition de notre problématique au \autoref{ch:litterature}, nous explorons le concept de notre prototype au \autoref{ch:concept}, puis nous exposons son développement au \autoref{ch:methodology}. Nous menons ensuite une évaluation expérimentale comparant différents usages du VESAD face à un téléphone seul sur la tâche de classification de \cite{Liu2014} au \autoref{ch:experiment} : nous y exposons sa conception ainsi que nos résultats. Enfin, nous discutons nos résultats et révisons notre concept au \autoref{ch:discussion}.