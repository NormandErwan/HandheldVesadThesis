La réalité augmentée (RA) est une « technique [\dots] consistant à superposer en temps réel des images virtuelles [\dots] à des images issues du monde réel [\dots]. » \citep{OQLFRA2017}. La RA consiste donc à générer des objets virtuel en trois dimension (3D) combinés avec l'environnement réel d'un utilisateur, donnant l'illusion que le virtuel coexiste avec le réel. Ainsi la RA permet d'\emph{augmenter la perception} du réel et permet d'\emph{augmenter les interactions} possibles d'un utilisateur avec son environnement \citep{Azuma1997}. La RA peut toucher tous les sens humains, mais est principalement utilisée pour le sens visuel \reffigureETSp{HoloLens.jpg}.

Les deux précédents années, 2016 et 2017, ont été très importantes pour la RA. Microsoft a révolutionné les visiocasques de RA avec la sortie de son Microsoft HoloLens (\url{https://www.microsoft.com/hololens}) en 2016, qui s'imposera probablement en standard dans l'industrie. Puis, en 2017, Google et Apple ont permis à n'importe qui de concevoir des applications de RA sur les milliards de téléphones intelligents Android et iOS dans le monde avec leurs plateformes de développement ARCore (\url{https://developers.google.com/ar/}) et ARKit (\url{https://developer.apple.com/arkit/}). Couplé aux annonces pour 2018 des visiocasques Magic Leap (\url{https://www.magicleap.com/}) et Meta (\url{https://www.metavision.com/}), le marché de la RA est promis à devenir très important dans les années à venir.

\figureETS[0.5]{HoloLens.jpg}{
  Illustrations publicitaire du Microsoft HoloLens. On peut voir le visiocasque de RA sur la tête de l'utilisatrice, projettant du contenu virtuel sur le mur en face d'elle.\\
  Tiré de \cite{Microsoft2018}.
}

Si la RA reste encore une technologie encore peu connue, techniquement moins mature que la réalité virtuelle (RV), elle reste cependant particulièrement prometteuse et va probablement révolutionner nos quotidiens personnels et professionnels dans les cinq à dix prochaines années \reffigureETSp{GartnerHypeCycle2017.jpg}, comme ont pu le faire les téléphones intelligents à la fin des années 2000 \citep{Chaffey2018}. En effet, la RA permettrait de nous accompagner dans nos quotidiens en construisant des interfaces humain-machines (IHM) naturelles avec nos ordinateurs, téléphones et objets connectés. Cette vision est également partagée par des entreprises importantes dans les secteurs de la RV et la RA comme Unity3D : « We believe [augmented reality] should be human and object centric, and use the world as a device » \citep{UnityFutureMRPartIII2017}.

\figureETS{GartnerHypeCycle2017.jpg}{
  Courbe d'intérêt pour les nouvelles technologies, juillet 2017. L'intérêt est présenté en fonction du temps, en cinq phases. La RA est dans la troisième phase du « gouffre des désillusions », la RV dans la quatrième phase de la « pente de l'illumination ».\\
  Adapté de \cite{GartnerHypeCurve2017}.
}

Ainsi, comme pour les interfaces graphiques des PCs dans les années 1990 ou celles des téléphones intelligents dans les années 2000, nous nous interrogeons sur la forme qu'aurait une IHM d'un système d'exploitation utilisant de la RA. En ce sens, nous pensons alors intéressant de combiner les téléphones intelligents avec des visiocasques de RA. Ces deux technologies sont en effet mobiles mais présentent des caractéristiques d'IHM différentes qui semblent complémentaires :
\begin{enumerate}
  \item Les téléphones intelligents sont de plus en plus puissants mais ont atteint une limite de taille physique d'écran pour pouvoir être tenus avec une seule main. Leurs utilisateurs bénéficieraient d'un écran de taille comparable à celui d'un PC permettant d'utiliser plusieurs applications simultanément, de naviguer parmi de nombreux fichiers ou photos, ou encore de visualiser de grandes cartes ou des photos haute-résolution.
  \item Les visiocasques de RA permettent d'entourer un utilisateur de nombreux et grands écrans virtuels \citep{Ens2014}. Cependant, 
  les techniques d'interactions sont encore peu étudiées en RA \citep{Piumsomboon2013}, tandis qu'elles sont bien maîtrisées pour les écrans tactiles \citep{Wobbrock2009}. De plus, interagir avec la main à travers des interfaces intangibles (« flottant en l'air ») est difficile \citep{Chan2010} et fatiguant \citep{Hincapie-Ramos2014}, comparé aux interactions avec un écran tactile.
\end{enumerate}

Nous proposons de concevoir un prototype de téléphone intelligent tenu en main dont l'écran est étendu par RA et de l'évaluer expérimentalement par rapport à un téléphone seul non étendu.

Après une revue de littérature au \autoref{ch:litterature}, où sera également définie la problématique et les objectifs de ce mémoire, le concept de notre prototype sera exploré au \autoref{ch:concept}, puis sa conception et son développement seront décrits au \autoref{ch:methodology}. L'évaluation expérimentale et ses résultats seront ensuite exposés et discutés au \autoref{ch:experiment}. Enfin, le concept sera discuté au regard de l'expérience au \autoref{ch:futur_work}, suivi de la conclusion.