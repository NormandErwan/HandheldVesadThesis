La réalité augmentée (RA) est une \textquote{technique [\dots] consistant à superposer en temps réel des images virtuelles [\dots] à des images issues du monde réel [\dots]}. \citep{OQLFRA2017}, c'est-à-dire générer des objets virtuel en trois dimension (3D) combinés avec l'environnement réel d'un utilisateur, donnant l'illusion que le virtuel coexiste avec le réel. Ainsi la RA permet d'\emph{augmenter la perception} du réel et permet d'\emph{augmenter les interactions} possibles d'un utilisateur avec son environnement \citep{Azuma1997}. La RA peut toucher tous les sens humains, mais est principalement utilisée pour le sens visuel.

\figureLayoutETS{HoloLens}{%
  \subfigureETS[0.2]{HoloLens_1.jpg}{Affichage d'un modèle 3D dans la pièce.}%
  \figurehspace%
  \subfigureETS[0.2]{HoloLens_2.jpg}{Les applications prennent généralement la forme de fenêtres virtuelles que l'utilisateur place sur un mur.}%
}{
  Illustrations publicitaire du visiocasque de RA Microsoft HoloLens.\\
  Tiré de \cite{Microsoft2018}.
}

Les deux précédentes années ont été très importantes pour la RA. Microsoft a révolutionné l'industrie de la RA avec la sortie de son visiocasque en 2016, le HoloLens (\url{https://www.microsoft.com/hololens}), qui semble s'imposer comme un standard dans l'industrie \reffigureETSp{HoloLens}. Puis, en 2017, Google et Apple ont permis à n'importe qui de concevoir des applications de RA sur les milliards de téléphones intelligents Android et iOS dans le monde avec leurs plateformes de développement ARCore (\url{https://developers.google.com/ar/}) et ARKit (\url{https://developer.apple.com/arkit/}). Couplé aux annonces pour 2018 des visiocasques Magic Leap (\url{https://www.magicleap.com/}) et Meta (\url{https://www.metavision.com/}), le marché de la RA est promis à devenir très important dans les années à venir.

\figureETS[0.9]{GartnerHypeCycle2017.jpg}{
  Courbe d'intérêt pour les nouvelles technologies, présenté en fonction du temps, en cinq phases. La RA est dans la phase trois du \textquote{gouffre des désillusions}, la RV dans la phase quatre de la \textquote{pente de l'illumination}.\\
  Adapté de \cite{GartnerHypeCurve2017}.
}

La RA reste cependant encore une technologie encore peu connue et techniquement moins mature que la réalité virtuelle (RV). Elle est tout de même particulièrement prometteuse et va probablement profondément transformer nos quotidiens personnels et professionnels dans les cinq à dix prochaines années \reffigureETSp{GartnerHypeCycle2017.jpg}, comme ont pu le faire les téléphones intelligents à la fin des années 2000 \citep{Chaffey2018}. La RA nous accompagnera en construisant des interfaces humain-machines (IHM) naturelles avec nos ordinateurs, téléphones et objets connectés. Cette vision est également partagée par des entreprises importantes dans les secteurs de la RV et la RA comme Unity3D : \textquote{We believe [augmented reality] should be human and object centric, and use the world as a device} \citep{UnityFutureMRPartIII2017}.

Ainsi, comme pour les interfaces graphiques des ordinateurs dans les années 1990 ou celles des téléphones intelligents dans les années 2000, nous nous interrogeons sur la forme qu'aurait une IHM d'un système d'exploitation utilisant de la RA. En ce sens, nous pensons alors intéressant de combiner les téléphones intelligents avec des visiocasques de RA. Ces deux technologies sont en effet mobiles mais présentent des caractéristiques d'IHM différentes qui semblent complémentaires :
\begin{enumerate}
  \item Les téléphones intelligents sont de plus en plus puissants et possèdent des écrans hautes définitions mais ont atteint une limite de taille physique d'écran pour pouvoir être tenus dans la main. Leurs utilisateurs bénéficieraient d'un plus grand écran pour d'utiliser plusieurs applications simultanément, de naviguer parmi de nombreux fichiers ou photos, ou encore de visualiser de grandes cartes ou des photos haute-résolution.
  \item Les visiocasques de RA permettent d'entourer un utilisateur de nombreux et grands écrans virtuels \citep{Ens2014}. Cependant, 
  les techniques d'interactions sont encore peu étudiées en RA \citep{Piumsomboon2013}, tandis qu'elles sont bien maîtrisées pour les écrans tactiles \citep{Wobbrock2009}. En outre, les interactions en l'air avec la main à travers des interfaces intangibles (\textquote{flottant en l'air}) est difficile \citep{Chan2010} et fatiguant \citep{Hincapie-Ramos2014}, comparé aux interactions avec un écran tactile qui se font sur une surface physique stable.
\end{enumerate}
\bigskip

\figureETS{HandheldVESADMidAirInArOut.jpg}{
  Photomontage d'une vue à la troisième personne de notre tâche expérimentale, dans la condition \condition{VESAD} : une grille est affichée sur l'écran physique du téléphone et sur une fenêtre virtuelle située autour formant ainsi un unique écran étendu.
}

Nous proposons alors de concevoir un prototype de téléphone intelligent dont l'écran est étendu par RA et de l'évaluer expérimentalement par rapport à un téléphone non étendu. Nous appelons l'IHM de ce prototype \texten{Virtuality Extended Screen-Aligned Display} (VESAD) : une fenêtre virtuelle est affichée en RA, placée de l'écran du téléphone. Cela donne à l'utilisateur le sentiment d'un seul écran étendu de la taille de celui d'un ordinateur mais tenu en main et donc mobile \reffigureETSp{HandheldVESADMidAirInArOut.jpg}.

Après notre revue de littérature et la définition de notre problématique au \autoref{ch:litterature}, nous explorons le concept de notre prototype au \autoref{ch:concept}, puis nous exposons son développement au \autoref{ch:methodology}. Nous menons ensuite une évaluation expérimentale comparant différents usages du VESAD face à un téléphone seul sur la tâche de classification de \cite{Liu2014} au \autoref{ch:experiment} : nous y exposons sa conception ainsi que nos résultats que nous discutons. Enfin, nous révisons notre concept au regard de ces résultats au \autoref{ch:futur_work}.