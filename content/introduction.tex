\section*{Introduction}
Définition RA + rapide historique. \\

% Limites
La réalité augmentée (RA) peut toucher tous les sens humains. Cependant, on se concentra exclusivement sur la RA pour le sens visuel dans cette revue de littérature. Par conséquent le terme RA sera utilisé ici comme la réalité augmentée pour le sens visuel. \\
Il existe aujourd'hui plusieurs techniques de RA, classées en trois catégories \reffig{BimberRaskar2005-Figure31}. La catégorie dominant actuellement le marché est celle des visiocasques (ou \foreignlanguage{english}{Head-Mounted Display}) \cite[]{VanKrevelenPoelman2010}~: on le voit par la présence médiatique des visiocasques grand public tel que l'Oculus Rift, le HTC Vive ou Microsoft HoloLens. Seule cette catégorie sera explorée dans cette revue de littérature. Les techniques de cette catégorie consistent à placer un casque devant les yeux de l'utilisateur pour y diffuser la RA. Un premier type de ces techniques, dites \foreignlanguage{english}{video see-through}, vont remplacer l'environnement par une image filmée et augmentée de cet environnement. Cela va se faire en utilisant des caméras à l'avant du casque et en modifiant les images filmées pour les renvoyer l'écran du casque. Un second type de ces techniques, dites \foreignlanguage{english}{optical see-through}, vont laisser voir à l'utilisateur directement l'environnement, et, par un jeu de miroirs et de lentilles, vont y superposer les images de RA.

\figureETS{figures/BimberRaskar2005-Figure31.png}{Catégories des techniques d'affichages de RA \\ Tiré de \cite[p. 72]{BimberRaskar2005}}{BimberRaskar2005-Figure31}