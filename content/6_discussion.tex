\chapter{Discussion}
\label{ch:discussion}

\section{Discussion des résultats}
\label{sec:discussion_results}

Dans l'ensemble, nos résultats indiquent que le \condition{VESAD tactile} a été la meilleure des trois IHMs. Elle a tout d'abord été la plus rapide, avec le plus petit temps de complétion ($\approx \SI{62}{\s}$), en particulier quand la tâche était maîtrisée des participants ($\approx \SI{42}{\s}$). Elle a ensuite été largement préférée des participants, et jugé la plus performante. Enfin, elle a été la plus performante dans le classement : d'une part, car les participants utilisaient en moyenne une seule sélection et un seul zoom par disque à classer et, d'autre part, car le temps passé avec un disque sélectionné était beaucoup plus court qu'avec les autres IHMs. Ce n'est guère étonnant, le \condition{VESAD tactile} bénéficiant à la fois d'un grand affichage et d'interactions stables et précises sur l'écran tactile. Malgré tout, notre première hypothèse est non supportée par nos résultats : le \condition{Téléphone} (non étendu) n'a pas été le moins performant des trois IHMs testées.

\figureETS[0.5]{Grubert2015_5}{
  Démonstration de la montre à écran étendu (condition SWRef) de \cite{Grubert2015}, similaire à notre technique \condition{VESAD tactile}.\\
  Adapté de \cite{Grubert2015Video}, à \SI{2}{\minute} \SI{4}{\second}.
}

On peut tout de même comparer le \condition{VESAD tactile} et le \condition{Téléphone}, car les techniques d'interactions utilisées sont les mêmes, seule change la taille de l'écran. Nous avons en effet observé que les participants s'habituaient très vite à fusionner l'affichage sur l'écran tactile avec l'affichage virtuel autour du téléphone. La différence importante de temps de complétion entre ces deux IHMs va donc dans le sens des résultats de \cite{Raedle2014}, qui avaient mis en évidence qu'une tâche de navigation dans un grand document devenait difficile avec un écran plus petit que celui d'une tablette (\SI{23.5x13.2}{\cm}). En revanche, cette différence contraste avec les résultats de \cite{Grubert2015}, qui avaient mesuré, dans une tâche de navigation, qu'une montre intelligente à l'écran étendu performait moins qu'un téléphone seul. Cependant, leur implémentation de l'écran étendu jouait en sa défaveur : le plan virtuel était bien aligné mais vu comme \SI{2.5}{\m} en arrière de la montre, en plus d'un champ de vision limité du visiocasque (\SI{30.5x17.2}{\degree}), d'un écart de résolution entre l'écran et le visiocasque et d'une latence importante entre les deux affichages \figrefp{Grubert2015_5}. Contrairement à \citeauthor{Grubert2015}, le \condition{VESAD tactile} a surpassé le \condition{Téléphone} dans notre étude. Il a été 30\% plus rapide que le \condition{Téléphone}, avec deux fois moins de zooms. En outre, les participants ont également jugé le \condition{Téléphone} mentalement plus exigeant que les deux autres techniques. Il est intéressant de noter qu'il n'y a eu aucune différence entre ces deux IHMs en termes de défilements. Ainsi, il nous semble qu'\emph{un VESAD bien conçu et implémenté est plus performant que l'écran seul}.

\figureLayoutETS{Raedle2014Movements}{%
  \subfigureETS[0.18]{Raedle2014_3}{Mouvements de \textquote{\texten{scanning}} lors des premiers essais (exploration du document).}%
  \figurehspace%
  \subfigureETS[0.18]{Raedle2014_4}{Mouvements lors des derniers essais (le document est maintenant connu et mémorisé par le participant).}%
}{
  Mouvements (en bleu) face à l'affichage mural d'un participant avec l'écran de la taille (en haut) d'une tablette ou (en bas) d'un téléphone. Les points rouges sont les cibles à atteindre.\\
  Tiré de \cite{Raedle2014}.
}

\figureETS[0.5]{Guiard2004_8}{
  Séquence d'actions typique pour atteindre une cible avec une IHM Pan+Zoom sur un ordinateur : l'utilisateur dézoome pour la repérer, puis zoome et défile vers cette cible pour pouvoir la pointer.\\
  Tiré de \cite{Guiard2004}.
}

En outre, nous avons observé des tactiques similaires de sélection des disques entre le le \condition{VESAD tactile} et le \condition{Téléphone}. Une première tactique consiste à zoomer suffisamment la grille pour pouvoir lire le texte sur les disques et à parcourir toutes les cellules de la grille jusqu'à trouver celle correspondant au disque sélectionné. \cite{Raedle2014} avaient également remarqué ces mouvements de \textquote{\texten{scanning}} \figrefp{Raedle2014_3} avec les tailles d'écrans correspondant à la tablette et au téléphone, quand les participants découvraient le document ; ces similitudes entre nos défilements statiques et leurs défilements physiques et dynamiques sont intéressantes. La seconde tactique est similaire a celle qui avait été décrite par \cite{Guiard2004} pour les IHMs Pan+Zoom sur ordinateurs : (i) un dé-zoom pour repérer une cible ou la placer dans la vue, (ii) suivi de zooms et défilement pour la rendre assez grosse pour finalement (iii) la sélectionner \figrefp{Guiard2004_8}. Les participants de notre expérience effectuaient ce mouvement pour lire le texte sur les disques. Cependant, \cite{Guiard2004} indiquaient à juste titre que la première phase n'était pas nécessaire si la difficulté était assez faible, ce que nous avons vu être souvent le cas pour le \condition{VESAD tactile}. Certains participants profitaient en effet du grand écran et de sa résolution suffisante pour le laisser zoomé, n'effectuant alors que des défilements. Cela est en accord avec les mouvements observés par \cite{Raedle2014} quand les participants connaissaient le document (quand ils en avaient une carte mentale) \figrefp{Raedle2014_4}.

Pourtant, plusieurs participants ont expliqué que le \condition{VESAD} leur semblait l'IHM avec le plus de potentiel. Nous les avons souvent vu tenter spontanément de toucher la grille la première fois qu'il ont vu l'écran étendu. De plus, ils se sont jugés aussi performants qu'avec le \condition{Téléphone} et l'ont un peu plus préféré que ce dernier. Ils l'ont aussi trouvé moins exigeant mentalement que le \condition{Téléphone}, ce qui indique que l'écran étendu donnait un avantage. Quelques participants ont aussi fait remarquer qu'ils préféraient toucher directement une cible sur l'écran virtuel plutôt que devoir la déplacer sur l'écran tactile avec le \condition{VESAD tactile}. Ajouter à cela la meilleure performance du \condition{VESAD tactile}, notre deuxième hypothèse est donc supportée : \emph{les participants ont préféré l'écran étendu au téléphone seul}.

Malgré tout, le \condition{VESAD} a été le moins performant. Cela est principalement dû au suivi de la main avec le Leap Motion qui manquait de stabilité et de précision. Par ailleurs, nos gestes de défilement et de zoom n'étaient pas bien conçus. Il s'est en effet avéré difficile de faire un déplacement précis de la main le long d'un plan virtuel en 3D. Nous l'avions pourtant vu dans notre étude pilote, mais nos corrections n'ont malheureusement pas été suffisantes. Ces deux facteurs jouant ainsi en sa défaveur, la comparaison avec les deux autres IHMs est donc difficile. Pourtant, la littérature a montré que la sélection d'une cible avec une main virtuelle était plus difficile que sur une surface tangible \citep{Chan2010, Jones2012, Argelaguet2013}. Les mauvaises notes sur l'exigence physique, la frustration ainsi que les longs temps passés avec un item sélectionné semblent aller dans ce sens. On peut donc tout de même estimer que nos notre troisième hypothèse soit supportée : avec un téléphone à écran étendu, \emph{les interactions sur l'écran tactile ont été plus performantes qu'avec une main virtuelle}.

\figureLayoutETS{LeapMotion2018}{%
  \subfigureETS[0.23]{LeapMotion2018_1}{La zone au-dessus d'une fenêtre	virtuelle permet de la manipuler.}%
  \figurehspace%
  \subfigureETS[0.23]{LeapMotion2018_2}{La manipulation est activée par un pincement sur le haut de la fenêtre virtuelle.}%
  \figurehspace%
  \subfigureETS[0.23]{LeapMotion2018_3}{Le reste de la fenêtre réagit seulement aux sélections par pointage d'un doigt.}%
}{
  Démontration de l'IHM du visiocasque de RA North Star. L'interface est transparente et la main virtuelle fait une excellente occlusion avec le contenu 3D.\\
  Adapté de \cite{LeapMotion2018}.
}

Enfin, de nos observations et des remarques des participants, nous voyons quelques pistes pour améliorer les gestes pour la main virtuelle. Nous avons omis une caractéristique essentielle de cette technique d'interaction : contrairement à une souris, ou un écran tactile, \emph{le suivi de la main est réalisé en continu et sans retours physiques}, en plus d'être en 6 DoFs. Si on ne se sert que de cette information, cela devient très contraignant pour l'utilisateur qui ne peut pas \emph{donner son intention} de vouloir interagir ou non avec l'objet qu'il \textquote{touche}. Ainsi, nous avons plusieurs fois observé des participants ne pas comprendre pourquoi ils avaient effectué une sélection ou une manipulation de la grille ou encore secouer la main comme pour enlever la sphère blanche. Comme en appuyant sur un bouton, il est nécessaire que l'utilisateur confirme une commande par une \emph{action discrète}. Un objet physique, incluant écran tactile ou souris, est soit touché ou non, il n'y a pas d'intermédiaires. \cite{LeapMotion2018} a récemment présenté son visiocasque de RA, utilisant une main virtuelle : si leur travail n'est pas une publication scientifique il reste néanmoins intéressant, car il montre des experts concevant et utilisant une IHM de RA. S'ils proposent tout de même des sélections avec un simple pointage du doigt (geste H11 de \cite{Piumsomboon2013}, \figref{LeapMotion2018_3}), mais avec beaucoup de retours visuels, les manipulations des fenêtres virtuelles s'activent avec un pincement (geste H2 de \cite{Piumsomboon2013}, \figref{LeapMotion2018_2}). Ce geste de pincement est également utilisé par le HoloLens pour faire une sélection. Appliqué à notre VESAD, ce geste permettrait alors d'effectuer les défilements en pinçant la grille : ainsi, elle suivra les mouvements de la main (sous contrainte de rester sur le même plan) jusqu'au relâchement du pouce et de l'index. Le zoom pourrait alors se faire simplement par la combinaison d'un appui long sur l'écran tactile et d'un geste de pincement sur la grille. Il est également probable que le zoom soit à privilégier sur l'écran tactile.

Toutefois, notre expérience comporte plusieurs limites. Premièrement, nous avons mal contrôlé la difficulté de notre tâche. Nous n'avons en effet pas pu mesurer de différence entre les conditions de \variable{TAILLE} et la \variable{DISTANCE}, contrairement à \cite{Liu2014}. Cela limite nos conclusions, car il aurait été intéressant de voir l'effet de ces quatre différents \textquote{niveaux} de difficulté sur la performance des trois IHMs. Nous avions dans notre étude pilote prévu 15 disques à classer au lieu de cinq et des tailles de texte beaucoup plus petites, qui s'étaient révélées impraticables avec le \condition{VESAD}. Nous avons alors placé trop proches nos deux conditions de \variable{TAILLE}. En outre, nous avons directement utilisé les valeurs \variable{DISTANCE} de la tâche de \citeauthor{Liu2014}. Il aurait été plus sage de faire une ou deux itérations supplémentaires pour tester la validité de la difficulté. Deuxièmement, nous l'avons décrit, le mauvais suivi de la main des participants et la mauvaise conception des techniques de navigation avec la main virtuelle ont défavorisé le \condition{VESAD}, qui a été l'IHM la moins performante des trois évaluées.

Troisièmement, nous avons malheureusement du faire un compromis sur la densité visuelle de notre visiocasque pour avoir un large champ de vision. Si cela a désavantagé le \condition{Téléphone}, nous avons cependant maintenu constants la latence, la densité visuelle, le champ de vision et le poids sur la tête des participants en faisant utiliser le visiocasque pour chaque \variable{IHM}, contrairement à \cite{Grubert2015}. Ces variables parasites contrôlées, nos comparaisons entre le \condition{Téléphone} et les deux autres IHMs ont alors une meilleure validité.

De manière plus générale, on peut voir l'utilisation nécessaire d'un visiocasque comme une limite. Nous avons en effet utilisé un visiocasque de RA vidéo, impliquant une vision dégradée (latence et densité visuelle) de l'environnement réel. Toutefois, grâce aux développements récents et importants dans l'industrie, nous anticipons un usage répandu à l'avenir de visiocasques de RA optiques plus légers, performants et avec large champ de vision. Ce type de visiocasque a l'avantage de ne pas dégrader la vision de l'environnement réel. Il est possible qu'un meilleur visiocasque change nos résultats, mais il nous semble que cette étude expérimentale supporte que \emph{certaines tâches bénéficieraient d'un écran étendu}.


\section{Conception d'une IHM pour un VESAD}
\label{sec:discussion_design}

Avant la conception d'une IHM, nous pensons crucial que le VESAD soit d'abord bien implémenté, c'est-à-dire avoir : (1) un excellent suivi avec 6 DoFs de l'écran physique à étendre, (2) un bon alignement entre l'écran virtuel et l'écran physique et (3) une synchronisation sans latence entre les deux écrans pour donner l'illusion d'un seul écran étendu. Notre bibliothèque de RA ArucoUnity répond à ces deux premières problématiques. Plusieurs participants nous ont cependant indiqué d'améliorer le troisième point. En outre, il nous semble important d'utiliser un visiocasque avec un grand champ de vision pour voir complètement l'écran étendu. Enfin, un visiocasque avec une bonne densité visuelle est préférable.

\figureLayoutETS{HandheldVESADZones}{%
  \subfigureETS[0.2]{HandheldVESADZones_1}{Interactions avec une main virtuelle, les interactions se font du côté de la main ne tenant pas le téléphone et éventuellement au-dessus et en dessous de l'écran physique.}%
  \figurehspace%
  \subfigureETS[0.2]{HandheldVESADZones_2}{Interactions via l'écran tactile : la partie virtuelle autour reste accessible par défilements.}%
}{
  Une IHM pour un téléphone à écran étendu contient des zones d'interactions (en vert), des zones pour les interactions moins fréquentes (en jaune) et des zones uniquement dédiée à la visualisation (en blanc).
}

Pendant la conception de l'IHM pour un écran étendu, il est essentiel de \emph{penser l'interface à travers la technique d'interaction} utilisée. En effet, comme l'ont très bien souligné \cite{Ens2014}, le temps et les erreurs de pointage augmentent pour les cibles plus lointaines en utilisant une main virtuelle. Cela a également été le cas avec le \condition{VESAD tactile} : les participants devaient effectuer beaucoup de défilements et zooms pour amener une cible sur l'écran. Sur un écran tactile seul ou avec une souris sur un écran de PC, l'espace de visualisation est facilement accessible, quand il dépasse facilement celui d'interaction avec un VESAD. Des techniques comme Go-Go permettent d'atténuer ce problème pour la main virtuelle en agrandissant l'espace d'interaction accessible par un gain sur les mouvements de la main \figrefp{Argelaguet2013GoGo}. \citeauthor{Ens2014} suggèrent de positionner des fenêtres virtuelles à équidistance de l'utilisateur dans une disposition sphérique \figrefp{Ens2014_3}. Cette idée serait intéressante à explorer pour un VESAD : nous avons observé plusieurs participants tourner le téléphone pour rapprocher une cible. Une autre approche est d'utiliser des pointeurs virtuels \citep{Argelaguet2013}.

\figureLayoutETS{Argelaguet2013GoGo}{%
  \subfigureETS[0.15]{Argelaguet2013_2}{Avec une technique de main virtuelle classique, seuls les objets à portée de main sont sélectionnables.}%
  \figurehspace%
  \subfigureETS[0.15]{Argelaguet2013_3}{La technique Go-Go propose de rediriger les mouvements de la main vers un espace de sélection plus grand.}%
}{
  Illustration de la technique Go-Go par rapport à une technique de main virtuelle.\\
  Adapté de \cite{Argelaguet2013}.
}

\figureETS[0.5]{Ens2014_3}{
  Démontration du Personal Cockpit : un ensemble de fenêtres 2D positionnées selon une sphère autour de l'utilisateur.\\
  Adapté de \cite{Ens2014Video}.
}

Ainsi, nous recommandons de \emph{favoriser les interactions seulement dans des zones facilement accessibles}, que nous résumons avec la \figref{HandheldVESADZones} : les zones blanches sont à réserver seulement pour l'affichage, les vertes sont les plus facilement accessibles tandis que les jaunes, plus difficiles d'accès, sont à réserver pour les interactions peu fréquentes. Ainsi, avec une main virtuelle, l'interface est plutôt asymétrique \figrefp{HandheldVESADZones_1} : elle est alignée avec le téléphone qui est tenu avec une main tandis que les interactions se font avec l'autre main ; dès lors, quand le téléphone est tenu de la main gauche, l'espace d'interaction se trouve sur sa droite (et inversement). En outre, nous déconseillons les interactions du côté de la main tenant le téléphone (à gauche de la main gauche pour un droitier) qui demandent de croiser les mains. Nous améliorons alors notre basculement d'affichage wrist \figrefp{HandheldVESADApp}, en plaçant les applications sélectionnables à droite du téléphone, à portée de la main \figrefp{HandheldVESADApps3}. À l'inverse, quand l'écran tactile est utilisé, par exemple quand le téléphone est tenu à deux mains ou si un utilisateur préfère utiliser une seule main, l'interface est symétrique \figrefp{HandheldVESADZones_2} : l'écran tactile est bien sûr accessible, ainsi que l'espace immédiatement autour, via des défilements.

\figureETS[0.5]{HandheldVESADApps3}{
  Remaniement de wrist \figrefp{HandheldVESADMap} : les applications sur lesquelles basculer sont placées à droite du téléphone, facilement sélectionnables. Les notifications sont placées au-dessus de l'écrans, car elles sont peu accédées.
}

Enfin, tout comme la disposition d'un site web doit s'adapter (\texten{responsiveness}) à la taille d'écran des différents appareils qui peuvent y accéder (par exemple : téléphone intelligent, tablette, ordinateur), \emph{une IHM pour un VESAD doit s'adapter à différents scénarios d'interaction}. Ainsi, la navigation physique avec une main virtuelle et la navigation virtuelle via l'écran tactile doivent toutes deux être supportées, au choix de l'utilisateur. De plus, la disposition du contenu doit s'adapter dynamiquement au contexte d'utilisation (comment est tenu le téléphone) et au type d'interaction utilisé.


\section{Directions futures}
\label{sec:discussion_future}

En premier lieu, la technique de main virtuelle de la condition \condition{VESAD} devrait être améliorée. L'implémentation de meilleures techniques de suivi de pose de la main serait à explorer, par exemple en utilisant une caméra de profondeur, plus adaptée à la résolution de ce problème qu'une caméra RVB classique \citep{Taylor2016}. Plus simplement, d'autres placements du Leap Motion pourrait améliorer la détection : par simplicité, nous avons placé ce petit boitier sous la caméra, mais il est probable que son angle de vue ne soit pas le meilleur ainsi. Le placer au-dessus du casque, légèrement incliné vers le sol pourrait améliorer le suivi.

Pour éviter de trop nombreuses comparaisons dans notre expérience, nous n'avons pas évalué des techniques présentes dans la littérature et l'industrie. Le HoloLens utilise un pointeur virtuel suivant les mouvements de la tête, les sélections se faisant avec un geste de pincement (geste H2 de \cite{Piumsomboon2013}) face au casque. \cite{Piumsomboon2014} avaient trouvé que les commandes vocales étaient plus efficaces qu'utiliser une main virtuelle pour changer l'échelle d'objets en 3D. Il serait alors intéressant de reproduire notre étude expérimentale pour les comparer à d'autres techniques d'interactions, en incluant également le geste de pincement virtuel que nous venons de proposer pour la main virtuelle. Une première variante du \condition{VESAD tactile} utiliserait l'écran physique comme pavé tactile déplaçant un curseur sur tout l'écran étendu. Une seconde variante consisterait à contrôler un pointeur avec le regard et y rediriger les actions faites l'écran tactile \figrefp{Pfeuffer2014} : cet agrandissement de l'espace d'interaction tout en gardant les entrées utilisateurs sur l'écran tactile permettrait de rapidement et facilement accéder à des éléments distants. Cette dernière pourrait aussi s'appliquer au pointeur virtuel du HoloLens \figrefp{Pfeuffer2017}.

\figureLayoutETS{Pfeuffer}{%
  \subfigureETS[0.2]{Pfeuffer2014}{Gaze+Touch : les interactions sur l'écran tactile sont redirigées vers la cible du regard.}%
  \figurehspace%
  \subfigureETS[0.2]{Pfeuffer2017_2}{Gaze+Pinch : un geste de pincement sélectionne la cible (en 3D) du regard.}%
}{
  Interactions avec un pointeur virtuel suivant le regard.\\
  Adapté de a) \cite{Pfeuffer2014} et b) \cite{Pfeuffer2017}.
}

De même, d'autres types de tâches devraient être évalués, comme un scénario de multi-tâche, similaire à \cite{Ens2014}, plus proche des usages quotidiens avec un téléphone intelligent. La faisabilité des deux nouvelles techniques d'interactions que nous avons proposée devrait aussi être évaluées, notamment wrist, dans une tâche de commutation d'applications. Slide-to-hang pourrait être implémenté, en mode multi-fenêtres dans le VESAD, avec un geste de pincement \figrefp{LeapMotion2018_2} pour manipuler les fenêtres.

Finalement, nous pensons que notre concept de téléphone à écran étendu mériterait d'être exploré. En particulier, nous nous demandons :
\begin{itemize}
  \item Quelle taille devrait faire l'écran étendu ? Dans notre expérience, nous l'avons déterminée arbitrairement ; s'il n'y a virtuellement pas de limites, il est probable que certaines tailles d'écrans soient plus optimales en fonction de la tâche et du contexte d'utilisation.
  \item Explorer les changements d'usage et de contexte, notamment les différents basculements, automatiques et manuels, entre la main gauche, droite ou les deux tenant le téléphone.
  \item Concevoir et évaluer des applications, comme celles que nous avons présentées au \autoref{ch:concept} : par exemple, carte de navigation utilisant tout l'écran étendu, navigateur web ou application de messagerie fractionné en plusieurs onglets ou encore galerie d'images.
  \item Expérimenter des applications pour l'industrie : par exemple pour aider un employé d'un entrepôt à trouver un produit, un superviseur sur un site de construction ou un technicien sur une machine à réparer. Toutes ces professions pourraient bénéficier d'un écran large tenant dans la main, par exemple pour afficher une vue d'ensemble de leur lieu de travail, tout en utilisant des interactions tactiles précises, stables et connues. La technique slide-to-hang permettrait de basculer vers un usage main libre au besoin.
\end{itemize}