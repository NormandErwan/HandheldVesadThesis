\chapter{Développement de l'expérience}
\label{annex:experiment}

\section{Introduction}
L'expérience a demandé un long effort de conception et de développement de plusieurs semaines. Cette annexe décrit ce projet, permettant de mieux comprendre cette expérience et pouvoir la reproduire. Le code source du projet est disponible en ligne, sous la license libre MIT, à l'adresse suivante: \url{https://github.com/NormandErwan/master-thesis-experiment}.\\
Les principales tâches étaient de concevoir la tâche expérimentale, les techniques d'interactions, la synchronisation entre le visiocasque et le téléphone, et pouvoir mesurer les participants. Les différents choix techniques, la gestion de la réalité augmentée et le suivi des mains sont décrits dans le \refsection{ch:methodology}. L'expérience est décrite dans la \refsection{sec:experiment_description}.

\section{Réalisation de la tâche expérimentale}

Enfin, il est important d'avoir une très bonne luminosité dans la pièce pour que les marqueurs soient bien détectés par les caméras.

\section{Réalisation des techniques d'interaction}

\section{Synchronisation entre le visiocasque et le téléphone}

\section{Mesures des participants}