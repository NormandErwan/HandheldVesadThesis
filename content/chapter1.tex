%%- Premier chapitre de démonstration -%%
\chapter{Test de long titre de Chapitre, avec retour à la ligne. Lorem ipsum dolor sit amet, consectetur adipiscing elit. Pellentesque justo justo, porta sagittis feugiat eget, ornare rhoncus ligula. Nunc non odio sed lacus rutrum rhoncus.}


\section{Tests de mise en page}

Dans cette section, différents environnements de mise en page sont présentés. 

\subsection{Test des listings}

Présentation des principaux listings: les énumations et les listes.


\subsubsection{Énumérations: environement enum}

Test de l'environment enum:
\begin{enumerate}
 \item test 1
 \item test 2
\end{enumerate}

\subsubsection{Listes: environement itemize}

Test de l'environement itemize 
\begin{itemize}
 \item test 1
 \item test 2
\end{itemize}

\subsection{Test des équations}

Mise en page des équations

\begin{equation}
   \beta = 8
\end{equation}

\begin{equation}
   \bm{\gamma} = \alpha \times 3
\end{equation}

\section{Seconde section}

Exemple de seconde section pour illustrer la mise en page de la table des matières


%%- Deuxiemme chapitre de démonstration -%%
\chapter{Ajout d'un second chapitre}

\section{Test de mise en page d'un tableau}

Les tableaux sont soumis aux mêmes contraintes que les figures, en dehors de la position de la légende qui doit être au dessus.


\begin{table}
		\parbox{0.65\textwidth}{\caption{Test de longue légende pour un tableau, avec retour à la ligne.}} % Contrainte manuelle de la largeur de la légende
		\begin{tabular}{|c|c|c|c|c|c|c|c|}
		\hline
			{\bf titre} & {\bf titre} & {\bf titre} & {\bf titre} & {\bf titre} & {\bf titre} & {\bf titre} & {\bf titre} \\
	  \hline
			blá & blá & blá & blá & blá & blá & blá & blá \\
	  \hline
			blá & blá & blá & blá & blá & blá & blá & blá \\
	  \hline
			blá & blá & blá & blá & blá & blá & blá & blá \\
	  \hline
			blá & blá & blá & blá & blá & blá & blá & blá \\
	  \hline
			blá & blá & blá & blá & blá & blá & blá & blá \\
	  \hline
			blá & blá & blá & blá & blá & blá & blá & blá \\
	  \hline
		\end{tabular}
\end{table}


\section{Test des références}

\subsection{Références à la bibliographie}

Citation d'une référence de la bibliographie \cite{Arica2002}.

\subsection{Références à un label du document}

Référence à une Figure associée à un label: Figure \ref{fig:vueEts}.

\subsection{Références à des adresses}

\subsubsection{Test de href}

Utilisation de href, pour intégrer un lien dans une portion de texte:
\href{http://www.etsmtl.ca/Etudiants-actuels/Cycles-sup/Realisation-etudes/Guides-gabarits}{Lien vers la page des gabarits de l'ÉTS}.

\subsubsection{Test de url}

Utilisation de url pour citer un lien cliquable:
\url{http://www.etsmtl.ca/Etudiants-actuels/Cycles-sup/Realisation-etudes/Guides-gabarits}.