\chapter{Concept}
\label{ch:concept}

\section{Présentation}

Nous proposons de concevoir un prototype de téléphone intelligent tenu en main dont l'écran est étendu par RA et de l'évaluer expérimentalement par rapport à un téléphone seul non étendu. Nous appelons cette IHM : un téléphone augmenté par un \foreignlanguage{english}{Virtuality Extended Screen-Aligned Display} (VESAD), comme le montre la \reffigureETS{HandledVESAD2018.png}.

\figureETS{HandledVESAD2018.png}{
  Photomontage d'un téléphone tenu en main augmenté par un VESAD : un affichage en RA étendant l'écran et aligné avec le téléphone. L'écran étendu permet d'afficher de multiples fenêtres d'affichage autour de l'écran tactile.
}

Appliquer Ens2014 - Ethereal Planes.

Deux modes (comme le HoloLens) :\\
vue étendue (l'écran tactile et l'écran étendu ne forme qu'une seule surface)\\
vue multi-fenêtres (une app sur l'ecran tactile et plusieurs apps autour)

\section{Applications potentielles}
Déplacer du contenu entre l'écran tactile et l'ecran étendu (comme une application)

Multi-apps (notes + wikipedia + youtube)

Application étendue multi-fenêtres (messagerie : contacts à gauche, conversation au centre, fichiers échangés à droite)

Une carte étendue

Notifications en profondeurs\\
Justification : Robertson1998 - Data mountain\\
Ça fonctionne de s'appuyer sur la mémoire spatiale humaine : justifie notre concept d'organiser l'espace autour du cellulaire comme une idée pertinente. On pourrait organiser les applications dans le plan et en profondeur le long d'un plan qui part de la base du cellulaire avec une pente positive : cela permet de voir toutes les applications, comme sur des gradins d'un stade (c'est ce que Data Moutain suggère comme placement des app dans l'espace)

Copier/coller : zone en bas de l'écran tactile