\chapter{Étude expérimentale}
\label{ch:experiment}

\section{Tâche expérimentale}
\label{sec:experiment_task}

Nous souhaitons évaluer dans cette expérience les avantages d'un téléphone dont l'écran étendu par un visiocasque de RA à large champ de vision, ainsi que comparer deux techniques d'interaction : navigation, pointage et sélection avec l'écran tactile ou avec une main virtuelle.

Plusieurs tâches auraient pu convenir à cette expérience. Comme le VESAD est une technique relativement nouvelle et peu étudiée \citep{Grubert2015}, et la technique d'interaction de la main virtuelle peu maîtrisée \citep{Argelaguet2013, Piumsomboon2013}, nous avons besoin d'une tâche fondamentale demandant de la navigation, dont des zooms, et des sélections \citep{Bowman2004}. Nous avons alors envisagé une tâche de pointage de Fitts \citep{Soukoreff2004, Berge2014}, où l'utilisateur doit chercher et sélectionner des cibles, de navigation dans un large document, comme une carte \citep{Baudisch2002, Raedle2014}, ou encore un scénario de multi-tâches où le participant doit travailler avec plusieurs fenêtres \citep{Czerwinski2003, Ens2014}.

Nous avons finalement décidé de répliquer la tâche de classification de \cite{Liu2014} : sa validité externe est plus importante (plus réaliste) qu'une pure tâche de pointage, demande beaucoup de navigations et de reconstruction mentale comme un large document tout en demandant de sélectionner et manipuler le contenu, enfin est plus facile à implémenter et à contrôler qu'un scénario multi-tâches. Cette tâche consistait en une grille de plusieurs cellules contenant des disques à classer \figrefp{Liu2014}. Les participants devaient déplacer les disques mal classés dans la bonne cellule, avec des disques de même type. Une lettre inscrite en leur centre permettait de connaître leur type et les disques mal classés étaient colorés en rouge : cette indication visuelle facilitait la recherche et équilibrait avec le temps consacré au classement. Une tâche plus réaliste demanderait d'examiner en détail chaque disque pour savoir les classer (\citeauthor{Liu2014} prennent l'exemple de catégoriser les articles d'une conférence). Aussi, la petite taille des lettres sur les disques et la grandeur de la grille imposaient aux participants des zooms et défilements. Aussi, la grille était grande pour imposer des défilements Enfin, les classements se faisaient par un pointage vers le disque puis une sélection par un bouton, puis un second pointage et sélection dans la cellule. Le but de la tâche peut être résumé aux participants avec la consigne de [traduction ]\textquote{tout mettre en vert} \citep{Liu2014}.

Notre grille comporte $3 \times 5$ cellules rectangulaires, pouvant chacune contenir six disques maximum. La grille est générée aléatoirement à chaque nouvel essai, avec cinq disques par cellules dont au moins quatre correctement classés, et cinq disques mal classés au total dans la grille \figrefp{TaskGrid}. Il y a donc 15 cellules $\times$ 5 disques/cellule $=$ 75 disques dans la grille. Au début de chaque essai (pour toutes les conditions), elle est affichée à l'échelle, la taille d'une cellule étant celle de l'écran du téléphone que nous avons utilisé, soit \SI{68x121}{\mm} (un disque mesure alors \SI{33x33}{\mm}) \autorefp{sec:experiment_material}. Ainsi, la grille a une forme approximativement carrée, un peu plus grande (\SI{34x36.3}{\cm}) que la condition tablette de \cite{Raedle2014}. La vue de la grille est de taille limitée dans toutes les conditions : elle est soit limitée par l'écran du téléphone, soit nous donnons une taille fixée à l'écran étendu. Dans ce dernier cas, nous avons arbitrairement choisi de limiter l'écran étendu à la taille de la grille au début de l'essai ; ainsi, si un participant zoome la grille avec l'écran étendu, sa vue sera tronquée à \SI{34x36.3}{\cm}.

\figureETS[0.55]{TaskGrid}{
  La grille de notre tâche expérimentale au début d'un essai dans la condition \variable{TAILLE}=\condition{Grand}.
}

La tâche de \cite{Liu2014} opérationnalise un scénario d'allocation de ressources (par exemple, chaque catégorie a une capacité limitée). Le texte y est alors une variable indépendante importante. N'étudiant pas cette problématique, nous avons décidé d'utiliser plus simplement une seule lettre par cellule, de A à O. Chaque disque \textquote{appartient} donc à une cellule et ne peut y être correctement classé que dans celle-ci. Nous colorons également les items correctement classés en vert et mal classés en rouge. Nous colorons en outre un disque sélectionné en bleu \figrefp{ExperimentPhoneInArOut}. De sorte qu'en participant doit lire la lettre d'un disque vert pour savoir le type de la cellule.

Nous avons placé le code source de l'expérience en ligne (\url{https://github.com/NormandErwan/MasterThesisExperiment}), sous la licence libre BSD-3, pour qu'il pouvoir être reproduit ou réutilisé au besoin. Nous l'avons développé et testé pour Unity 2017.


\section{Plan expérimental}
\label{sec:experiment_design}

\figureLayoutETS{ExperimentTechniques}{%
  \subfigureETS[0.24]{ExperimentPhoneOnly}{Condition \condition{Téléphone}}%
  \figurehspace%
  \subfigureETS[0.24]{ExperimentPhoneInArOut}{Condition \condition{VESAD tactile}.}%
  \figurehspace%
  \subfigureETS[0.24]{ExperimentMidAirInArOut}{Condition \condition{VESAD}.}%
}{
  La grille de notre tâche expérimentale pour chaque \variable{IHM}. Pour \condition{VESAD tactile}, la main de l'utilisateur est cachée par l'écran étendu et un disque est sélectionné (en bleu). Pour \condition{VESAD}, une sphère blanche indique la position repérée de l'index de l'utilisateur, une croix sur la grille est la projection de cette sphère et un segment noir les relie ; ils deviennent bleus quand la sphère touche la grille.
}

Nous avons manipulé simultanément les variables indépendantes suivantes :
\begin{itemize}
  \item \variable{IHM} : l'IHM évaluée, soit utilisation ou non d'un écran étendu, et la technique d'interaction :
  \begin{itemize}
    \item \condition{Téléphone} : affichage et interactions seulement sur le téléphone \figrefp{ExperimentPhoneOnly} ;
    \item \condition{VESAD tactile} : affichage sur l'écran étendu et interactions sur le téléphone \figrefp{ExperimentPhoneInArOut} ;
    \item \condition{VESAD} : affichage sur l'écran étendu et interactions dans la partie virtuelle de l'écran avec une main virtuelle \figrefp{ExperimentMidAirInArOut}.
  \end{itemize}
  \item \variable{TAILLE} : la taille du texte des disques :
  \begin{itemize}
    \item \condition{Grand} : \SI{12}{\pt}, soit \SI{4.22x4.22}{\mm} ou 12\% du diamètre du disque ;
    \item \condition{Petit} : \SI{10}{\pt}, soit \SI{3.51x3.51}{\mm} ou 10\% du diamètre du disque.
  \end{itemize}
  \item \variable{DISTANCE} : la distance moyenne des disques à classer avec leurs cellules respectives :
  \begin{itemize}
    \item \condition{Proche} : entre 1,25 et 1,45 ;
    \item \condition{Loin} : entre 2,5 et 2,75.
  \end{itemize}
  \item \variable{GROUPE} : l'ordre de passage des IHMs \autorefp{tab:experiment_groups}.
\end{itemize}
\medskip

La variable indépendante \variable{IHM} combine les deux affichages et les deux techniques d'interaction que nous souhaitons évaluer. La combinaison affichage sur le téléphone seul et interaction avec une main virtuelle ne nous intéressait pas. \cite{Jones2012} l'ont déjà été évaluée contre le téléphone seul \autorefp{subsec:litterature_ar_hci_interactions}.

La \variable{TAILLE} et la \variable{DISTANCE} nous permettent de contrôler la difficulté de la tâche. En réduisant la \variable{TAILLE} de la lettre dans chaque disque, on impose au participant d'effectuer des zooms plus importants pour les lire, ce qui l'empêche alors de voir beaucoup de disques à la fois. En augmentant la \variable{DISTANCE}, on impose plus de recherches au participant pour trouver la cellule correspondant à un disque à classer. Dans les deux cas, on lui demande un effort de mémoire plus important.

Comme \citeauthor{Liu2014}, nous utilisons la distance euclidienne pour calculer la \variable{DISTANCE} entre un disque mal classé et sa cellule respective. Nous considérons dans le calcul qu'une cellule fait une unité de largeur et une unité de hauteur : il y a donc une distance de 1 entre deux cellules adjacentes. La variable indépendante donne alors le nombre de cellules d'écart entre un disque mal classé et sa cellule respective, par exemple entre 2,5 et 2,75 cellules d'écart en moyenne pour la condition \condition{Loin}. Les valeurs que nous utilisons sont celles de \cite{Liu2014}. À la génération de la grille, on vérifie qu'elle respecte la condition courante, autrement on en régénère une nouvelle. Nous avions mené une étude pilote avec trois participants, demandant de classer dix disques avec de plus petites tailles de texte. Cependant, cette configuration rendait les essais particulièrement éprouvants et longs à compléter. C'est pourquoi nous avons réduit le nombre de disques à classer et augmenté la taille du texte (tout en la gardant suffisamment petite pour que les participants aient besoin de zoomer).

Notre plan expérimental est quasi complet. Les variables indépendantes \variable{IHM}, \variable{TAILLE} et \variable{DISTANCE} sont croisées : les participants passent à travers toutes les conditions (les combinaisons des valeurs des variables). Cependant, pour contrôler un effet d'apprentissage parasite à travers les IHMs, les participants sont partagés en trois \variable{GROUPE} emboîtés : chaque groupe passe à travers les conditions \variable{IHM} dans un ordre différent, suivant un carré latin $3 \times 3$ \autorefp{tab:experiment_groups}. L'ordre de passage de la \variable{TAILLE} et de la \variable{DISTANCE} est par contre fixe : du plus simple (avec les conditions \condition{Grand} et \condition{Proche}), au plus difficile (\condition{Petit} et \condition{Loin}). Nous avons donc mesuré 12 participants $\times$ 3 IHMs $\times$ 2 distances $\times$ 2 tailles = 144 observations. Pour réduire la variabilité de nos résultats, chaque participant a répété deux essais par condition, que nous avons ensuite moyennés pour avoir une observation par participant, et non par essai \citep[p. 24]{Dragicevic2016}.

\begin{tableETS}{tab:experiment_groups}{Ordre de passage pour chaque \variable{GROUPE} de chaque \variable{IHM} suivant un carré latin.}
  \begin{tabular}{| c | c | c | c |}
    \hline
    \textbf{Groupe} & \textbf{IHM 1} & \textbf{IHM 2} & \textbf{IHM 3}\\
    \hline
    1 & \condition{Téléphone} & \condition{VESAD tactile} & \condition{VESAD} \\
    \hline
    2 & \condition{VESAD tactile} & \condition{VESAD} & \condition{Téléphone} \\
    \hline
    3 & \condition{VESAD} & \condition{Téléphone} & \condition{VESAD tactile} \\
    \hline
  \end{tabular}
\end{tableETS}

Enfin, les participants ont porté le visiocasque de RA \autorefp{ch:methodology} dans toutes les conditions, même dans la condition \condition{Téléphone} qui ne bénéficie pas de la RA, pour garder identique le poids sur la tête ainsi que la même résolution et latence de l'image.


\section{Techniques d'interactions}
\label{sec:experiment_interactions}

Les interactions des participants consistent à déplacer les disques pour les classer et naviguer dans la grille (par défilements et zooms). Le déplacement d'un disque consistait à, quelle que soit la condition \variable{IHM} : (1) pointer puis sélectionner le disque puis (2) sélectionner un espace libre dans une cellule. Une deuxième sélection sur le disque sélectionné permettait de le désélectionner, tandis que sélectionner un second disque désélectionnait automatiquement le précédent. Si un disque est déposé dans une mauvaise cellule, il est alors coloré en rouge, et cela est compté comme une erreur, augmentant d'un le nombre de disques à classer pour l'essai courant.

Avec le \condition{Téléphone} et le \condition{VESAD tactile}, les interactions se font via l'écran tactile : la sélection d'un disque se fait par un \texten{tap} \figrefp{Wobbrock2009_1}, le défilement par un \texten{pan} \figrefp{Wobbrock2009_2} et le zoom par un \texten{pinch} \figrefp{Wobbrock2009_3}. Ces gestes, recommandés par \cite{Wobbrock2009}, sont standards et connus des utilisateurs de téléphones intelligents (par exemple pour Android : \url{https://material.io/design/interaction/gestures.html}). Il s'agit d'un défilement statique \citep{Mehra2006}. Ainsi, l'écran étant étendu ou non, il faut donc d'abord amener sur l'écran tactile le disque ou la cellule que l'on souhaite sélectionner, en déplaçant la grille par un défilement ou un zoom.

\figureLayoutETS{Wobbrock2009}{%
  \subfigureETS[0.15]{Wobbrock2009_1}{Sélection par un \texten{tap} : un appui bref ($<\SI{500}{\ms}$) de l'index.}%
  \figurehspace%
  \subfigureETS[0.15]{Wobbrock2009_2}{Défilement par un \texten{pan} : un déplacement de l'index.}%
  \figurehspace%
  \subfigureETS[0.15]{Wobbrock2009_3}{Zoom par un \texten{pinch} : un déplacement de l'index et du pouce, le changement de distance entre les deux doigts contrôlant le zoom.}%
}{
  Les différents gestes utilisés par les participants sur l'écran tactile du téléphone pour le \condition{Téléphone} et le \condition{VESAD tactile}.\\
  Adapté de \cite{Wobbrock2009}.
}

Pour le \condition{VESAD}, les interactions se font via une main virtuelle, suivant les mouvements avec 6 DoFs de la main dominante du participant. La sélection se fait avec un appui long sur le disque ou la cellule à sélectionner \figrefp{Piumsomboon2013_4}, le défilement et le zoom par un \texten{pan} à travers la grille \figrefp{Piumsomboon2013_5}. Ces trois techniques d'interactions utilisent donc le geste simple de pointer et déplacer l'index dans le contenu virtuel ; il s'agit du geste H11 que \cite{Piumsomboon2013} recommandent \figrefp{Piumsomboon2013_1}. Pendant le défilement, la grille suit simplement les déplacements de l'index. Pour le zoom, la grille grossit quand l'index la \textquote{tire} vers (quand il s'éloigne de) l'écran, mais se réduit quand il la \textquote{pousse} vers (quand il se rapproche de) l'écran.

Nous avions tout d'abord utilisé un \texten{tap} pour la sélection et un geste de \texten{pinch} pour le zoom dans l'étude pilote, mais il était difficile de faire de tels appuis brefs avec précision et, l'index et le pouce étant proches, il était courant de déclencher involontairement un zoom pendant un défilement. Nous avons donc ajouté simplement ajouté trois boutons en bas de l'écran du téléphone pour choisir entre activer la sélection, le défilement ou le zoom : le participant appuie sur le bouton correspondant avec le pouce de sa main tenant le téléphone pour changer le mode d'interaction. Ces boutons ne cachent pas les disques de la cellule \figrefp{ExperimentMidAirInArOut} tout en restant assez gros (\SI{22x22}{\mm}). \cite{Piumsomboon2013} n'ayant pas proposé de geste pour déplacer du contenu, nous utilisons celui nous semblant le plus facile, en utilisant un index. 
En outre, nous avons ajouté des retours visuels de la main virtuelle au participant pour le \condition{VESAD} comme le conseille \cite{Chan2010}. Nous avons ajouté des retours continus : une sphère blanche pour indiquer la position repérée de l'index de la main dominante, la projection de cette sphère sur la grille sous forme d'une croix noire, ainsi qu'un segment noir les reliant tous les deux \figrefp{ExperimentMidAirInArOut}. Nous avons aussi ajouté un retour discret pour confirmer quand l'index touche la grille : la sphère, la croix et la ligne sont alors colorés en bleu.

\figureLayoutETS{Piumsomboon2013}{%
\subfigureETS[0.2]{Piumsomboon2013_4}{Sélection par un \texten{touch} : un appui long ($>\SI{500}{\ms}$) de l'index à travers le disque ou la cellule.}%
\figurehspace%
\subfigureETS[0.2]{Piumsomboon2013_5}{Défilement et zoom par un \texten{pan} à travers la grille.}%
}{
Les différents gestes utilisés par les participants avec la main virtuelle pour le \condition{VESAD}.\\
Adapté de \cite{Piumsomboon2013}.
}

Pour les trois IHMs, le défilement a un gain de $1:1$, tandis que le zoom est centré sur le téléphone et non sur la grille, comme le conseille \cite{Guiard2004} et que nous avons vérifié en développant l'expérience : [traduction] \textquote{The focus point is generally coincident with the center of the view, more rarely with the cursor position.}.

Enfin, dans les conditions avec le \condition{VESAD tactile} ou le \condition{VESAD}, la grille est affichée par-dessus les images de la caméra \autorefp{subsec:prototype_operation}. La main dominante utilisée pour les interactions est donc cachée en permanence par la partie virtuelle de la grille même quand le participant la place en avant du téléphone \figrefp{ExperimentPhoneInArOut}. Cela peut causer un sentiment étrange en créant un conflit avec les autres indices de profondeurs, l'expérience pouvant alors devenir inconfortable. Nous avons donc réduit ce problème d'occlusion d'une part en rendant la grille semi-transparente toujours laisser visible la main, proche de la solution de \cite{Piumsomboon2014} et, d'autre part, en indiquant la position repérée de l'index par une sphère blanche pour le \condition{VESAD}. Aucun participant n'a fait de remarque à ce sujet.


\section{Matériel}
\label{sec:experiment_material}

Nous avons bien évidemment utilisé notre visiocasque de RA, conçut avec un large champs de vision pour cette expérience et décrit au \autoref{ch:methodology}. Pour le faire fonctionner, nous avons utilisé un ordinateur de bureau sous Windows 10, avec un processeur Intel Core i5 7400 (\SI[product-units = single]{4x3.0}{\GHz}), \SI{8}{\giga\byte} DDR4 de mémoire vive, une carte graphique NVIDIA GeForce GTX 1060 de \SI{6}{\giga\byte}. Pour le téléphone, nous avons utilisé un Xiaomi Redmi Note 4 : sous Android 7, il est récent et léger, à faible prix et possède une bonne puissance de calcul ainsi qu'écran \SI{1920x1080}{\px} de \SI{5.5}{\inch}. Le suivi du téléphone était fait grâce à notre bibliothèque de RA ArucoUnity \autoref{sec:aruco_unity} avec trois marqueurs imprimés sur une planche rigide fixée à l'arrière du téléphone \figrefp{ExperimentSmartphone}.

\figureETS[0.3]{ExperimentSmartphone}{
  Le téléphone était suivi avec 6 DoFs par des marqueurs grâce à ArucoUnity.
}

Pour la localisation des mains nous avons utilisé un Leap Motion : c'est un dispositif peu dispendieux, particulièrement utilisé pour concevoir des IHMs avec une main virtuelle pour les visiocasques de RV et très bien intégré avec les moteurs de jeu Unity et Unreal Egine. Nous l'avons fixé la face avant du visiocasque sous la caméra \figrefp{ArRift_1}, et se connecte simplement au PC par USB 3.0.

Enfin, nous avons développé la bibliothèque DevicesSyncUnity pour synchroniser le visiocasque avec le téléphone (\url{https://github.com/NormandErwan/DevicesSyncUnity}). Basée sur la bibliothèque de mise en réseau haut-niveau UNet, fournie avec Unity, elle nous permet de facilement synchroniser les affichages sur le visiocasque et sur le téléphone en fonction les actions de l'utilisateur, donnant le sentiment d'interagir avec un seul appareil. Il existe une latence perceptible de quelques centièmes de seconde, mais aucun participant n'en a fait la remarque.


\section{Procédure de l'expérience}
\label{sec:experiment_procedure}

Nous avons tout d'abord demandé à chaque participant de lire soigneusement une copie imprimée du formulaire d'information et de consentement, en répondant à leurs questions. Nous avons ensuite ré-expliqué à l'oral les points les plus essentiels : pour qui est faite la recherche, le déroulé de l'expérience, la tâche, la garantie de l'anonymat de leurs données, enfin la possibilité de prendre une pause, de quitter l'expérience ou de demander la suppression de leurs données n'importe quand. Si la personne souhaitait participer, nous lui demandions de signer le formulaire d'information et de consentement, lui en proposions une copie, et lui faisions remplir le pré-questionnaire.

Nous faisions ensuite tester le visiocasque au participant, d'abord en mode RV, en utilisant l'écran d'accueil de l'Oculus, car moins susceptible de provoquer des nausées : il n'y a aucune latence perceptible dans ce mode et une meilleure densité visuelle de l'image \autorefp{subsec:solution_discusion}. Puis en activant les caméras, nous demandions au participant de regarder autour de lui et de bouger ses mains pour s'habituer à la latence et à la dynamique plus restreinte de l'image (des ombres vues par l'\oe il humain sont totalement noires dans le visiocasque, comme on peut le voir sur la \figref{ExperimentPhoneOnly}). Nous donnions alors le téléphone dans la main non-dominante du participant pour qu'il puisse démarrer les essais. Il y avait, pour chaque IHM, un essai d'entraînement au début (pas de consignes données et aucune mesures) et une pause obligatoire à la fin. L'ensemble des essais durait environ 50 minutes. Avant chaque essai, nous demandions au participant d'\textquote{aller le plus vite possible en faisant le moins d'erreurs possibles}. Nous lui précisions qu'une erreur était comptée seulement s'il déplaçait un disque dans une mauvaise cellule.

Une fois les essais terminés, nous faisions remplir le post-questionnaire pour recueillir les mesures subjectives, ainsi que le formulaire de compensation et leur donnions 20\$ en échange de leur participation.

L'\appendixref{experiment_forms} contient les deux formulaires et les deux questionnaires.


\section{Mesures}
\label{sec:experiment_measures}

Dans notre pré-questionnaire, nous demandions d'abord la catégorie d'âge (catégories de cinq ans), le sexe et les problèmes de vues éventuel du participant. Nous demandions ensuite leur main dominante et la main utilisant la souris : quelques participants gauchers utilisaient leur main droite avec la souris, mais on tout-de-même préféré tenir le téléphone avec leur main non dominante, après essais. Nous demandions enfin leur utilisation de l'ordinateur, de logiciels 3D et leurs expériences passées en VR et RA. Nous ne nous sommes pas servis de ces mesures dans nos analyses.

Nous avons effectué les mesures des essais seulement sur l'ordinateur, comme nous l'avions synchronisé avec le téléphone. Nous avons tout d'abord utilisé, comme principale mesures de la performance, le temps de complétion de chaque essai ainsi que le nombre d'erreurs (les disques placés dans une mauvaise cellule par le participant). Nous avons compté, en complément des erreurs, le nombre de disques sélectionnés pendant un essai : pendant sa recherche, un participant peut oublier le disque qu'il avait sélectionné ou décider de changer de disque à classer ; une sélection d'un nouveau disque désélectionne automatiquement le précédent, sans que ce soit considéré comme une erreur (cela était précisé aux participants). Le nombre de sélections peut être résumé ainsi : sélections $=$ 5 disques à classer $+$ erreurs $+$ sélections annulées.

Nous avons en outre fait des mesures sur la navigation des participants lors des essais : le nombre de défilements et de zooms, ainsi que le temps passé et la distance parcourue durant ces opérations. Même si la main virtuelle était désactivée pour certaines conditions, nous suivions en continu la position de la main dominante, lors des essais ; la distance que nous avons enregistrée est celle de la projection de l'index de la main sur la grille. C'est une mesure simple mais fonctionnelle et permettant de comparer chaque \variable{IHM}. Nous avons également compté le temps passé avec un disque sélectionné, pour quantifier le temps de recherche. Enfin, nous avons enregistré la distance entre la tête et le téléphone pour mesurer la navigation physique.

Notre post-questionnaire nous a permis de relever des mesures subjectives sur chaque \variable{IHM}. Nous nous sommes principalement inspiré du NASA TLX \cite{Rubio2004}, avec des questions portant sur l'exigence mentale, l'exigence physique, la performance et la frustration, auxquelles nous avons ajouté des questions sur la facilité de compréhension et la rapidité. Pour chaque question, le participant notait les trois IHMs sur une échelle de Likert (de 1 à 5). Nous avons par contre supprimé la question sur la charge temporelle, que nous trouvions redondante avec les autres. Le participant devait ensuite classer par ordre de préférence les trois IHMs. Enfin, nous avons présenté les scénarios d'applications, dont la carte de navigation en vue étendue, une application multi-fenêtres et des notifications en 3D, et recueilli leurs commentaires.


\section{Participants}
\label{sec:experiment_participants}

Nous avons recruté 16 personnes volontaires pour participer à l'expérience parmi notre entourage. Cependant, nous travaillons avec les données de 12 d'entre eux : il y a eu des erreurs de mesures pour deux d'entre elle, et deux personnes ont arrêtés l'expérience après l'entraînement, le visiocasque leur donnant des nausées.

Ces 12 participants, dont trois femmes, étaient âgés entre 18 et 49 ans (deux au-dessus de 25 ans). Tous avaient une vision normale ou portaient un dispositif de correction de vision ; une personne a dit ne pas savoir distinguer les couleurs pastels mais a été capable, lors de l'entraînement, de reconnaître les éléments à classer sur la grille. Dix étaient droitiers et deux gauchers. Huit d'entre eux avaient déjà utilisé un visiocasque de RV, dont un avait déjà utilisé plusieurs visiocasques de RA. Tous utilisent tous les jours un ordinateur dont huit utilisent régulièrement des logiciels avec un environnement 3D.


\section{Résultats}
\label{sec:experiment_results}

Nos mesures des essais et du post-questionnaire ainsi que nos analyses sont disponibles en ligne (\url{https://github.com/NormandErwan/HandheldVesadAnalysis}), dans le domaine public (\url{http://unlicense.org}). Nous gardons privées les informations du pré-questionnaire afin que les participants ne puissent être identifiés.

Toutes les barres d'erreurs dans nos figures montrent l'intervalle de confiance à 95\%. De même, quand nous reportons une valeur $X$, nous précisons son IC à 95\% ainsi : $X$ [$\text{IC}_{bas}$ ; $\text{IC}_{haut}$]. Nous calculons tous les ICs en utilisant la technique de \texten{bootstraping} \citep[p. 25]{Dragicevic2016}, avec $B=1000$ simulations.

\subsection{Temps de complétion}
\label{subsec:experiment_results_time}

\figureLayoutETS{tct_distributions_all}{%
  \parbox{0.9\textwidth}{%
    \centering%
    \subfigureETS[0.2]{tct_distributions}{Données mesurées.}%
    \\%
    \subfigureETS[0.2]{tct_distributions_log}{Données transformées avec le logarithme népérien.}%
  }
}{
  Histogrammes du temps de complétion d'un essai pour chaque \variable{IHM}.
}

On transforme tout d'abord les mesures de temps de complétion d'un essai avec un logarithmique népérien pour rendre leurs distributions approximativement normales et réduire l'influence des mesures extrêmes \citep[p. 25]{Dragicevic2016}. La \figref{tct_distributions_all} montre les distributions des temps mesurés et transformés. On vérifie ensuite la normalité des distributions avec le test de Shapiro-Wilk\footnote{L'hypothèse nulle testée est que la distribution suit la loi normale ; on attend donc une valeur-p supérieure à 0,05 pour ne pas la rejeter.} \citep{Wobbrock2016} pour chaque \variable{IHM}, car c'est la variable indépendante nous importe le plus : elles sont normales pour les trois IHMs \condition{Téléphone} ($W = \num{0.98}$ ; $p = \num{0.76}$), \condition{VESAD tactile} ($W = 0,97$ ; $p = \num{0.45}$) et \condition{VESAD} ($W = \num{0.97}$ ; $p = \num{0.45}$). En outre, le test de Levene\footnote{L'hypothèse nulle testée est que les variances des distributions sont égales.} ($W = \num{0.76}$ ; $p = \num{0.47}$) nous permet de vérifier que leurs variances sont égales \citep{Wobbrock2016}.

On analyse alors l'effet de toutes les variables indépendantes sur le temps de complétion avec une analyse de variance (ANOVA). Nous pouvons utiliser ce test, car les distributions testées sont normales, indépendantes et ont la même variance \citep{Wobbrock2016}. On utilise le modèle suivant : \variable{TEMPS} $\sim$ \variable{IHM} $\times$ \variable{TAILLE} $\times$ \variable{DISTANCE} $+$ \variable{IHM} $\times$ \variable{GROUPE}. Le \autoref{tab:tct_anova} donne les résultats : les variables \variable{IHM} ($F_{2,22} = \num{62.2}$ ; $p = \num{2e-19}$) et \variable{GROUPE} ($F_{2,22} = \num{2.1}$ ; $p = \num{5e-5}$), ainsi que leur interaction \variable{IHM} $\times$ \variable{GROUPE} ($F_{4,44} = \num{3.1}$ ; $p = \num{1e-5}$) ont un effet significatif sur le temps de complétion.

\begin{tableETS}{tab:tct_anova}{Résultats de l'ANOVA sur le temps de complétion pour toutes les variables indépendantes.}
  \begin{tabular}{| l | C{1.5cm} | C{1.5cm} | C{1.8cm} | c | c |}
    \hline \textbf{Effet} & \textbf{Somme des carrés} & \textbf{Degrés de liberté}\footnotemark & \textbf{Degrés de liberté de l'erreur}\footnotemark & \textbf{F} & \textbf{p} \\
    \hline \variable{IHM} & 12,3 & 2 & 22 & 62,2 & \num{2e-19} \\
    \hline \variable{TAILLE} & 0,1 & 1 & 11 & 1,5 & \num{2e-1} \\
    \hline \variable{DISTANCE} & 0,03 & 1 & 11 & 0,3 & \num{6e-1} \\
    \hline \variable{GROUPE} & 2,1 & 2 & 22 & 10,8 & \num{5e-5} \\
    \hline \variable{IHM}$\times$\variable{TAILLE} & 0,4 & 2 & 22 & 2,0 & \num{1e-1} \\
    \hline \variable{IHM}$\times$\variable{DISTANCE} & 0,07 & 2 & 22 & 0,4 & \num{7e-1} \\
    \hline \variable{TAILLE}$\times$\variable{DISTANCE} & 0,3 & 1 & 11 & 2,9 & \num{9e-2} \\
    \hline \variable{IHM}$\times$\variable{GROUPE} & 3,1 & 4 & 44 & 7,8 & \num{1e-5} \\
    \hline \variable{IHM}$\times$\variable{TAILLE}$\times$\variable{DISTANCE} & 0,5 & 2 & 22 & 2,5 & \num{8e-2} \\
    \hline Résidus & 12,5 & 126 & & & \\
    \hline
  \end{tabular}
\end{tableETS}

\figureETS[0.575]{tct}{
  Temps de complétion moyens par essai pour chaque \variable{IHM}.
}

On les compare alors deux-à-deux avec le test t de Student, avec une correction de Benjamini-Hochberg pour limiter le nombre de faux positifs \autorefp{subsec:litterature_ar_hci_evaluation}. Ce test suppose aussi la normalité, l'indépendance et une variance égale des distributions. Les résultats, au \autoref{tab:tct_ttest}, indiquent des différences significatives entre les trois IHMs. On calcule les moyennes avec intervalles de confiances à 95\% sur les données transformées et y applique la fonction exponentielle. La \figref{tct} donne une donc une bonne preuve que les différences sont également importantes : le \condition{VESAD tactile} est plus rapide de \SI{22}{\s} (+33\%) que le \condition{Téléphone}, lui-même plus rapide de \SI{49}{\s} (+36\%) que le \condition{VESAD}.

\footnotetext[4]{On calcule les degrés de liberté (ddl) d'une variable indépendante ainsi : $ddl = conditions - 1$.}
\footnotetext{On calcule les ddl de l'erreur pour une variable indépendante ainsi : $ddl_{erreur} = ddl \times (observations - 1)$.}

\begin{tableETS}{tab:tct_ttest}{Résultats des tests t sur toutes les paires d'\variable{IHM}.}
  \begin{tabular}{| c | c | c | c |}
    \hline \textbf{Technique 1} & \textbf{Technique 2} & \textbf{T} & \textbf{p} \\
    \hline \condition{Téléphone} & \condition{VESAD tactile} & \num{4.1} & \num{9e-5} \\
    \hline \condition{VESAD} & \condition{Téléphone} & \num{5.6} & \num{3e-5} \\
    \hline \condition{VESAD} & \condition{VESAD tactile} & \num{9.0} & \num{6e-14} \\
    \hline
  \end{tabular}
\end{tableETS}

\figureETS[0.7]{tct_ordering}{
  Temps de complétion moyens (moyennes géometriques) par essai pour chaque condition \variable{IHM} $\times$ \variable{GROUPE}.
}

Cependant, la \figref{tct_ordering} indique que l'effet du \variable{GROUPE} a été important seulement pour certaines conditions : les participants qui ont commencé avec le \condition{Téléphone} (\condition{Groupe 1}) ont été plus lents avec cette IHM, tandis que ceux qui ont terminé avec \condition{VESAD tactile} (\condition{Groupe 3}) ont été plus rapide sur cette IHM. Dans les autres conditions, il paraît ne pas avoir de différence entre les temps de complétion. Cela semble indiquer que, d'une part, il y a effectivement une courbe d'apprentissage avec cette tâche, en particulier quand les utilisateurs doivent reconstruire mentalement la grille, le \condition{Téléphone} en affichant une vue non entière. D'autre part, le \condition{VESAD tactile} semble être le plus rapide des trois IHMs quand la tâche est maîtrisée.

\subsection{Erreurs et sélections}
\label{subsec:experiment_results_errors}

\figureETS[0.75]{selections_errors_distributions}{
  Histogrammes et nuages de points des erreurs et du nombre de sélections par \variable{IHM}.
}

On visualise tout d'abord les distributions des erreurs et du nombre de sélections. La \figref{selections_errors_distributions} donne une bonne indication qu'il y a corrélation entre erreurs et sélections pour chaque \variable{IHM}. Il semble également que les utilisateurs \condition{Téléphone} font plus de sélections, à erreurs égales, que sur les autres IHMs.

Les distributions des deux variables ne suivant pas une loi normale, nous utilisons des tests non-paramétrique \citep{Wobbrock2016}. On utilise alors le test de Kruskal-Wallis (avec une correction de Benjamini-Hochberg) pour vérifier les effets des variables indépendantes. Les résultats indiquent que seuls la \variable{IHM} ($H = \num{15.1}$ ; $p = \num{0.004}$) et le \variable{GROUPE} ($H = \num{11.5}$ ; $p = \num{0.01}$) ont un effet significatif sur le nombre de sélections, mais que seul le \variable{GROUPE} ($H = \num{11.1}$ ; $p = \num{0.01}$) a un effet significatif sur les erreurs. La \figref{errors} nous confirme que nous n'avons pas mesuré de différences importantes en termes d'erreurs pour la \variable{IHM}.

\figureLayoutETS{selections_errors}{%
  \subfigureETS[0.25]{selections}{Nombre moyen d'éléments sélectionnés.}%
  \figurehspace%
  \subfigureETS[0.25]{errors}{Nombre moyen d'erreurs (éléments qui ont été mal classés par le participant).}%
}{
  Sélections et erreurs moyennes par essai pour chaque \variable{IHM}.
}

Pour mieux comprendre l'effet sur le nombre de sélection, on compare deux-à-deux les conditions \variable{IHM} avec le test de Wilcoxon-Mann-Whitney (avec une correction de Benjamini-Hochberg). Les résultats indiquent qu'il y a des différences significatives entre pour les paires \{\condition{Téléphone}, \condition{VESAD tactile}\} ($p = \num{6e-4}$) et \{\condition{Téléphone}, \condition{VESAD}\} ($p = \num{0.03}$), mais pas pour la paire \{\condition{VESAD tactile}, \condition{VESAD}\} ($p = \num{0.06}$). La \figref{selections} nous confirme que le \condition{Téléphone} demande le plus de sélections, soit \num{8.18} [\num{7.42} ; \num{9.06}] en moyenne, et que le \condition{VESAD tactile} en demande le moins, soit \num{6.31} [\num{5.97} ; \num{6.66}] en moyenne, mais cette différence est peu importante.

\figureLayoutETS{selections_errors_ordering}{%
  \parbox{.71\textwidth}{%
    \centering%
    \subfigureETS[0.25]{selections_ordering}{Nombre moyen d'éléments sélectionnés.}%
    \\%
    \subfigureETS[0.25]{errors_ordering}{Nombre moyen d'erreurs.}%
  }%
}{
  Sélections et erreurs moyennes par essai pour chaque condition \variable{IHM} $\times$ \variable{GROUPE}.
}

Comme pour le temps de complétion, il y une bonne indication que le \variable{GROUPE} n'a un effet important que dans certaines conditions \figref{selections_errors_ordering} : le \condition{Groupe 1} a commis un plus d'erreurs et effectué plus de sélections en commençant avec le \condition{Téléphone}, tandis que le \condition{Groupe 3} a fait un peu moins d'erreurs et sélections en terminant avec le \condition{VESAD tactile}. Les autres conditions \variable{IHM} $\times$ \variable{GROUPE} ne paraissent pas importantes. Les participants dans la condition \{\condition{Groupe 1}, \condition{Téléphone}\}, n'ayant pas encore une bonne image mentale de la grille, pouvaient probablement parfois oublier quel disque était sélectionné ou changer d'avis sur le disque à classer. Enfin, le nombre un peu plus élevé d'erreurs pour \condition{VESAD tactile} est probablement dû à des classements involontaires que nous avons parfois observés, le suivi de la main par le Leap Motion n'étant pas toujours bon.

\subsection{Navigation et classements}
\label{subsec:experiment_results_operations}

Les comportements des utilisateurs durant les essais nous permettent de mieux comprendre ces différences entre les IHMs.

\figureETS[0.65]{navigation_count}{
  Nombre total moyen par essai de défilements et de zooms pour chaque \variable{IHM}.
}

\figureETS[0.85]{navigation_distance}{
  La distance parcourue par l'index (en haut à gauche) pendant qu'un disque était sélectionné, (en haut à droite) pendant les défilements, (en bas à gauche) pendant les zooms, ainsi que (en bas à droite) la distance des mouvements de la tête par rapport au téléphone. Toutes les mesures sont des moyennes par essai pour chaque \variable{IHM}.
}

Tout d'abord, nous comparons navigations virtuelles (défilements et zooms) et physiques. La \figref{navigation_count} nous indique que la navigation virtuelle a été principalement utilisée avec le \condition{Téléphone} et le\condition{VESAD tactile} avec en moyenne plus de 25 défilements et cinq zooms par essai. Autrement dit, chaque disque a demandé en moyenne plus d'un défilement et au moins un zoom pour être classé. En revanche, les participants ont peu utilisé les techniques de défilement et de zoom virtuels avec le \condition{VESAD}, mais privilégié une navigation physique \figrefp{navigation_distance} ; nous avons observé des déplacements bi-manuels, ou des rotations de la grille, pour placer au même endroit la main virtuelle et un disque à sélectionner. Des participants ont, par ailleurs, fait la remarque que le suivi du téléphone était plus stable que celui de la main. La navigation physique reste importante pour les deux premières IHMs \figrefp{navigation_distance} : nous avons fréquemment observé des mouvements de \textquote{rapides zooms physiques} dans toutes les conditions, où le participant rapprochait le téléphone de son visage, pour pouvoir lire une lettre d'un disque. Le \condition{VESAD tactile} demande moins de ces mouvements, car le grand écran donnait la possibilité aux participants de laisser travailler avec une grille avec un plus fort zoom.

Ensuite, on peut mieux expliquer la meilleure performance du \condition{VESAD tactile} par rapport \condition{Téléphone}. Nous n'avons mesuré aucune différence importante en termes de nombre de défilements, ainsi que le temps passé \figrefp{navigation_time} et la distance parcourue pendant les défilements. Cependant, les participants ont utilisé seulement cinq zooms en moyenne par essai avec le \condition{VESAD tactile} contre deux fois plus pour le \condition{Téléphone}. De même, les durées et distances moyennes des zooms par essai \figrefp{navigation_time} sont toutes deux moitié moins importantes avec le \condition{VESAD tactile} qu'avec le \condition{Téléphone}. Plus généralement, les participants passaient 30\% moins de temps avec un disque sélectionné avec le \condition{VESAD tactile} par rapport au \condition{Téléphone} \figrefp{navigation_time}.

\figureETS[0.65]{navigation_time}{
  Le temps total moyen par essai passé (gauche) avec un disque sélectionné, (milieu) à défiler et (droite) à zoomer. Ces temps peuvent se supporposer.
}

On peut apporter enfin quelques explications supplémentaires des mauvaises performances du \condition{VESAD}. Le temps passé avec un disque sélectionné est un peu plus important que celui du \condition{Téléphone} : malgré le grand écran, nous avons observé des difficultés des participants pour pointer une cellule pour déplacer le disque. En effet, l'opération de zoom virtuel était jugée trop difficile (il y a très peu de zoom pour cette IHM, mais des durées moyennes longues) par les participants qui préféraient laisser la grille dans sa configuration initiale. Les disques et les cellules étaient alors petits et difficiles à pointer avec la main virtuelle. De plus, cette technique d'interaction demande beaucoup plus de mouvements que travailler avec l'écran tactile \figrefp{navigation_distance}. Par ailleurs, quand les disques à sélectionner se trouvaient du côté de la main tenant le téléphone (à gauche de la main gauche pour un droitier donc), les participants devaient croiser leurs mains, ce que la plupart d'entre eux ont trouvé cela particulièrement ardu et inconfortable.

\figureETS[0.8]{ranks_distributions}{
  Distributions des notes données par les participants à chaque \variable{IHM}. Une note élevée est meilleure ($5$ correspond à une faible frustration).
}

\subsection{Évaluations des participants}
\label{subsec:experiment_results_evaluations}

\figureLayoutETS{preferences_all}{%
  \subfigureETS[0.22]{preferences_distribution}{Distribution des préférences.}%
  \figurehspace%
  \subfigureETS[0.22]{preferences}{Préférences moyennes (moyennes géometriques), la meilleure note est $1$ et la moins bonne $3$.}%
}{
  Préférences des participants pour chaque \variable{IHM}.
}

\figureETS[0.8]{ranks}{
  Notes moyennes (moyennes géometriques) des participants à chaque \variable{IHM} (une note élevée est meilleure).
}

Comme pour les erreurs, nous utilisons des tests non-paramétriques pour analyser les mesures subjectives du post-questionnaire, leurs distributions étant non-normales et le nombre \cite{Wobbrock2016}. La \figref{ranks_distributions} montre les distributions des notes et la \figref{preferences_distribution} celle des préférences. Comme pour le temps de complétion, on transforme d'abord les données avec le logarithme népérien, pour y calculer moyennes avec IC à 95\% que l'on reconvertit avec l'exponentielle.

On utilise alors le test de Kruskal-Wallis pour déterminer s'il y a des différences significatives aux réponses de chaque questions, dû à la \variable{IHM}. Puis, on utilise un test de Wilcoxon-Mann-Whitney pour comparer les questions avec une différence significative. On applique une correction de Benjamini–Hochberg aux valeurs-p retournées par les deux tests. Les IC à 95\% des moyennes des réponses aux notes \figrefp{ranks} et au classement \figrefp{preferences} nous indiquent l'importance de l'effet. Les résultats obtenus sont les suivants :
\begin{itemize}
  \item Facilité de compréhension ($p = \num{0.007}$) : seul le \condition{Téléphone} est significativement meilleur que le \condition{VESAD} ($p = \num{0.01}$), mais il semble que le \condition{VESAD tactile} soit un peu moins bon aussi. Cela n'est guère étonnant, car le \condition{Téléphone} est connu et maîtrisé des utilisateurs. Les différences ne sont tout de même pas trop importantes
  \item Mentalement exigeant ($p = \num{0.007}$) : le \condition{Téléphone} est significativement pire (-26\%) que le \condition{VESAD tactile} ($p = \num{0.005}$) et le \condition{VESAD} ($p = \num{0.01}$). Cela donne une bonne preuve que le petit écran rend la tâche plus difficile.
  \item Physiquement exigeant : aucune différence significative ($p = \num{0.1}$). Les trois IHMs ont des valeurs assez basses ($\approx$3), ce qui semble souligner la difficulté de la tâche. Cependant, la distribution des notes indique que le \condition{VESAD} a été le plus difficile \figrefp{ranks_distributions}, probablement car pointer avec la main en 3D est plus dur et lent que sur une surface 2D \citep{Argelaguet2013}. Quant au \condition{Téléphone}, plusieurs participants se sont plaint de faire beaucoup de défilements et de zooms ; certains ont trouvé pénible de devoir amener une cible sur l'écran tactile plutôt que pouvoir la sélectionner directement avec le \condition{VESAD tactile}.
  \item Rapidité ($p = \num{0.04}$) : le \condition{VESAD tactile} a été jugé significativement plus rapide que le \condition{VESAD} ($p = \num{0.005}$) avec un effet qui semble important (+40\%). Nous cependant n'avons pas pu mesurer de différence entre le \condition{Téléphone} et les deux autres IHMs, les notes étant très variées \figrefp{ranks_distributions}.
  \item Performance ($p = \num{0.006}$) : le \condition{VESAD tactile} est significativement, et de manière importante, plus rapide que le \condition{Téléphone} ($p = \num{0.005}$, +25\%) et \condition{VESAD} ($p = \num{0.01}$, +47\%). Nous n'avons cependant mesuré aucune différence entre ces deux dernières IHMs.
  \item Frustration : aucune différence significative ($p = \num{0.1}$), même si les distributions semblent pointer que le \condition{VESAD} l'était un peu plus, probablement dû au mauvais suivi de la main.
  \item Préférences ($p = \num{0.006}$) : le \condition{VESAD tactile} est significativement, et largement, préféré au \condition{VESAD} ($p = \num{0.01}$) et au \condition{Téléphone} ($p = \num{0.004}$). Il n'y a pas de différence significative entre ces deux dernières IHMs, même si le \condition{VESAD} semble un peu plus préféré ; avec un suivi de la main fiable, plusieurs participants nous ont dit qu'ils le voyaient le plus prometteur.
\end{itemize}

Pour finir, tous les participants ont trouvé intéressant et utile l'idée du multi-tâche (multiples applications, ou application multi-fenêtres) sur un téléphone à écran étendu \figrefp{HandheldVESADApps} et la majorité d'entre eux pour la carte de navigation \figrefp{HandheldVESADMap}.