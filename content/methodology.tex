\chapter{Méthodologie}
\label{ch:methodology}

\section{Procédure}
Comme décrit dans le \refsection{ch:litterature}, il n'existe aucun visiocasque de RA sur le marché avec un grand champs de vision. Le premier sous-problème de ce travail de recherche a donc été de développer un visiocasque de RA à large champs de vision.\\
La procédure suivie a été la suivante :
\begin{enumerate}
  \item Sélection d'une caméra stéréo fisheye, d'un visiocasque de RV et d'un moteur 3D
  \item Réalisation de la librairie de réalité augmentée ArucoUnity
\end{enumerate}
Principe RA :
Choisir un modèle approprié qui décrit correctement la caméra
Calibrer la caméra pour savoir les paramètres de ce modèle
L'utiliser pour faire correspondance 2d image d'un marker avec sa position-rotation 3d
Placer la caméra virtuelle comme la caméra réelle + matcher le modèle de projection de la caméra virtuelle et de la caméra réelle : donne effet d'alignement du virtuel avec le réel

\section{Choix techniques}
\label{sec:technical_choices}
\subsection{Choix matériels}

\subsection{Choix logiciels}

Unity3d utilisé pour gestion du oculus et 3d top. Plus simple à prendre en main que Unreal, mais plus rigide (framework boîte noire, difficile de sortir sentiers battus)

Citer Mobile Marker-based Augmented Reality as
an Intuitive Instruction Manual HANNAH REUTERDAHL
Pour la liste des markers et frameworks : dire ce qui est compatible Unity ou c++ mais aucun ne permet de travailler avec caméra fisheye : d'où aruco unity


\section{Réalisation de la librairie de réalité augmentée ArucoUnity}
Camera Models and Fundamental Concepts Used in Geometric Computer Vision
Camera model : équation pour expliquer comment la caméra projette le monde 3d en une image
Pinhole p.41
Calib p.103