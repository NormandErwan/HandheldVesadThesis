Nous avons exploré, dans ce travail de recherche, l'extension de l'écran d'un téléphone intelligent par réalité augmentée, créant la perception d'un unique grand écran étendu tenu en main. Nous avons appelé cette extension : VESAD, pour \texten{Virtuality Extended Screen-Aligned Display}. Plus précisément, nous souhaitions évaluer les avantages à utiliser un téléphone à écran étendu par rapport à un téléphone seul. Nous voulions également comparer différentes techniques d'interactions : soit en utilisant l'écran tactile, soit en manipulant la partie virtuelle de l'écran avec la main (technique de main virtuelle).

Pour répondre à ces questions, nous avons exploré la conception de cette interface humain-machine (IHM). Puis, nous avons développé notre visiocasque de RA avec un large champ de vision pour visualiser l'écran étendu au complet. Enfin, nous avons mené une étude expérimentale pour répondre à ces questions en comparant ces différentes techniques d'interactions sur le VESAD face à un téléphone seul. Nous posions les hypothèses que les participants auraient été plus performants avec l'écran étendu via les interactions tactiles, mais qu'ils auraient préféré utiliser la technique de main virtuelle (probablement perçue comme plus naturelle).

Nous avons tout d'abord motivé cette problématique par une revue de la littérature \autorefp{ch:litterature} où nous avons noté qu'il y avait un besoin de recherches, prioritairement des études expérimentales, sur les IHMs en RA \citep{Billinghurst2005, Billinghurst2015}. En particulier, il est encore peu clair quelles techniques d'interactions sont les plus adaptées à la RA \citep{Argelaguet2013, Piumsomboon2013, Piumsomboon2014}, ni la forme des interfaces en RA \citep{VanDam1997, Ens2014a, Serrano2015}. Plusieurs études ont cependant pointé l'intérêt à combiner des visiocasques de RA avec des appareils que nous utilisons professionnellement \citep{Grubert2015, Serrano2015a}. Nous nous sommes alors interrogé si les résultats des études sur l'agrandissement des écrans \cite{Baudisch2002, Guiard2004} et des affichages muraux \cite{Liu2014, Raedle2014} s'appliqueraient également à un téléphone à écran étendu.

Notre première contribution est l'exploration des IHMs et interactions possibles avec ce concept d'extension de l'écran \autorefp{ch:concept}, qui avait déjà été proposé par \cite{Grubert2015} pour étendre l'écran d'une montre intelligente. Un VESAD peut être utilisé selon deux modes de vues : en \emph{vue multi-fenêtres}, pour supporter du multi-tâche, et en \emph{vue étendue}, où une application utilise tout l'écran étendu (par exemple une carte de navigation). Nous présentons aussi deux nouvelles techniques d'interaction : \texten{wrist}, où l'utilisateur fait une rotation rapide du poignet pour substituer l'affichage sur le VESAD avec un autre (similaire à un changement de bureau sur un PC), et \texten{slide-to-hang} pour détacher l'écran étendu en une fenêtre virtuelle séparée du téléphone. Enfin, nous proposons une modification au cadre théorique Ethereal Planes \citep{Ens2014a}, permettant de concevoir des IHMs utilisant des fenêtres 2D en RA.

Nous détaillons ensuite au \autoref{ch:methodology} la conception de notre propre visiocasque de RA à large champ de vision (\SI{100x98}{\degree} pour chaque \oe il). Nous reprenons le principe du visiocasque de \cite{Steptoe2013}, repris par \cite{Steptoe2014} et \cite{Piumsomboon2014}, qui consiste à utiliser une caméra stéréoscopique \texten{fisheye} avec un visiocasque de RV. Notre deuxième contribution est la documentation précise de ce développement, permettant à d'autres chercheurs de reproduire ce visiocasque. Un VESAD fonctionne à condition qu'il y ait (1) un excellent suivi en 3D de l'écran physique à étendre, (2) un bon alignement de l'écran virtuel avec l'écran physique et (3) une synchronisation sans latence entre ces deux écrans. Notre troisième contribution est la réalisation de la bibliothèque libre de réalité augmentée ArucoUnity (\url{https://github.com/NormandErwan/ArucoUnity}), qui porte sur le moteur 3D Unity les fonctions de suivi de marqueurs et de calibration de caméra de la bibliothèque libre de vision par ordinateur OpenCV. Elle résout les points (1) et (2).

Notre quatrième contribution sont les résultats de notre étude expérimentale \autorefp{ch:experiment}, menée avec 12 participants. Nous avons reproduit une tâche de classement (mode de vue étendu), impliquant de la navigation et des sélections, issue de la littérature sur les affichages muraux, et implémenté des techniques d'interactions décrites par \cite{Wobbrock2009} pour l'écran tactile et \cite{Piumsomboon2013} pour la main virtuelle. Nos résultats indiquent que la combinaison du téléphone à écran étendu avec les interactions tactiles s'est montrée la plus rapide, en particulier quand la tâche était maîtrisée des participants, et la plus performante en termes de navigation. Les participants l'ont également préférée. Néanmoins, plusieurs d'entre eux ont vu plus de potentiel dans les interactions avec la main virtuelle avec l'écran étendu. Enfin, lors de la conception d'une IHM pour un VESAD, nous recommandons de limiter l'espace d'interaction aux zones faciles à accéder. Nous donnons également des pistes d'amélioration pour l'utilisation d'un écran étendu avec une main virtuelle.

Notre recherche présente bien évidemment des limites. Tout d'abord, le mauvais suivi de la main des participants dans l'étude expérimentale a joué en la défaveur du \condition{VESAD}, qui a été la moins performante des trois IHMs évaluées. En outre, nous avons mal conçu les gestes pour manipuler l'écran étendu du \condition{VESAD}, rendant les défilements et zooms avec la main virtuelle particulièrement difficiles. Ensuite, nous n'avons pas correctement contrôlé la difficulté de notre tâche, car nous n'avons mesuré aucune différence entre les conditions \textquote{faciles} et \textquote{difficiles} de notre tâche. Enfin, la principale limite de notre étude est la faible densité visuelle (mesuré en pixels par degré) ainsi que la latence de notre visiocasque, ce qui a désavantagé le \condition{Téléphone} seul. Il est possible qu'améliorer le visiocasque utilisé change nos résultats, mais il nous semble très probable que certaines tâches bénéficieraient d'un écran étendu.

Nous esquissons pour finir quelques directions futures à ce travail de recherche. L'amélioration du suivi de la main et de techniques d'interactions avec le \condition{VESAD} nous semble être le plus essentiel. Il serait intéressant alors de reproduire notre étude expérimentale et d'y comparer d'autres techniques d'interactions comme un pointeur virtuel (répandue parmi les visiocasques de l'industrie, comme le Microsoft HoloLens) ou des commandes vocales. Enfin, notre concept de téléphone à écran étendu devrait être évalué sur d'autre d'autres types de tâche comme, par exemple, un scénario de multi-tâche, plus proche des usages quotidiens avec un téléphone intelligent.