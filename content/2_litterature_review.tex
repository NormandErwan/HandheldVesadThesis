\chapter{Revue de littérature}
\label{ch:litterature}

\section{Historique et concepts de la RA}
Shuterland1968 : premier casque\\

Après Sutherland : recherche militaire et gouvernement surtout (bilinghurst, 2015), le domaine acad s'est pas trop développé

MilgramKishino1994\\
Mixed Reality Continuum : continuum combiner réel et virtuel, où réel aucun élément virtuel vs réel. Ar est entre deux, où du virtuel augmente la vue du réel de l'utilisateur\\
RA et RV très proches techniquement mais pas même attentes (bilinghurst, 2015) : VR c'est grand FOV et réalisme 3D, RA c'est tracking

RekimotoNagao1995\\
montre que les IHM RA doivent être guidée dans le but d'êtres invisibles et d'augmenter les interactions avec le réel. Avec les interfaces graphiques sur un écran, le réel et l'ordinateur sont séparés (ça fonctionne pour les téléphones). Décrit aussi que la RV ou ubiquitous computing sont aussi d'autres manières de rendre le pc invisible

Azuma1997 : définition formelle, propriétés RA (contenu 3D virtuel, aligné avec le contenu réel, en temps réel), premier état de l'art

BuxtonFitzMaurice1998, BimberRaskar2005, VanKrevelenPoelman2010 :\\
- plateformes en RA (CAVE, mobile, HMD, lentilles) : CAVE fonctionne très bien, mais couteux et encombrant ; HMD fonctionnent maintenant aussi bien que CAVE, plus légers et permettent expérience partagée ; mobile est populaire mais limités en taille et en puissance ; lentilles semblent être avenir idéal (VanKrevelenPoelman2010)\\
- catégories HMD : video see-through vs. optical see-through

AzumaBaillotBehringerEtAl2001 : création de ISMAR et du domaine de recherche à part enière

Limitations techniques encore : tracking et affichage. Même si se réduit et bientôt va être suffisament bon pour être présent dans nos quotidiens pros et perso.\\
AzumaBaillotBehringerEtAl2001 : les RA mobiles sont possibles\\
VanKrevelenPoelman2010, CarmignianiFurhtAnisettiEtAl2011, HuangHuiPeyloEtAl2013, Billinghurst2015 : beaucoup de RA mobiles dans les produits commerciaux, peu avec des HMDs (HoloLens change un peu la donne et en même temps montre qu'il y a du besoin si Microsoft si risque c'est qu'il y voit un potentiel) et beaucoup de challenges techniques de tracking et display encore à résoudre\\
Billinghurst2005 : There is a need to develop interface metaphors and interaction techniques specific to AR


\section{Espaces de travail en RA}
Serrano2015
Serrano2015a
Grubert2015
EnsFinneganIrani2014
Ens2014 - Ethereal Planes : cadre de conceptions pour des fenetres 2D dans un espace de travail en RA. Redonner application du Personal Cockpit.


\section{Conception et évaluation d'IHMs 3D}
Bowman2004 : summarizes various types of 3D interactions into three categories: (1) navigation, (2) selection, and (3)
manipulation

Piumsomboon2014 : a exploré manipulation d'objets 3D avec mid-air


\section{Interfaces intangibles}
Billinghurst2015, p.169 : interface tangibles (Tangible User Interface (TUI))
Kato et al. [2000] proposed the concept of Tangible AR (TAR). TAR uses Tangible UI as input interaction metaphor while using AR for visualizing virtual information overlaid on the physical object used for interaction. the interaction space and display space are seamlessly merged together
The basic goal of designing a Tangible AR interface is to map physical objects (input) with virtual objects (output) using an appropriate interaction metaphor. 
Space multiplexed vs space multiplexed

Tosas2004 : idée d'interfaces intangible flottant en l'air (écran ou clavier à toucher)

Principal problème est qu'il n'y a aucun retour tactile
Chan2010 - Touching the void : mid air touch in intangible displays. Naturle car simplifie la manip d'objet : le display et l'interactions sont combinés (de la même manière qu'on manipule des objets réels). Expérience d'acquisition d'objets : les personnes évaluent mal la profondeur de leur doigt (donc quand elles ont touché la cible), car pb double vision : vise le doigt et donc cible est floue. Conclusion : il faut utiliser des retours visuels pour guider l'utilisateur. Deux types de feedbacks : continu pour situer sa main, discret pour confirmer une action.

Piumsomboon2013 : fait une taxonomie des gestes mid-air pour l'AR, sur le modèle de Wobbrock2009


\section{Extension d'un écran d'affichage}
Baudisch2002


\section{Grands affichages}
Liu2014


\section{Problématique}
Cette revue de littérature a permit d'identifier un besoin de conception d'IHMs en RA s'appuyant sur des visiocasques. En particulier, nous souhaitons explorer la conception d'IHM pour un système combinant un téléphone intelligent avec un visiocasque de RA. Ainsi, la problématique de ce mémoire se définie ainsi :

\begin{displayquote}
  Est-il intéressant de proposer à un utilisateur d'étendre l'écran de son téléphone intelligent par un visiocasque de RA ? Quelles seraient les meileures techniques d'interactions à utiliser sur un tel système ?
\end{displayquote}

Les hypothèses de la problématique sont les suivantes :
\begin{enumerate}
  \item Notre système	permettrait d'être plus performant sur des tâches de navigation, de classification ou demandant d'utiliser plusieurs applications que sur un téléphone seul, quelle que soit la technique d'interaction utilisée.
  \item Les utilisateurs apprécieront pouvoir interagir directement avec l'écran étendu autour du téléphone sur notre système.
  \item Les utilisateurs seront en revanche plus performants en interagissant seulement avec l'écran tactile du téléphone sur notre système.
\end{enumerate}

Pour y répondre, nous divisons cette question en quatre sous-problèmes :
\begin{enumerate}
  \item concevoir une IHM d'un téléphone à l'écran aggrandi par RA ;
  \item développer un visiocasque de RA à large champs de vision (supérieur à 90° horizontalement) ;
  \item développer un prototype de cette IHM à l'aide du visiocasque ;
  \item réaliser une expérimentation évaluant différentes interfaces et techniques d'interactions sur ce prototype sur une tâche de classification.
\end{enumerate}

Les résultats à ces objectifs permettront de donner des recommandations pour de futures recherches d'IHM en RA.