\chapter{Revue de littérature}
\label{ch:litterature}

\section{Historique et concepts de la RA}
Shuterland1968 : premier casque

MilgramKishino1994 : Reality-Virtuality Continuum

RekimotoNagao1995 : le but de la RA 

Azuma1997 : définition formelle, propriétés RA (contenu 3D virtuel, aligné avec le contenu réel, en temps réel), premier état de l'art

AzumaBaillotBehringerEtAl2001 : création de ISMAR et du domaine de recherche à part enière

BuxtonFitzMaurice1998, BimberRaskar2005, VanKrevelenPoelman2010 : 
- plateformes en RA (CAVE, mobile, HMD, lentilles) : CAVE fonctionne très bien, mais couteux et encombrant ; HMD fonctionnent maintenant aussi bien que CAVE, plus légers et permettent expérience partagée ; mobile est populaire mais limités en taille et en puissance ; lentilles semblent être avenir idéal (VanKrevelenPoelman2010)
- catégories HMD : video see-through vs. optical see-through

AzumaBaillotBehringerEtAl2001 : les RA mobiles sont possibles

VanKrevelenPoelman2010, CarmignianiFurhtAnisettiEtAl2011, HuangHuiPeyloEtAl2013 : beaucoup de RA mobiles et beaucoup de challenges techniques

Limitations techniques encore : tracking et affichage. Même si se réduit et bientôt va être suffisament bon pour être présent dans nos quotidiens pros et perso.


\section{}
Tosas2004 - 

Chan2010 - Touching the void : mid air touch in intangible displays. Naturle car simplifie la manip d'objet : le display et l'interactions sont combinés (de la même manière qu'on manipule des objets réels). Expérience d'acquisition d'objets : les personnes 