Je tiens tout d'abord à remercier mon directeur de maîtrise, Michael J. McGuffin, professeur à l'École de Technologie Supérieure (ÉTS), pour m'avoir laissé beaucoup de liberté dans le choix du sujet de ce mémoire, pour m'avoir guidé tout au long de ma maîtrise, pour avoir soutenu financièrement et matériellement ce projet, pour son aide lors de l'écriture de notre article scientifique et de ce mémoire, pour m'avoir permis d'avoir deux expérience professionnelles en RA à mi-temps en parallèle de cette maîtrise et enfin pour m'avoir permis de rencontrer de nombreux chercheurs à Montréal et à l'inspirante conférence UIST 2017.

Je souhaite aussi remercier l'Université de Technologie de Compiègne (UTC), en particulier Annick Pourplanche, responsable des séjours à l'étranger, et Philippe Trigano, responsable du département de Génie Informatique, ainsi que les gouvernements français, québécois et canadiens pour m'avoir permis de réaliser ce diplôme et de vivre deux ans dans un pays que je tenais beaucoup à découvrir.

Je remercie aussi le personnel du département de génie logiciel et des TI de l'ÉTS pour avoir répondu à mes nombreuses demandes d'aide technique.

Je remercie ma famille qui m'a toujours encouragé dans ce projet. Je remercie aussi mes amis, qu'ils viennent de Bretagne, de Compiègne, de Montréal ou d'ailleurs pour m'avoir soutenu et conseillé quand c'était nécessaire, je pense en particulier à Peet Vincenti et Audrey Bramy qui ont terminé bien avant moi leurs maîtrises à l'ÉTS. Enfin, je remercie Agathe Cestelli qui m'a soutenu et m'a attendu depuis la France : je suis heureux de pouvoir de retrouver enfin...