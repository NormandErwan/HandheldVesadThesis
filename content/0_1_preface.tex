2015 fut une année pleine de promesses et de changements. Après avoir terminé ma première expérience professionnelle en tant que développeur avec beaucoup de plaisir, à VideoStich (renommée Orah en 2016, \url{https://www.orah.co/}), start-up parisienne créant un logiciel de capture et d'édition de vidéo 360 en 4K, je terminais mon cursus ingénieur à l'Université de Technologie de Compiègne (\url{https://www.utc.fr/}) par quatre mois particulièrement intensifs et inspirants de cours, de projets et de contrats comme développeur indépendant. Immédiatement après, je m'envolai pour Montréal durant l'été pour en session d'échange à l'École de Technologie Supérieure. Ces quelques derniers mois de 2015 me permirent de prendre mes marques dans cette belle nouvelle vie québécoise. C'est dans un de mes cours de cette session que je fis la rencontre de mon futur directeur de recherche et que je pus alors commencer ma maîtrise en réalité augmentée début 2016.

Deux ans et quelques mois plus tard, je suis fier de pouvoir enfin présenter mon mémoire de cette maîtrise qui fut elle aussi pleine de défis, de rencontres et d'expériences.