\section*{Introduction}

La réalité augmentée (RA) consiste à combiner du contenu virtuel (généré par une système informatique) à l'environnement d'une personne, et cela en temps réel~: la perception du réel par les sens humains est donc augmentée. L’œil peut voir plus d'informations, celles du réel et celles ajoutées au réel, ou l'oreille peut entendre plus qu'elle n'est capable d'entendre : toutes ces informations supplémentaires sont le contenu virtuel que la machine créé et ajoute au réel. \\
Le mobile.

La réalité augmentée peut toucher tous les sens humains. Cependant, elle est souvent désignée sous ce terme comme s'adressant au sens visuel, et c'est cette signification qui sera utilisé dans cette revue de littérature. \\
Il existe aujourd'hui plusieurs techniques de RA, classées en trois catégories \reffig{BimberRaskar2005-Figure31}. La catégorie dominant actuellement le marché est celle des visiocasques (ou \foreignlanguage{english}{Head-Mounted Display}) \cite[]{VanKrevelenPoelman2010}~: on le voit par la présence médiatique des visiocasques grand public tel que l'Oculus Rift, le HTC Vive ou Microsoft HoloLens. Seule cette catégorie sera explorée dans cette revue de littérature. Les techniques de cette catégorie consistent à placer un casque devant les yeux de l'utilisateur pour y diffuser la RA. Un premier type de ces techniques, dites \foreignlanguage{english}{video see-through}, vont remplacer l'environnement par une image filmée et augmentée de cet environnement. Cela va se faire en utilisant des caméras à l'avant du casque et en modifiant les images filmées pour les renvoyer l'écran du casque. Un second type de ces techniques, dites \foreignlanguage{english}{optical see-through}, vont laisser voir à l'utilisateur directement l'environnement, et, par un jeu de miroirs et de lentilles, vont y superposer les images de RA.

\figureETS{figures/BimberRaskar2005-Figure31.png}{Catégories des techniques d'affichages de RA \\ Tiré de \cite[p. 72]{BimberRaskar2005}}{BimberRaskar2005-Figure31}

\section*{Historique}
Le domaine de recherche de la RA débute avec \cite{Sutherland1968} qui développa le premier visiocasque~: ce prototype permettait déjà de visualiser du contenu virtuel en 3D, affiché à l'utilisateur du visiocasque selon sa perspective visible depuis la position de sa tête, lui donnant ainsi l'illusion d'un contenu virtuel réellement présent dans l'espace. \\
La recherche en RA a continué lentement pendant les décennies suivantes, où quelques travaux sont notables. (\textbf{TODO citer} Caudell and Mizell, 1975 Myron Krueger, Steven Feiner, Blair MacIntyre and Doree Seligmann). \\
C'est dans les années 90, où la RA devient un domaine de recherche à part entière~; cela se voit tout d'abord par la création de plusieurs conférences dédiées, réunies aujourd'hui sous le nom de International Symposium on
Mixed and Augmented Reality (ISMAR), une conférence majeure pour la recherche et l'industrie \cite[]{VanKrevelenPoelman2010}. \cite{MilgramKishino1994} propose une première clarification de la terminologie du domaine, en mettant en contraste la RA avec le domaine, connexe, de la réalité virtuelle (RV), le long une échelle ordonnée nommée \foreignlanguage{english}{Reality-Virtuality Continuum} \reffig{MilgramKishino1994-Figure1}. Cette échelle oppose les environnements réels aux environnements VR, qui, eux, immergent totalement l'utilisateur dans un monde virtuel, la RA se situant entre les ces deux extremum. Enfin, \cite{Azuma1997} réalise le premier état de l'art du domaine. \\
Les années 2000 et le mobile

\figureETS{figures/MilgramKishino1994-Figure1.png}{L'échelle \foreignlanguage{english}{Reality-Virtuality Continuum} de Milgram\\ Tiré de \cite[p. 3]{MilgramKishino1994}}{MilgramKishino1994-Figure1}

Applications \cite[]{VanKrevelenPoelman2010} \cite{CarmignianiFurhtAnisettiEtAl2011}

Mais elles sont limitées + sont plus demandantes que la VR technologiquement c'est pour ça que pas encore mature sur le marché + ont héritage -> Historique + attentes aujourd'hui
Premier prototype en 1966 avec Sutherland. Puis les années 90 ont vu l'arrivée des PC avec la miniaturisation, qui s'est poursuivie par la suite avec les PDA et les smartphones dans les années 2000. L'AR est peu mobile, les prototypes sont souvent dans les labos. L'AR est devenue un champs de recherche à part entière à la fin des 90s

Ammène au mobile qui a explosé cette décennie

Or l'AR va exploser cette décennie
Donc AR et le mobile ? + il y a la puissance, les capteurs, et l'AR a besoin de mobilité (1 des critères pour être pouvoir faire partis de la vie quotidienne) \cite[]{VanKrevelenPoelman2010}


Visualisation
Interfaces
partir des critères du personal cockpit
+ taches

Grands écrans vs petits

Influences du hardware
FoV
Résolution


Interactions

bouger yeux vs bouger tête (partir de Sutherland qui ne tracke pas les yeux car pas si nécessaire)
main
touch sur mobile


\section*{Conclusion}
