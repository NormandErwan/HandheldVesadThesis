\chapter{Étude expérimentale}
\label{ch:experiment}

\section{Description de la tâche expérimentale}
\label{sec:experiment_description}

Voir la liste de tâches : \url{https://docs.google.com/document/d/1X5-XW-9GTCz254PXlskjNka-kS-X7MhGXB6Ixv8vAjk/edit#}

Leap in:
User-Defined Gestures for Augmented Reality \url{https://hal.inria.fr/hal-01501749/document}
Pourquoi on teste l'interaction directe et pas mid-air interaction : c'est lent Vulture: a mid-air word-gesture keyboard \url{https://dl.acm.org/citation.cfm?id=2556964}
Grasp-Shell vs Gesture-Speech: A comparison of direct and indirect natural interaction
techniques in Augmented Reality \url{https://ir.canterbury.ac.nz/bitstream/handle/10092/11090/12652683_paper138-cr.pdf?sequence=1}

Touch in:
User-Defined Gestures for Surface Computing \url{https://www.microsoft.com/en-us/research/wp-content/uploads/2009/04/SurfaceGestures_CHI2009.pdf}
Référence industrie (Android) : \url{https://material.io/guidelines/patterns/gestures.html}

justification for why we won’t be doing experimental comparison with ray cast: because Leap Motion and HoloLens require your hand to be visible, not pulled back, so ray cast selection wouldn’t help the user avoid fatigue

Le détail du développement de l'expérience est décrit dans l'annexe \ref{appendix:experiment} \nameref{appendix:experiment}.


\section{Protocole experimental}
\subsection{Procédure}

\subsection{Matériel}
Un ordinateur de bureau avec un processeur Intel Core i5 7400, 8 Go DDR4 de mémoire vive, et une carte graphique NVIDIA GeForce GTX 1060, sous Windows 10 a été utilisé pour faire tourner le visiocasque de RA. Une carte WiFi externe a été ajouté pour la communication avec le téléphone.\\
Le téléphone utilisé était un Xiaomi Redmi Note 4. C'est un téléphone récent, sous une version récente d'Android (version 7), avec une bonne puissance de calcul et un grand écran HD de 5,5 pouces. Il est léger et tient confortablement dans une main.\\
Le visiocasque de RA, l'outil de suivi des mains ainsi que les logiciels utilisés sont décrits dans la \refsection{sec:technical_choices}.

\subsection{Participants}
Nous avons recruté 12 personnes volontaires pour participer à l'expérience, agés entre 18 et 49 ans : six hommes et six femmes. Tous avaient une vision normale ou portaient un dispositif de correction de vision.

\subsection{Hypothèses}

\subsection{Collecte des données}


\section{Résultats}