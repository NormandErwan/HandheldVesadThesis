\chapter{Étude expérimentale}
\label{ch:experiment}

\section{Description de la tâche expérimentale}
\label{sec:experiment_description}

Voir la liste de tâches : \url{https://docs.google.com/document/d/1X5-XW-9GTCz254PXlskjNka-kS-X7MhGXB6Ixv8vAjk/edit#}

Leap in:
User-Defined Gestures for Augmented Reality \url{https://hal.inria.fr/hal-01501749/document}
Pourquoi on teste l'interaction directe et pas mid-air interaction : c'est lent Vulture: a mid-air word-gesture keyboard \url{https://dl.acm.org/citation.cfm?id=2556964}
Grasp-Shell vs Gesture-Speech: A comparison of direct and indirect natural interaction
techniques in Augmented Reality \url{https://ir.canterbury.ac.nz/bitstream/handle/10092/11090/12652683_paper138-cr.pdf?sequence=1}

Touch in:
User-Defined Gestures for Surface Computing \url{https://www.microsoft.com/en-us/research/wp-content/uploads/2009/04/SurfaceGestures_CHI2009.pdf}
Référence industrie (Android) : \url{https://material.io/guidelines/patterns/gestures.html}

justification for why we won’t be doing experimental comparison with ray cast: because Leap Motion and HoloLens require your hand to be visible, not pulled back, so ray cast selection wouldn’t help the user avoid fatigue

Le détail du développement de l'expérience est décrit dans l'annexe \ref{annex:experiment} \nameref{annex:experiment}.


\section{Protocole experimental}
\subsection{Procédure}
Le formulaire d'information et de consentement (voir \refsection{annex:consent})

\subsection{Matériel}
Un ordinateur de bureau avec un processeur Intel Core i5 7400, 8 Go DDR4 de mémoire vive, et une carte graphique NVIDIA GeForce GTX 1060, sous Windows 10 a été utilisé pour faire tourner le visiocasque de RA. Une carte WiFi externe a été ajouté pour la communication avec le téléphone.\\
Le téléphone utilisé était un Xiaomi Redmi Note 4. C'est un téléphone récent, sous une version récente d'Android (version 7), avec une bonne puissance de calcul et un grand écran HD de 5,5 pouces. Il est léger et tient confortablement dans une main.\\
Le visiocasque de RA, l'outil de suivi des mains ainsi que les logiciels utilisés sont décrits dans la \refsection{sec:technical_choices}.

\subsection{Participants}
Nous avons recruté 12 personnes volontaires pour participer à l'expérience, agés entre 18 et 49 ans : six hommes et six femmes. Tous avaient une vision normale ou portaient un dispositif de correction de vision.

\subsection{Collecte des données}
\subsubsection{Pré-questionnaire}

\subsubsection{Mesures lors de l'expérience}

\subsubsection{Post-questionnaire}

\subsection{Hypothèses}
Mesures :
\begin{itemize}
  \item Temps total : PHONE_IN_AR_OUT > PHONE_ONLY > LEAP_IN_AR_OUT : imprécisions interactions sur LEAP_IN_AR_OUT vs. grand espace et précision touch sur PHONE_IN_AR_OUT
  \item Erreurs, nombre de sélections, nombre de déselections : LEAP_IN_AR_OUT > PHONE_ONLY > PHONE_IN_AR_OUT : imprécisions interactions sur LEAP_IN_AR_OUT vs. grand espace et précision touch sur PHONE_IN_AR_OUT
  \item Pan count, time, distance : PHONE_ONLY > (PHONE_IN_AR_OUT = LEAP_IN_AR_OUT) : si on voit un espace plus grand, moins de drags et plus efficaces
  \item Zoom count, time, distance : PHONE_ONLY > (PHONE_IN_AR_OUT = LEAP_IN_AR_OUT) : si on voit un espace plus grand, moins de zooms et plus efficaces
  \item Head phone distane : (PHONE_IN_AR_OUT = LEAP_IN_AR_OUT) > PHONE_ONLY : pour mieux voir les lettres qui sont loins du téléphone
\end{itemize}

Post-questionnaire :
\begin{itemize}
  \item question 1 sur facilité de compréhension : LEAP_IN_AR_OUT est plus difficile à comprendre et demande un temps d'essai
  \item question 2 sur facilité d'utilisation : LEAP_IN_AR_OUT est plus difficile à utiliser car touche de l'intangible (pb de placement du doigt, quand c'est touché ou non, quand arrêter le touché)
  \item question 3 sur la charge mentale : notre tâche demande de la recherche, de l'observation, de la mémorisation et des décisions ; je pense que PHONE_IN_AR_OUT demande moins de charge mentale
  \item question 4 sur la fatigue : LEAP_IN_AR_OUT est la plus fatiguante car bras en l'air et longtemps car lent, PHONE_IN_AR_OUT la moins fatiguante car la plus rapide
  \item question 5 sur la pression temporelle : je l'ai sentie plus forte sur la technique PHONE_ONLY que sur les autres quand j'ai testé les conditions
  \item question 6 sur interagir rapidement : LEAP_IN_AR_OUT le moins car imprécision, PHONE_IN_AR_OUT egal avec PHONE_ONLY car interaction connue et maitrisé des participants
  \item question 7 sur la performance : je trouve intéressant que les participants puissent noter leur performance ressentie, pour la comparer avec leur temps et nombre d'erreurs ; les gens vont se sentir un peu moins bon sur LEAP_IN_AR_OUT, un peu plus sur PHONE_IN_AR_OUT
  \item question 8 sur l'effort : je suis un peu plus sceptique sur celle ci, car elle se recoupe avec la question 3 sur la charge mentale et la \item question 4 sur la fatigue physique, mais elle est présente dans le NASA TLX ; je ne sais pas quelle hypothèse formuler sur celle ci
  \item question 9 sur la frustration : je trouve la technique LEAP_IN_AR_OUT est très frustrante, la technique PHONE_IN_AR_OUT est la moins frustrante
  \item question 10 préférence : LEAP_IN_AR_OUT > PHONE_IN_AR_OUT > PHONE_ONLY car effet de nouveauté
  \item question 11 si usage pro : PHONE_IN_AR_OUT > PHONE_ONLY > LEAP_IN_AR_OUT (rapport à la performance et à la nouveauté)
\end{itemize}

\section{Résultats}