\chapter{Conception d'un visiocasque de RA à large champs de vision}
\label{ch:methodology}

\section{Motivation}
Pour concevoir un VESAD, nous avons besoin d'un visiocasque avec champs de vision suffisament large pour visualiser complètement l'écran étendu, le cas contraire limite l'intérêt de notre concept. Le second sous-problème de ce travail de recherche a donc été de développer un visiocasque de RA à large champs de vision.

Un téléphone est tenu en moyenne à une distance $D=\SI{34}{\cm}$ \citep{Bababekova2011} de la tête. Ainsi, si l'on souhaite étendre un téléphone à l'équivalent d'un écran {\NoAutoSpacing 16:9} de \SI{24}{\inch}, soit une taille de $(L,H)=\SI{53x30}{\cm}$, on peut calculer le champs de vision $(FoV_x,FoV_y)$ nécessaire pour chaque \oe il pour le visualiser complètement \reffigureETSp{Fov.png}, avec l'\autoref{eq:fov} : il doit être d'au moins \SI{76x48}{\degree}.

\begin{equation}
  \label{eq:fov}
  \left \{
  \begin{array}{r c c c c c l}
    FoV_x & = & 2 \arctan (\frac{L}{2 \times D}) & = & 2 \arctan (\frac{53}{2 \times 34}) & = & \ang{76}\\
    FoV_y & = & 2 \arctan (\frac{H}{2 \times D}) & = & 2 \arctan (\frac{30}{2 \times 34}) & = & \ang{48}
  \end{array}
  \right .
\end{equation}

\figureETS{Fov.png}{
  Représentation simplifiée du champs de vision (en anglais : \texten{Field of View}), qui peut être mesuré en degrés horizontalement ($FoV_x$), verticalement ($FoV_y$) ou en diagonale. Comme le décrit l'\autoref{eq:fov}, pour qu'un plan soit visible dans un champs de vision donné, il doit être placé à une certaine distance de la caméra ou de l'\oe il.
}

Nous avons d'abord voulu utiliser le Microsoft HoloLens, un visiocasque optique de RA avec une très courte latence, une excellente résolution et une très bonne documentation et un support natif sur le moteur de jeu Unity. Cependant, son champs de vision de \SI{30x17.5}{\degree} pour chaque \oe il \citep{Kreylos2015} est trop restreint, ne permettant de visualiser seulement l'équivalent d'un écran {\NoAutoSpacing 16:9} de \SI{7}{\inch}. Il n'existe en fait actuellement aucun visiocasque de RA sur le marché avec un grand champs de vision \citep[p. 25]{Millette2016}.


\section{Solution retenue}
\label{sec:prototype}

\subsection{Fonctionnement du visiocasque}
\label{subsec:prototype_operation}
Nous avons donc réalisé notre propre prototype de visiocasque similaire à l'AR-Rift \citep{Steptoe2013} : c'est un visiocasque de RA vidéo de conception simple qui a fait ses preuves, utilisé par \cite{Steptoe2014} et \cite{Piumsomboon2014}. Il consiste à coller une caméra physique \emph{stéréoscopique} (avec deux objectifs capturant deux images à la fois, une pour chaque \oe il) à un visiocasque de RV. Il permet en plus de localiser les mains de l'utilisateur.

Notre visiocasque fonctionne, de manière similaire à \cite{Steptoe2013}, sur le principe suivant :
\begin{enumerate}
  \item La caméra stéréoscopique physique filme l'environnement réel de l'utilisateur \reffigureETSp{ArRiftMarker_1.jpg}.
  \item Un algorithme va corriger les déformations des deux images de cette caméra \reffigureETSp{ArRiftMarker_2.jpg}.
  \item Une caméra stéréoscopique virtuelle d'un logiciel de rendu 3D va filmer des éléments virtuels ainsi que les deux images de la caméra physique en arrière-plan \reffigureETSp{ArRiftMarker_3.jpg}.
  \item Les deux images sont affichées dans le visiocasque, une pour chaque \oe il \reffigureETSp{ArRiftMarker_4.jpg}.
\end{enumerate}
\bigskip

\figureLayoutETS{ArRiftMarker}{%
  \parbox{0.72\textwidth}{% Allow line breaks
    \centering%
    \subfigureETS{ArRiftMarker_1.jpg}{Image non corrigée : l'image présente des distorsions importantes (les lignes droites sont courbées).}%
    \figurehspace%
    \subfigureETS{ArRiftMarker_2.jpg}{Image corrigée : les lignes droites le sont à nouveau.}%
    \\%
    \subfigureETS{ArRiftMarker_3.jpg}{Image corrigée à travers Unity : l'image est affichée en arrière-plan de l'environnement virtuel filmé par la caméra virtuelle (champs de vision en lignes blanches) pour l'\oe il gauche. Les caméras physique et virtuelle sont alignées pour faire coïncider environnements virtuels et réels.}%
    \figurehspace%
    \subfigureETS{ArRiftMarker_4.jpg}{Image corrigée et augmentée : un objet virtuel se trouve par dessus chacun des trois marqueurs.}%
  }%
}{
  Vue de l'image capturée par l'objectif gauche de l'Ovrvision Pro, puis corrigée et augmentée pour être affichée dans le visiocasque.
}

Pour donner l'illusion que les éléments virtuels sont alignés avec l'environnement réel, nous devons donc \emph{aligner} la caméra virtuelle avec la caméra physique, c'est-à-dire :
\begin{enumerate}
  \item \emph{Étalonner} (\texten{calibrate} en anglais) la caméra physique, c'est-à-dire mesurer ses \emph{paramètres intrinsèques} (ses propriétés, comme son champs de vision), et les \emph{distorsions} de ses objectifs (les déformations sur les images capturées). L'étalonnage est à faire une seule fois sur une caméra.
  \item \emph{Corriger} (\texten{undistort} en anglais) les distorsions à partir des résultats de l'étalonnage pour chaque image capturée \reffigureETSp{ArRiftMarker_2.jpg}.
  \item Configurer la caméra virtuelle avec les mêmes paramètres intrinsèques et y ajouter en arrière plan les images corrigées  \reffigureETSp{ArRiftMarker_3.jpg}.
\end{enumerate}
\bigskip

La caméra virtuelle ainsi configurée, il suffit de placer (position, rotation) les éléments virtuels s'ils faisaient partie de l'environnement réel. Pour savoir où les placer, on utilise un algorithme de \emph{suivi de marqueurs} (\texten{fiducial marker tracking} en anglais) : ce sont des codes-barres en 2D (deux dimensions) que l'on place dans l'environnement réel \reffigureETSp{ArRiftMarker_1.jpg} et dont on peut déterminer très rapidement la position et l'orientation par rapport une caméra physique dont on connait les paramètres intrinsèques \cite{Garrido-Jurado2014}. On place alors les éléments virtuels à ses mêmes positions et orientations par rapport à la caméra virtuelle \reffigureETSp{ArRiftMarker_3.jpg}. Les marqueurs étant détectés en temps réels, on a donc l'illusion que les éléments virtuels font partie de l'environnement réel \reffigureETSp{ArRiftMarker_4.jpg}.

La \reffigureETS{ArRiftMarker} montre ce processus pour une caméra \emph{monoscopique} (avec un seul objectif). Pour une caméra stéréoscopique, on mesure en plus la différence de position et d'orientation entre les deux objectifs lors de l'étalonnage. On crée ensuite deux caméras monoscopiques virtuelles placées de la même manière que les deux objectifs de la caméra physique. Chaque caméra virtuelle va alors filmer le même environnement virtuel, d'un point de vue légèrement différent, mais avec seulement une image d'un des deux objectifs en arrière-plan.

\subsection{Procédure de réalisation}
\label{subsec:prototype_procedure}
Pour realiser notre visiocasque, nous avons donc suivis la procédure suivante, basée sur celle de \cite{Steptoe2013} :
\begin{enumerate}
  \item Sélection d'une caméra physique stéréoscopique, d'un visiocasque de RV et d'un moteur 3D \autorefp{subsec:technical_choices}.
  \item Réalisation de la bibliothèque d'étalonnage et de RA par suivi de marqueurs Aruco Unity \autorefp{sec:aruco_unity}.
  \item Alignement des caméras physique et virtuelles \autorefp{sec:cameras_alignment}.
\end{enumerate}
\bigskip

Si \citeauthor{Steptoe2013} détaille correctement la première étape de la procédure ci-dessus, il n'expose que le principe de la deuxième étape sans solution accessible pour la localisation d'éléments réels et n'explique pas correctement la troisième étape. Nous souhaitons donc détailler un peu plus dans ce chapitre.

\subsection{Choix techniques}
\label{subsec:technical_choices}
Nous avons tout d'abord choisis l'Ovrvision Pro comme caméra physique stéréoscopique (\url{http://ovrvision.com/}). Annoncée par son comme solution clé en main pour réaliser un visiocasque de RA, elle prévue pour fonctionner avec le visiocasque de RV Oculus DK2 que nous avions déjà à disposition dans notre laboratoire. Composée de deux objectifs \texten{fisheye} (en \oe il de poisson) d'une définition de \SI{960x950}{\px} et un champs de vision de \SI{100x98}{\degree}, ses caractéristiques correspondent bien à ceux de l'Oculus DK2 \autorefp{tab:visual_densities}. Elle est de plus intégrée avec les moteurs de jeu standards dans l'industrie : Unity et Unreal Engine. Son installation se fait simplement en la collant sur la face avant de l'Oculus DK2, comme le montre la \reffigureETS{ArRift_1.jpg}, et en le connectant par USB 3.0 au PC.

\figureETS[0.6]{ArRift_1.jpg}{
  Notre prototype de visiocasque de RA, basé sur le concept de l'AR-Rift de \cite{Steptoe2013} : il est composé du visiocasque de RV Oculus DK2, de la caméra stéréoscopique Ovrvision Pro et du capteur de reconnaissance des mains Leap Motion (sous la caméra).
}

Nous souhaitions ensuite rapidement prototyper sur notre visiocasque. Nous avons alors choisis le moteur de jeu Unity : gratuit (mais au code source propriétaire), il est le standard dans l'industrie du jeu-vidéo pour prototyper et supporte notre caméra et le Leap Motion. Il est de fait très avantageux d'investir un peu de temps dans l'apprentissage d'un moteur de jeu, car il prends complètement en charge le rendu 3D, la simulation de la physique, les entrées sur clavier, souris ou écran tactile, l'affichage dans les visiocasques de RV, un support réseau et propose de nombreuses fonctions mathématiques. Unity est simple plus simple à prendre en main que son concurrent l'Unreal Engine, par son GUI intuitif et l'utilisation de scripts C\#, un langage haut niveau efficace. Enfin, nous avions une expertise dans le laboratoire avec un projet de quatre mois sur le Microsoft HoloLens et la maîtrise de \cite{Millette2016}.

Après plusieurs essais de configuration et une lecture du code source de l'Ovrvision Pro (\url{https://github.com/Wizapply/OvrvisionPro/}), nous avons constaté que la bibliothèque fournie était non fonctionnelle : (1) les images affichées pour chaque \oe il dans le casque étaient décalées, rendant le visiocasque particulièrement inconfortable à utiliser et (2) un décalage important était présent entre le contenu virtuel et l'environment réel. Nous avons donc besoin de ré-étalonner cette caméra et d'une bibliothèque de RA.

Enfin, nous avons donc réalisé la bibliothèque Aruco Unity (\url{https://github.com/NormandErwan/ArucoUnity/}) pour amener de la RA sur notre visiocasque. Basée sur la bibliothèque libre de vision par ordinateur OpenCV (\url{https://opencv.org/}), elle est la seule bibliothèque libre de RA par suivi de marqueurs sous Unity supportant plusieurs types de caméras : monoscopiques ou stéréoscopiques, avec des objectifs \texten{fisheye} ou << classiques >>, dit \emph{rectilinéaires} (similaire à la vision humaine). Elle est également la seule à permettre l'étalonnage de tout ces types de caméras.


\subsection{Discussion et limites}
\label{subsec:solution_discusion}
Nous pourrions simplifier le visiocasque en utilisant une caméra monoscopique, et afficher l'image capturée aux deux yeux. Cependant, comme le souligne \cite{Bourke1999}, la vision stéréoscopique est l'un des principal indice utilisé par le cerveau pour percevoir la profondeur : les yeux étant séparés horizontalement par une distance appelée écart pupillaire, chaque \oe il perçoit une image légèrement différente \reffigureETSp{Ovrvision} permettant au cerveau de percevoir en 3D. C'est pourquoi nous choisissons d'utiliser une caméra stéréoscopique.

\figureLayoutETS{Ovrvision}{%
  \subfigureETS[0.2]{Ovrvision_1.jpg}{Vue de l'objectif gauche.}%
  \figurehspace%
  \subfigureETS[0.2]{Ovrvision_2.jpg}{Vue de l'objectif droit.}%
}{
  Images corrigées de l'Ovrvision Pro : les deux objectifs sont décalés horizontalement d'environ \SI{60}{\mm} pour simuler un écart pupillaire humain.
}

Ensuite, si notre visiocasque est relativement simple à concevoir, son principal inconvénient est la faible densité visuelle de l'image affichée. La densité visuelle permet de mesurer la finesse d'affichage d'une image et se calcule à partir de la définition horizontale $l$ et le champs de vision : $Densite = \frac{l}{FoV_x}$ \citep{Boger2017}. Ainsi, sur les visiocasques de RV du marché et sur l'Ovrvision Pro, la caméra stéréoscopique que nous utilisons, elle est limitée à environ \SI{10}{\ppd} (pixels par degré) \autorefp{tab:visual_densities}. En comparaison, le Microsoft HoloLens a une densité visuelle d'environ \SI{42}{\ppd} quand la fovéa d'un \oe il humain est de \SIrange{60}{80}{\ppd} en moyenne \citep{Kistner2014}. Nous sommes donc face à une limite technologique : une meilleure qualité des images des caméras serait limitée par le visiocasque de RV, quelque qu'il soit. Une définition de 8K (\SI{7680x4320}{\px}), à champs de vision égal, par \oe il est alors nécessaire pour atteindre la même densité visuelle : $Densite_{8K} = 7680 / 110 = \SI{69.8}{\ppd}$.

\begin{tableETS}{tab:visual_densities}{Caractéristiques d'affichage de l'Ovrvision Pro et de visiocasques de RV et RA}
  \begin{tabular}{| C{3.5cm} | C{3.25cm} | C{3.25cm} | C{3cm} |}
    \hline
    \textbf{Nom} & \textbf{Définition\newline(pour chaque \oe il)} & \textbf{Champs de vision\newline(pour chaque \oe il)} & \textbf{Densité visuelle}\\
    \hline
    Ovrvision Pro & \SI{960x950}{\px} & \SI{100x98}{\degree} & \SI{9.6}{\ppd}\\
    \hline
    Oculus DK2 & \SI{960x1080}{\px} & \SI{94x105}{\degree} & \SI{10.2}{\ppd}\\
    \hline
    Oculus Rift & \SI{1080x1200}{\px} & \SI{94x93}{\degree} & \SI{11.5}{\ppd}\\
    \hline
    HTC Vive & \SI{1080x1200}{\px} & \SI{110x113}{\degree} & \SI{9.8}{\ppd}\\
    \hline
    Microsoft HoloLens & \SI{1268x720}{\px} & \SI{30x17.5}{\degree} & \SI{42.3}{\ppd}\\
    \hline
  \end{tabular}
\end{tableETS}

Une seconde limite évidente est la portabilité de notre visiocasque \reffigureETSp{ARRift_2.jpg} : il est volumineux et dépendant d'un PC demandant une bonne puissance de calcul. Cela reste un prototype de laboratoire qui convient à notre recherche, mais un visiocasque pour usage professionnel voire personnel devra dépasser ces problématiques. Correctement optimisé, un visiocasque de RA ne demande pas énormément de ressources : le Microsoft HoloLens est portable et a une très bonne autonomie. De même, il est à parier que les futures visiocasques de RV Oculus et Vive le seront aussi. De même, les objectifs de la caméra peuvent être miniaturisés et intégrés dans le visiocasque, comme le fait le HoloLens déjà.

\figureETS[0.6]{ARRift_2.jpg}{
  Notre prototype de visiocasque de RA porté par un utilisateur : il est assez volumineux et n'est pas portable. Un câble à l'arrière de la tête de l'utilisateur relie le visiocasque au PC. Des marqueurs infrarouges en bas du visiocasque sont illuminés.
}

En outre, notre visiocasque souffre également d'un flux vidéo peu fluide à 30 images par secondes (\texten{frames per second} ou FPS en anglais). Nous sommes limités artificiellement à cette vitesse par le visiocasque de RV Oculus que nous utilisons : en effet, il impose aux applications d'atteindre 60 FPS mais les bride dans le cas contraire. Dans notre cas, la capture des images de l'Ovrvision puis la correction des distorsions ainsi que le suivi de marqueurs sont des opérations trop lourdes en calculs pour le processeur. Les performances étaient plus faibles encore à l'origine ; nous décrivons dans la \autoref{subsec:aruco_unity_scripts} quelques optimisations que nous avons mis en place. 30 FPS nous semblant acceptable et par manque de temps, nous en sommes restés là.

\figureETS[0.6]{Mur-Artal2017.jpg}{
  Carte d'une pièce par localisation et cartographie simultanées (SLAM) et les différentes positions passées, en bleu, de la caméra. Dressée en temps réel, cette carte permet à la caméra de s'y situer et d'y intégrer de la RA.\\
  Tiré de \cite{Mur-Artal2017}.
}

Enfin, l'utilisation d'une solution de RA par suivi de marqueurs est bonne pour du prototypage car elle est techniquement simple à mettre en \oe uvre mais elle est inadéquate pour un produit industriel ou grand public. En effet, le placement de marqueurs sur les objets que l'on veut détecter, comme un téléphone, est une étape fastidieuse. C'est pourquoi le Microsoft HoloLens et les bibliothèques ARKit pour les téléphones iOS et ARCore pour les téléphones Android préfèrent utiliser des technologies de localisation et cartographie simultanées, ou \texten{Simultaneous Localisation And Mapping} (SLAM) en anglais. Un algorithme de SLAM permet à l'appareil de se situer en temps réel dans son environnement et de l'augmenter en dressant une carte de cette pièce \reffigureETSp{Mur-Artal2017.jpg}. Pourtant cette méthode n'est pas adaptée pour détecter de petits objets en mouvement, comme un téléphone, nous ne pouvions alors pas l'utiliser.

\figureETS[0.6]{Bradski2008_1.jpg}{
  Carte de profondeur d'une tasse sur une chaise, reconstituée par vision stéréoscopique.\\
  Adapté de \cite{Bradski2008}.
}

\figureETS[0.6]{Steptoe2013.jpg}{
  Visiocasque de \citeauthor{Steptoe2013} : deux caméra sont collées sur un visiocasque de RV, et des marqueurs infrarouges sous forme de boules grises permettent de suivre la position et l'orientation du visiocasque et de sa main.\\
  Adapté de \cite{Steptoe2013}.
}

D'autres alternatives existent. Nous aurions pu par exemple nous appuyer sur la différence entre les deux images de la caméra pour faire de la vision stéréoscopique : de manière similaire à une méthode SLAM, on peut construire une carte de profondeur de ce qui est vu par la caméra \reffigureETSp{Bradski2008_1.jpg}. Cette méthode nous a cependant semblée trop imprécise. Une troisième alternative serait d'utiliser des marqueurs infrarouges, comme sur les visiocasques de RV. Une ou deux caméras infrarouges filment le visiocasque contenant de nombreux marqueurs insérés sous sa coque \reffigureETSp{ARRift_2.jpg} et, connaissant la structure 3D de ces marqueurs, on détermine à partir des images précisément la position et l'orientation du visiocasque. Des systèmes équivalents, dit de capture de mouvements (\texten{motion capture}), comme Vicon, MotionAnalysis ou OptiTrack sont excellents pour prototyper tout en conservant une précision inférieure au millimètre \reffigureETSp{Steptoe2013.jpg}. Étant malheureusement très chers, nous avons également écarté ces solutions.


\section{Réalisation de la bibliothèque de réalité augmentée Aruco Unity}
\label{sec:aruco_unity}

\subsection{Motivation}
\label{subsec:aruco_unity_motivation}
Unity est un moteur de jeux-vidéos ainsi qu'une excellente solution pour développer avec des visiocasques de RV. Cependant, nous avons besoin de fonctionnalités manquantes pour permettre de la RA avec notre visiocasque. La bibliothèque OpenCV répond parfaitement à nos besoin, mais étant écrite en C++, elle est incompatible avec Unity utilisant le C\#. Nous avons donc réalisé la bibliothèque Aruco Unity pour rendre disponible OpenCV dans Unity. Son utilisation est décrite dans la \autoref{sec:cameras_alignment}.

La solution que nous avons retenue est d'utiliser le système de greffons (\texten{plugin} en anglais, \url{https://docs.unity3d.com/Manual/Plugins.html}) d'Unity pour appeler des bibliothèques externes en C\# ou en C depuis nos scripts. Cette technique, appelée \emph{liaison} (\texten{binding} en anglais) est classique dans les domaines des jeux-vidéos et des simulations physiques : elle permet de réutiliser du code existant sans avoir à le transposer dans un autre langage et surtout d'optimiser des portions de code critiques, le C++ étant conçu pour la performance.

Le code source d'OpenCV étant libre, une alternative aurait pu être de ré-écrire en C\# les fonctionnalités dont nous avions besoin. Le code source du module aruco nous semblait assez court pour cela, étant lui même une ré-implémentation propre de la bibliothèque originale (\url{https://www.uco.es/investiga/grupos/ava/node/26}) proposée par \cite{Garrido-Jurado2014}. Une ré-écriture des fonction d'étalonnage et de correction de caméra nous a semblé par contre trop risquée, n'ayant pas une bonne maîtrise de la théorie mis en \oe uvre. De plus, de nombreux projets de liaison existent vers d'autres langages comme Python (\url{https://docs.opencv.org/master/d6/d00/tutorial_py_root.html}), Javascript (\url{https://docs.opencv.org/master/d5/d10/tutorial_js_root.html}) ou Java (\url{https://opencv.org/platforms/android/}). L'écriture d'une liaison de C++ vers C\# semble donc plus sûre qu'une ré-écriture.

Enfin, quelques bibliothèques de RA par suivi de marqueurs sont déjà disponibles pour Unity, mais aucune ne nous a convenu :
\begin{itemize}
  \item Vuforia (\url{https://unity3d.com/fr/partners/vuforia}) ne supporte pas notre caméra et le code source est propriétaire donc non modifiable.
  \item ARToolKit (\url{https://github.com/artoolkit/arunity5}) ne supporte pas notre caméra non plus, mais malgré un code source libre, le projet est abandonné et il semble difficile d'y ajouter les fonctions d'étalonnage pour une caméra stéréoscopique avec objectifs \texten{fisheye}.
  \item OpenCV for Unity (\url{https://enoxsoftware.com/opencvforunity/}) propose des liaisons vers tous les modules d'OpenCV dont nous avons besoin, mais nous craignons de manquer de flexiblité avec son code source propriétaire si nous devons apporter des modifications dans le code source d'OpenCV.
\end{itemize}

\subsection{Architecture}
\label{subsec:aruco_unity_architecture}

\figureLayoutETS{ArucoUnityComponents}{
  \begin{tikzpicture}
    \begin{umlcomponent}{ArucoUnity}
      \umlbasiccomponent[name=Liaison]{Liaison C-C\#}
      \umlbasiccomponent[x=6, name=Scripts]{Scripts C\#}
      \umlassemblyconnector[interface={Exposer OpenCV}, name=SL]{Scripts}{Liaison}
    \end{umlcomponent}
    \umlbasiccomponent[name=ArucoUnityPlugin, y=-5]{Greffon ArucoUnity}
    \umlassemblyconnector[interface={Appeler l'interface C du greffon}, name=LP, anchors=-90 and 90]{Liaison}{ArucoUnityPlugin}
    \umlbasiccomponent[x=6, y=-5]{OpenCV}
    \umlassemblyconnector[interface=Appeler]{ArucoUnityPlugin}{OpenCV}
  \end{tikzpicture}
}{
  Diagramme des composants d'ArucoUnity : le greffon expose OpenCV dans une interface en C, appelée par la liaison qui expose une interface orienté objet en C\# similaire à celle d'OpenCV. Les scripts permettent de travailler directement dans l'interface d'Unity avec les fonctionnalité d'OpenCV.
}

Aruco Unity est composé de trois couches logicielles, s'appuyant sur OpenCV \reffigureETSp{ArucoUnityComponents}. Tout d'abord, nous avons créé une bibliothèque exposant les modules aruco, calib3d et ccalib d'OpenCV avec une interface en C (\url{https://github.com/NormandErwan/ArucoUnity/tree/master/ArucoUnityPlugin/}), que l'on peut appeler depuis C\#. Nous avons ensuite créé un projet Unity et placé la bibliothèque dans le dossier \code{Assets/Plugin} : elle est alors détectée comme un greffon par Unity \reffigureETSp{ArucoUnityPlugin.jpg}.

\figureETS{ArucoUnityPlugin.jpg}{%
  Le greffon d'ArucoUnity exposant les modules aruco, calib3d et ccalib d'OpenCV avec une interface en C dans le projet Unity d'Aruco Unity. Les autres modules présents sont des dépendances.
}

Dans le projet, nous avons ensuite écrit les scripts de liaison (\url{https://github.com/NormandErwan/ArucoUnity/tree/master/Assets/ArucoUnity/Scripts/Plugin/}) : ils reproduisent en C\# l'interface orientée objet d'OpenCV en appelant en arrière-plan notre greffon. On peut alors utiliser en toutes les fonctionnalités d'OpenCV dans Unity de manière similaire qu'en C++.

Enfin, nous avons écrit des scripts C\# pour Unity que nous utilisons dans la \autoref{sec:cameras_alignment}. Nous avions d'abord écrit de simples scripts séquentiels d'étalonnage et de suivi de marqueur, mais le besoin s'est vite faite sentir de les rendre plus robustes et performants. De plus, le travail dans Unity se fait préférentiellement en deux temps : les développeurs programment les fonctionnalités dans les scripts, et les artistes les utilisent et configurent à volonté avec l'interface graphique de Unity sans avoir à toucher au code source.

\subsection{Réalisation du greffon et de la liaison}
La première étape de réalisation a donc été le greffon pour Unity, pour rendre accessible les fonctions des modules aruco, calib3d et ccalib d'OpenCV. Il existe plusieurs techniques de liaison. Nous avons choisis la plus simple : P/Invoke (\texten{Platform Invocation Services}). Le principal défi dans son utilisation est la rigueur de la gestion de la mémoire : d'une part on gère manuellement la création et la destruction d'objets et d'autre part le greffon et les scripts C\# doivent se partager cette mémoire.

\begin{listingETS}{cpp}{lst:aruco_unity_plugin_cpp}{Interface simplifiée en C autour de la classe Mat d'OpenCV dans le greffon d'ArucoUnity.}
  extern "C" {
    Mat* au_cv_Mat_new1() {
      return new Mat();
    }

    Mat* au_cv_Mat_new2(int rows, int cols, int type) {
      return new Mat(rows, cols, type);
    }
    
    void au_cv_Mat_delete(Mat* mat) {
      delete mat;
    }

    int au_cv_Mat_total(Mat* mat) {
      return mat->total();
    }
  }
\end{listingETS}

L'\autoref{lst:aruco_unity_plugin_cpp} montre un exemple simplifié de l'interface en C la classe \code{Mat}, qui permet de manipuler matrices et images dans OpenCV. On utilise en fait le patron de conception adaptateur : chaque constructeur, destructeur et fonction est enveloppé par une fonction C. On nomme cette fonction \code{new} pour les constructeurs, \code{delete} pour les destructeurs ou avec le nom de la fonction appelée. Le polymorphisme n'étant pas supporté en C, chaque surcharge est envelopée par une fonction : on les numérote alors avec un suffixe de $1$ à $n$. Le C ne comportant pas non plus d'espace de nom, on ajoute à chaque fonction le préfixe \code{au_cv_} suivi du nom de la classe pour éviter des collisions de noms. Enfin, on entoure toutes les fonctions avec l'instruction \code{extern "C"} pour indiquer au compilateur et à l'éditeur de liens que nous travaillons bien avec des fonctions en C.

Le passage de paramètres est la problématique principale avec cette technique. Certains types primitifs de C\# comme \code{int} ou \code{float} ont des équivalences en C++, on peut alors les passer directement en paramètres au greffon. Cependant, d'autres types plus complexes sont représentés différement dans les deux languages, par exemple les classes, les chaînes de caractères \cite{Peterson2015}. Nous avons alors voulu faire simple : tous les objets OpenCV sont alloué avec le code C++, via les fonctions C pour la construction et la destruction que nous venons de décrire. Elles retournent alors un pointeur (qui n'est qu'un entier sur une adresse mémoire) au code C\#. Quand on souhaite exécuter une fonction sur un objet, il suffit de simplement y passer son pointeur comme paramètre. Sans rigueur, il y a plus de risques d'erreurs de segmentation ou de fuites de mémoire mais notre couche de liaison prend en charge cette responsabilité sans que l'utilisateur n'ait à s'en soucier.

\begin{listingETS}{cs}{lst:aruco_unity_plugin_cs}{Liaison simplifiée pour exposer en C\# la classe Mat d'OpenCV.}
  public class Mat
  {
    [DllImport("ArucoUnityPlugin")]
    static extern IntPtr au_cv_Mat_new1();

    [DllImport("ArucoUnityPlugin")]
    static extern IntPtr au_cv_Mat_new2(int rows, int cols, int type);

    [DllImport("ArucoUnityPlugin")]
    static extern void au_cv_Mat_delete(IntPtr mat);
    
    [DllImport("ArucoUnityPlugin")]
    static extern int au_cv_Mat_total(IntPtr mat);

    public IntPtr Ptr { get; }

    public Mat()
    {
      Ptr = au_cv_Mat_new1();
    }

    public Mat(int rows, int cols, int type)
    {
      Ptr = au_cv_Mat_new2(rows, cols, type);
    }

    ~Mat()
    {
      au_cv_Mat_delete(Ptr);
    }

    public int Total()
    {
      return au_cv_Mat_total(Ptr);
    }
  }
\end{listingETS}

L'\autoref{lst:aruco_unity_plugin_cs} montre la liaison associée à l'interface de l'\autoref{lst:aruco_unity_plugin_cpp}. Elle est composé de deux parties. Tout d'abord, il y a les appels au greffon : on déclare des fonctions statiques avec la même signature (nom et paramètres) que celles du greffon et on indique au compilateur de les appeler dans le greffon avec l'attribut \code{DllImport}. Ensuite, on construit une interface orienté objet reproduisant celle d'OpenCV : on utilise le patron de conception façade pour masquer la liaison qui est complexe à utiliser et source d'erreurs de segmentations ou de fuites de mémoire si mal utilisée. La classe \code{Mat} reconstruite en C\# appelle donc automatiquement un des constructeurs C++ quand l'utilisateur désire en créer une instance, par exemple \code{var mat = new Mat()}. Le pointeur vers la classe C++ sous-jacente est stocké dans l'instance C\# correspondante.

\begin{listingETS}{cs}{lst:aruco_camera_cs}{Classe C\# ArucoCamera simplifée pour passer une image à OpenCV.}
  public class ArucoCamera : MonoBehaviour, IArucoCamera
  {
    public Mat Image { get; protected set; }
    public Texture2D ImageTexture { get; protected set; }
    public byte[] ImageData { get; protected set; }

    // Exécuté au démarrage
    protected virtual void Start()
    {
      Image = new Mat(width, height, Format.8UC3); // Format RVB : 3 octets par pixels (3x8=24 bits)
      ImageTexture = new Texture2D(width, height, TextureFormat.RGB24);
      ImageData = new byte[3 * Images[cameraId].Total()];
      Image.DataByte = ImageData;
    }

    // Exécuté à chaque frame
    protected virtual void Update()
    {
      UpdateImages(); // Copie de l'image de la caméra dans ImageData

      ... // Correction des images et suivi des marqueurs

      // Mise à jour de l'image Unity
      ImageTexture.LoadRawTextureData(Image.DataIntPtr, 3 * Images[cameraId].Total());
      ImageTexture.Apply();
    }

    protected abstract void UpdateImages();
  }
\end{listingETS}

\begin{listingETS}{cs}{lst:ovrvision_aruco_camera}{Classe C\# OvrvisionArucoCamera simplifiée.}
  public class OvrvisionArucoCamera : StereoArucoCamera
  {
    [DllImport("ovrvision"]
    static extern int ovOpen(int cameraId, float markerLength, int mode);
    ... // Autres liaisons

    protected override void Start()
    {
      base.Start();
      ovOpen(cameraId: 0, markerLength: 1, mode: CameraMode.VR_960x950_60FPS);
      ovSetCamSyncMode(false);
    }

    protected override void UpdateImages()
    {
      ovPreStoreCamData(ProcessingMode.Demosaic); // Retourner les images couleurs non corrigées
      ovGetCamImageRGB(ImageDatas[0], 0); // Copie image oeil gauche
      ovGetCamImageRGB(ImageDatas[1], 1); // Copie image oeil droit
    }
  }
\end{listingETS}

Le code C++ travaille par instants avec les images capturées par la caméra, cette fois allouées en mémoire dans le code C\#. On passe alors simplement un pointeur de l'image, ainsi que sa taille, aux fonctions d'OpenCV. Concrètement, on crée un objet \code{Mat} dans OpenCV et un objet \code{Texture2D} dans Unity pour manipuler l'image ainsi qu'un tableau partagé entre les deux objets pour représenter le contenu de l'image. On utilise le format le plus courant pour encoder l'image dans le tableau : chaque pixel est représenté en couleurs rouge, vert, bleu (RVB) avec un octet (8 bits) par couleur, soit un total de $3 \times 8 = 24\text{ bits}$ par pixel. On travaille donc avec une paire d'images de $2 \times 3 \times 960 \times 950 = \SI{5.472}{\mega\byte}$ dans le cas de l'Ovrvision, ce qui est assez conséquent pour les opérations de suivi de marqueurs et de corrections d'images. C'est pourquoi on partage la mémoire pour éviter des copies entre Unity et le greffon, coûteuses en temps de processeur.

Nous avons aussi besoin d'accéder à la caméra Ovrvision Pro. La caméra est fonctionnelle, les deux images retournées sont synchronisées mais, comme nous l'avons expliqué dans la \autoref{subsec:technical_choices}, son étalonnage et le suivis de marqueurs ne sont pas bons avec les scripts C\# fournis par le constructeur. Un greffon similaire au nôtre étant disponible, nous avons développé la classe \code{OvrvisionArucoCamera} de liaison \autorefp{lst:ovrvision_aruco_camera} pour pouvoir étalonner cette caméra avec Aruco Unity et y intégrer de la RA.

\figureLayoutETS{CoordinatesSystem}{%
  \subfigureETS[0.1]{CoordinatesSystemLeft.png}{Unity : système de coordonnées main gauche.}%
  \figurehspace%
  \subfigureETS[0.1]{CoordinatesSystemRight.png}{OpenCV : système de coordonnées main droite.}%
}{
  Illustrations des systèmes de coordonnées utilisés par Unity et OpenCV ; une inversion de l'axe des Y permet de passer de l'un à l'autre.
}

Nous avons également ajouté quelques fonctions de conversion, Unity et OpenCV utilisant des conventions et des classes différentes pour encoder les transformations d'objets. Pour encoder la position, Unity utilise un système de coordonnées main gauche \reffigureETSp{CoordinatesSystemLeft.png} tandis qu'OpenCV utilise un système main droite \reffigureETSp{CoordinatesSystemRight.png}. On inverse alors l'axe des Y pour passer de l'un à l'autre avec la fonction \code{Vector3 Vec3d.ToPosition()}. En outre, OpenCV utilise des matrices de rotation ou des angles de rotations, selon les fonctions. OpenCV propose des fonctions de conversions, que nous avons retranscrites dans la liaison. On convertit les angles de rotations en quaternions d'Unity suivant l'algorithme de \cite{Baker1998} avec la fonction \code{Quaternion Vec3.ToRotation()}.


\subsection{Réalisation des scripts pour Unity}
\label{subsec:aruco_unity_scripts}

\figureLayoutETS{ArucoUnityArchitecture}{
  \begin{tikzpicture}
    \begin{umlpackage}{Cameras}
      \umlsimpleclass{IArucoCamera}

      \umlsimpleclass[x=4]{IArucoCameraController}
      \umluniassoc{IArucoCameraController}{IArucoCamera}
    \end{umlpackage}
  \end{tikzpicture}
}{Diagramme de classes simplifé de la couche de scripts C\# d'ArucoUnity. Seules les interfaces ou les classes abstraites de haut niveaux sont affichées : dans chaque cas, le patron de conception stratégie est appliqué pour supporter différents types de caméras. Elles font toutes parties de l'espace de nom \code{ArucoUnity}, non affiché ici.}

Ce travail avec les images des caméras et les conversions montre pourquoi nous avons besoin d'une couche supplémentaire de scripts C\# au dessus de celle de la liaison pour faciliter l'usage d'OpenCV dans Unity. Les deux systèmes ne sont en effet pas conçus pour travailler ensemble : cette complexité est ainsi masquée par la façade supplémentaire que nous créons avec les scripts.

Avec ces scripts nous mettons aussi en place le fonctionnement de notre visiocasque décrit à la \autoref{subsec:prototype_operation}. Nous avons d'abord écrit un simple script séquentiel qui (1) configurait au démarrage la caméra virtuelle avec les paramètres de la caméra physique et les images capturées en arrière-plan puis qui (2) corrigeait à chaque \texten{frame} les images et plaçait les objets virtuels, en suivant la procédure que nous décrivons à la \autoref{sec:cameras_alignment}. Nous y avons rapidement appliqué quelques principes de conception pour l'améliorer. Nous utilisons donc une seule responsabilité par classe \autorefp{ArucoUnityArchitecture} :
\begin{itemize}
  \item \code{IArucoCamera} récupère les images de la caméra physique.
  \item \code{IArucoCameraUndistortion} corrige ces images.
  \item \code{IArucoObjectsTracker} et \code{ArucoObject} détectent la position des marqueurs sur les images.
  \item \code{IArucoCameraDisplay} affiche les images et place les objets virtuels.
  \item \code{ArucoCameraCalibration} étalonne la caméra physique.
  \item \code{CameraParameters} gère les paramètres de l'étalonnage.
\end{itemize}
\bigskip

Cette architecture est évidemment simplifiée. Pour chacune de ces interfaces ou classes abstraites, nous utilisons le patron de conception stratégie : chaque implémentation supporte ainsi un type de caméra, permettant de multiples combinaisons. Par exemple, \code{IArucoCamera} est implémenté par \code{ArucoWebcam} et \code{OvrvisionArucoCamera} et peuvent être utilisées indépendamment avec les implémentations de \code{IArucoCameraDisplay} (\code{MonoArucoCameraDisplay} ou \code{StereoArucoCameraDisplay}). Un développeur peut ainsi ajouter facilement le support d'un nouvelle caméra ou la correction d'un nouveau type d'objectif photographique.

\figureLayoutETS{ArucoUnitySequence}{
  \begin{tikzpicture}
    \begin{umlseqdiag}
      \umlobject{ArucoCamera}
      \umlobject{ArucoCameraUndistortion}
      \umlobject{ArucoObjectsTracker}
      \umlobject{ArucoCameraDisplay}

      \begin{umlcallself}[op=UpdateImages()]{ArucoCamera}
        \begin{umlfragment}[type=par, inner xsep=3]
          \begin{umlcallself}[op={SwapImages()}]{ArucoCamera}
          \end{umlcallself}
          \umlfpart[thread]
          \begin{umlcallself}[op={GetImages()}]{ArucoCamera}
          \end{umlcallself}
        \end{umlfragment}
      \end{umlcallself}

      \begin{umlcall}[op={UndistortImages()}]{ArucoCamera}{ArucoCameraUndistortion}
        \begin{umlfragment}[type=par, inner xsep=3]
          \begin{umlcallself}[op={SwapImages()}]{ArucoCameraUndistortion}
          \end{umlcallself}
          \umlfpart[thread]
          \begin{umlcallself}[op={Undistort()}]{ArucoCameraUndistortion}
          \end{umlcallself}
        \end{umlfragment}
      \end{umlcall}

      \begin{umlcall}[op={ImagesUpdated()}]{ArucoCamera}{ArucoObjectsTracker}
        \begin{umlfragment}[type=par, inner xsep=3]
          \begin{umlcallself}[op={CopyImages()}]{ArucoObjectsTracker}
          \end{umlcallself}
          \begin{umlcall}[op={PlaceArucoObjects()}]{ArucoObjectsTracker}{ArucoCameraDisplay}
          \end{umlcall}
          \umlfpart[thread]
          \begin{umlcallself}[op={TrackArucoObjects()}]{ArucoObjectsTracker}
          \end{umlcallself}
        \end{umlfragment}
      \end{umlcall}
    \end{umlseqdiag}
  \end{tikzpicture}
}{Diagramme de séquence simplifé de chaque \texten{frame} de la couche de scripts C\# d'ArucoUnity.}

Enfin, nous améliorons les performances en externalisant certains calculs sur plusieurs fils d'exécution (\texten{thread} en anglais). Par défaut, les scripts tournent en effet sur le fil d'exécution principal d'Unity. L'ensemble des scripts doit donc s'exécuter rapidement pour avoir une application fluide, par exemple en moins de \SI{16}{\ms} pour atteindre 60 FPS. La capture des images, leurs corrections et le suivi des marqueurs sont trop les opérations les plus lentes. Nous créons donc un fil d'exécution pour chacune de ces trois calculs ci-dessous : chacun tourne alors de manière indépendante et ne bloque plus celui d'Unity \reffigureETSp{ArucoUnitySequence}. Si le fil d'exécution de la caméra a récupéré de nouvelles images, celui d'Unity va les copier dans les autres fils et recopie les images corrigées pour les afficher dans le visiocasque. Cette solution nous permet d'avoir d'excellentes performances avec une webcam où nous avions à l'origine peu de fluidité, mais n'est pas encore totalement satisfaisante avec l'Ovrvision \autorefp{subsec:solution_discusion} : les multiples copies des images prennent trop de temps.


\section{Alignement des caméras physique et virtuelle}
\label{sec:cameras_alignment}

\subsection{Introduction}
\label{subsec:cameras_alignment_introduction}
Notre bibliothèque d'étalonnage et RA par suivi de marqueur Aruco Unity étant fonctionnelle, nous avons pu mettre en place la RA dans notre visiocasque en alignant les caméras physiques et virtuelles. Nous détaillons donc ici le fonctionnement des scripts C\# décrit à la \autoref{sec:sec:aruco_unity}.

Pour cela, nous devons tout d'abord comprendre le principe d'une caméra \autorefp{subsec:camera_theory}, puis le cas le plus simple : l'alignement dans le cas le d'une caméra physique monoscopique avec un objectif rectilinéaire \autorefp{subsec:pinhole_camera_calibration}. Nous itérons ensuite avec des objectifs \texten{fisheye} \autorefp{subsec:fisheye_camera_calibration} puis enfin avec notre caméra stéréoscopique \autorefp{subsec:stereo_camera_calibration}. Nous abordons seulement la théorie nécessaire pour résoudre nos problèmes d'étalonnage, de correction de distorsions et d'alignement ; le lecteur peut se référer à \cite[chap. 18-19]{Kaehler2017} pour une théorie et des procédures plus complètes.

Aruco Unity permet de facilement étalonner une caméra directement dans Unity sans avoir à programmer. Après l'étalonnage, notre bibliothèque de RA corrige et aligne automatiquement les caméras sans que l'utilisateur n'ait à s'en soucier lors du suivi de marqueurs. Nous décrivons dans cette section les actions à faire dans Unity pour étalonner et aligner sa caméra ainsi que le code exécuté en arrière-plan. On utilise donc, via Aruco Unity, les modules d'OpenCV suivants :
\begin{itemize}
  \item aruco pour suivre les marqueurs (\url{https://docs.opencv.org/master/d9/d6a/group__aruco.html}) ;
  \item calib3d pour étalonner et corriger les objectifs rectilinéaires (\url{https://docs.opencv.org/master/d9/d0c/group__calib3d.html}) ;
  \item ccalib pour étalonner et corriger les objectifs \texten{fisheye} (\url{https://docs.opencv.org/master/d3/ddc/group__ccalib.html}).
\end{itemize}

Cette section a son équivalent en anglais sur plusieurs articles dans la documentation en ligne d'Aruco Unity (\url{https://normanderwan.github.io/ArucoUnity/}). Ils sont plus pratiques sans détailler la théorie pour permettre à nouvel utilisateur de rapidement construire son application de RA.

\subsection{Fonctionnement d'une caméra}
\label{subsec:camera_theory}

\subsubsection{Caméra virtuelle}
Une caméra est un dispositif qui capture une vue en 2D d'une scène en 3D. Cette vue est une projection en \emph{perspective} rectilinéaire. Le plus simple des dispositifs de projection en perspective est le \emph{sténopé} (\texten{pinhole} en anglais) : c'est une simple boite percée d'une petite ouverture, le \emph{centre de projection}, noté $O$, sur une de ses face, et d'un écran, \emph{le plan image}, sensible à la lumière sur la face opposée. La scène n'est donc visible que par un unique point de vue : ainsi, chaque point sur le plan image ne provenant que d'un seul rayon de lumière à travers le centre de projection, une image nette et inversée de la scène va se former \reffigureETSp{Pinhole.png}.

\figureETS[0.9]{Pinhole.png}{
  Un sténopé : une petite ouverture sur une face de la boite laisse entrer la lumière, projettant une scène 3D (par exemple les points $P_1$ et $P_2$) en une image inversée sur la face opposée à l'ouverture (les projections respectives sont les points $p_1$ et $p_2$).
}

L'image formée sur le plan image étant toujours nette, on peut modifier le champs de vision du sténopé en déplaçant le plan image le long de l'\emph{axe optique}, la droite perpendiculaire au plan image et passant par le centre de projection, la distance entre le plan image et le centre de projection étant la \emph{longueur focale} $f$. On calcule alors le champs de vision du sténopé avec l'\autoref{eq:fov} avec cette distance $f$, habituellement donnée en millimètres par le constructeur de la caméra. Ainsi, à taille équivalente du plan image, une longueur focale plus courte donne lieu à un plus grand champs de vision.

Pour simplifier nos équations, on peut modifier le modèle du sténopé tout en le gardant valide mathématiquement. On crée un nouveau plan image symétrique au véritable plan image par rapport au centre de projection, placé donc à la même distance $f$ du centre optique. La vue générée sur ce plan image virtuel reste la même mais est non inversée : les rayons de la scène sont toujours projeté vers le centre de projection mais interceptés par ce nouveau plan image \reffigureETSp{PerspectiveProjection.png}. Le repère est placé sur le centre de projection.

\figureETS[0.9]{PerspectiveProjection.png}{
  Projection en perspective rectilinéaire : un point $P=(X,Y,Z)$ va être projeté sur le plan image situé à une distance $z=f$ du centre de projection $O$ en un point $p=(x,y,f)$. Une caméra virtuelle est basée sur ce principe et on peut également simplifier le sténopé à ce modèle.
}

De manière simplifiée, c'est également le fonctionnement d'une caméra virtuelle monoscopique. En suivant l'\autoref{eq:virtual_camera_projection}, on projete chaque point $P=(X,Y,Z)$ de la scène en un point $p=(x,y,z)$ sur le plan image. Les unités sont arbitraires ; Unity utilise par exemple des mètres. En capturant un nombre suffisament élevé d'images par secondes (au moins 15 FPS) avec cette caméra, on a l'illusion de voir une vidéo.

\begin{equation}
  \label{eq:virtual_camera_projection}
  x = f \frac{X}{Z},\qquad y = f \frac{Y}{Z},\qquad z = f
\end{equation}

\subsubsection{Caméra physique}
Le sténopé est un modèle de caméra également suffisant pour décrire une caméra physique monoscopique, en y ajoutant un capteur numérique placé sur le plan image pour enregistrer l'image capturée. 

Cet ajout introduit de nouveaux paramètres que l'on va mesurer lors de l'étalonnage de la caméra \autorefp{subsec:pinhole_camera_calibration}. Comme nous travaillons seulement avec les images capturées dans ce procédé, les mesures sont donc faites en pixels et non en millimètres. De plus, nous devons déplacer le repère dans le coin en bas à gauche du plan image.

Le centre du capteur utilisé ne coïncide pas totalement avec l'axe optique : cela requerrait une précision trop importante et non nécessaire lors de la fabrication de la caméra. L'axe optique va donc intersecter le plan image non pas en son centre, mais sur un point décalé appelé \emph{centre optique} et noté $c$. On mesure en pixels le décalage $(c_x,c_y)$ sur l'image capturée.

En outre, la longueur focale mesurée de la caméra peut être différente sur les deux axes $X$ et $Y$ du plan image. On note $f_x$ et $f_y$ les deux longueurs focales mesurées. Cela est dû a certains capteurs, de mauvais qualité, produisant des pixels rectangulaires et non carrés. On pourrait également les calculer en connaissant la taille physique $(L,H)$ et la définition $(l,h)$ du capteur et la longeur focale physique de la caméra avec l'\autoref{eq:focal_lengths}, mais on n'a pas toujours accès avec fiabilité à toutes ces informations (par exemple avec un téléphone).

\begin{equation}
  \label{eq:focal_lengths}
  f_x = f \frac{L}{l},\qquad f_y = f \frac{H}{h}
\end{equation}

Ces nouveaux paramètres font partie de la \emph{matrice intrinsèque} de la caméra, notée $K$. Elle permet de transformer un point $P$ de la scène est transformé en un point $p'$ \autorefp{eq:projection}, qu'on projette ensuite sur le plan image en divisant ses coordonnées par $Z$ comme dans l'\autoref{eq:virtual_camera_projection}.

\begin{equation}
  \label{eq:projection}
  \begin{pmatrix}
    x'\\
    y'\\
    z'
  \end{pmatrix}
  =
  \underbrace{
    \begin{pmatrix}
      f_x & 0 & c_x\\
      0 & f_y & c_y\\
      0 & 0 & 1
    \end{pmatrix}
  }_\text{K}
  \begin{pmatrix}
    X\\
    Y\\
    Z
  \end{pmatrix}
\end{equation}

On résume donc la projection d'un point $P=(X,Y,Z)$ de la scène par une caméra physique monoscopique en un point $p=(x,y)$ sur l'image en suivant l'\autoref{eq:physical_camera_projection}.

\begin{equation}
  \label{eq:physical_camera_projection}
  x = f_x \frac{X}{Z} + c_x,\qquad y = f_y \frac{Y}{Z} + c_x
\end{equation}

Enfin, une caméra physique monoscopique recquiert également un objectif photographique. En effet, l'ouverture infiniment petite du sténopé ne laissant passer que trop peu de lumière demande infiniment de temps pour qu'une image visible se forme. C'est un problème pour une caméra si l'on souhaite capturer plusieurs images par secondes. La solution est de faire une ouverture plus grande pour laisser passer plus de lumière et de placer donc un objectif photographique pour continuer à concentrer la lumière sur le plan image en une seule image nette. L'inconvénient est que les lentilles utilisées introduisent nécessairement des distorsions en barillet dans l'image \reffigureETSp{Distorsion}. De plus, le capteur et l'objectif ne sont pas forcément bien montés parallèles, entraînant une distorsion tangente supplémentaire, le plan image ne coïncidant pas totalement avec le capteur. D'autres types de distorsions existent mais peuvent être négligés \citep[p. 377]{Bradski2008}. L'ensemble des \emph{paramètres de distorsion} qui sont mesurés est noté $D$. Pour plus de détails sur ces paramètres, nous redirigeons le lecteur vers \cite[p. 375]{Bradski2008}.

\figureLayoutETS{Distorsion}{%
  \subfigureETS[0.15]{Distorsion_1.png}{Aucune distorsion.}%
  \figurehspace[10]%
  \subfigureETS[0.15]{Distorsion_2.png}{Distorsion en barillet, la grille est vue bombée vers l'extérieur.}%
  \figurehspace[10]%
  \subfigureETS[0.15]{Distorsion_3.png}{Distorsion en coussinet, la grille est vue écrasée vers l'intérieur.}%
}{
  Illustrations d'une grille vue à travers un système optique avec différents types de distorsion.
}

Ces distorsions éloignent le fonctionnement de la caméra du modèle idéal du sténopé. C'est pourquoi il est important de corriger les déformations que cela produit sur les images, comme si elles avaient été capturées par un sténopé avec les même paramètres intrinsèques et donc aucune distorsions, pour conserver un bon alignement avec la caméra virtuelle.

\subsection{Étalonnage d'une caméra}
\label{subsec:pinhole_camera_calibration}
Étalonner une caméra monoscopique rectilinéaire consiste à déterminer sa matrice intrinsèque $K$ et ses paramètres de distorsion $D$. Ce procédé fonctionne sur un principe simple. Nous avons décrit les équations qui permettent de projeter un point $P$ de la scène en un point $p$ sur le plan image. Connaissant les coordonnées de ces deux points, nous pouvons donc écrire les vecteurs de translation $T = (t_x, t_y, t_z)$ et rotation $R = (\theta_x, \theta_y, \theta_z)$ permettant de transformer un point $P$ de la scène en un point $p$ sur le plan image. En connaissant assez de couple de points $(P_i, p_i)$, on peut alors poser un ensemble d'équations pour résoudre $K$. Ainsi, les déviations des points $p_i$ du modèle de sténopé ainsi calculé permettent de déterminer $D$.

\figureETS{Bradski2008_2.jpg}{
  Une caméra physique monoscopique est étalonné avec OpenCV en capturant plusieurs images d'un échiquier imprimé sur une planche rigide capturé dans différentes positions et angles de vues.\\
  Tiré de \cite[p. 382]{Bradski2008}.
}

Concrètement, on utilise une image d'un échiquier qu'on capture avec la caméra dans plusieurs positions et angles de vues \reffigureETSp{Bradski2008_2.jpg}, les points utilisés étant les intersections entre les carrés noirs et blancs. On connait par avance les coordonnées relatives des points $P_i$ entre eux (il suffit de le mesurer sur l'échiquier) et ces intersections sont très facile à détecter sur l'image capturée par la caméra. La première étape de l'étalonnage consiste donc à capturer avec la caméra quelques images d'un échiquier dont on connait la configuration.

\figureETS[0.9]{CharucoBoardCreation.jpg}{
  Création d'un échiquier d'étalonnage avec Aruco Unity : à gauche les scripts de configuration, à droite l'échiquier vu dans la scène Unity. Les marqueurs, décrits dans la \autoref{sec:aruco_unity}, dans les carrés blancs permettent l'amélioration de la détection de l'échiquier.
}

On crée alors notre échiquier d'étalonnage \reffigureETSp{CharucoBoardCreation.jpg} :
\begin{itemize}
  \item On crée un projet Unity configuré avec Aruco Unity \autorefp{sec:aruco_unity}.
  \item Dans ce projet, on ajoute un nouvel objet vide.
  \item On ajoute à l'objet le script \code{ArucoCharucoBoard.cs}, qui représente l'échiquier, et on le configure :
  \begin{itemize}
    \item la taille des carrés en pixels avec le champs \code{SquareSideLength} ;
    \item le nombre de carrés avec \code{SquaresNumberX} et \code{SquaresNumberY} ;
    \item la taille des marqueurs avec \code{MarkerSideLength} ;
    \item la marge entre les marqueurs et les carrés blancs qui les contiennent avec \code{MarginLength}.
  \end{itemize}
  \item On ajoute à l'objet le script \code{ArucoObjectCreator.cs}, qui permet de créer l'image de l'échiquier, et on fait pointer \code{ArucoObject} vers \code{ArucoCharucoBoard.cs}.
  \item On démarre la scène Unity : l'image de l'échiquier s'affiche et est enregistrée sur le disque dur dans le dossier du projet (le chemin est indiqué par \code{OutputFolder}).
  \item On imprime l'image que l'on colle fermement sur une surface rigide pour qu'elle reste le plus plate possible.
\end{itemize}

\figureLayoutETS{PinholeCameraCalibration}{%
  \subfigureETS[0.4]{PinholeCameraCalibration_1.jpg}{Sélection de la première caméra disponible (identifiant \code{0} dans le champs \code{WebcamId}).}%
  \figurehspace%
  \subfigureETS[0.4]{PinholeCameraCalibration_2.jpg}{Lien vers l'échiquier utilisé (champs \code{CalibrationBoard}).}%
}{
  Configuration de la scène Unity \code{Calibrate} pour étalonner une caméra monoscopique rectilinéaire.
}

On prépare ensuite la scène de calibration \reffigureETSp{PinholeCameraCalibration}:
\begin{enumerate}
  \item On ouvre la scène Unity \code{Assets/ArucoUnity/Scenes/Calibrate.unity} déjà préparée pour l'étalonnage d'une caméra monoscopique rectilinéaire.
  \item Sur le script \code{ArucoWebcam.cs}, dans l'objet du même nom, on choisit la caméra à étalonner en renseignant son identifiant numérique dans le champs \code{WebcamId}.
  \item On crée à nouveau un objet représentant notre échiquier imprimé, en y configurant cette fois la taille des carrés en mètres, permettant à Aruco Unity de construire automatiquement les positions relatives à l'échiquer des points $P_i$.
  \item Sur le script \code{PinholeCameraCalibration.cs}, dans l'objet du même nom, on fait pointer le champs \code{CalibrationBoard} vers notre objet d'échiquier.
  \item On peut ajuster les paramètres de calibration (\texten{calibration flags} en anglais) du module calib3d avec \code{PinholeCameraCalibrationFlags.cs} et les paramètres de détection du module aruco avec \code{DectectorParametersController.cs}, ces deux scripts ayant par défaut les valeurs recommandées par les documentations de leurs modules respectifs.
\end{enumerate}

\figureETS[0.75]{PinholeCameraCalibration_3.jpg}{%
  Scène \code{Calibrate} jouée pour étalonner une webcam. Les marqueurs de l'échiquier sont surlignés pour indiquer s'il est correctement détecté.
}

On peut maintenant débuter l'étalonnage en jouant la scène \reffigureETSp{PinholeCameraCalibration_3.jpg}. Le flux vidéo de la caméra est affiché alors dans Unity ainsi qu'une IHM : le bouton \code{Add Image} permet de capturer l'image courante dans une liste, le bouton \code{Reset} permet de vider cette liste et le bouton \code{Calibrate} permet d'étalonner la caméra en utilisant cette liste d'images. Il est important de garder la caméra dans une position fixe durant l'étalonnage et de varier seulement la position et l'orientation de l'échiquier entre les images. On désactive aussi la mise au point automatique de la caméra, ce paramètre impactant en effet la matrice intrinsèque	$K$ de la caméra et les distorsions de l'objectif. Enfin, l'échiquier est surligné dans le flux vidéo pour aider à le positioner.

\begin{listingETS}{cpp}{lst:pinhole_calibration}{Code C++ simplifié d'Aruco Unity pour étalonner une caméra monoscopique avec un objectif rectilinéaire.}
  auto dictionary = getPredefinedDictionary(name);
  auto board = CharucoBoard::create(squaresNumberX, squaresNumberY, squareSideLength, markerSideLength, dictionary);

  vector<Mat> images = ... // Transmis depuis Unity

  auto P = new vector<vector<Point3f>>(images.size()); // Points P
  auto p = new vector<vector<Point2f>>(images.size()); // Points p

  for (int i = 0; i < images.size(); i++) {
    // Positions et identifiants des marqueurs
    auto markerPos = new vector<vector<Point2f>>();
    auto markerIds = new vector<int>();

    detectMarkers(images[i], dictionary, markerPos, markerIds, detectionParams);

    getBoardObjectAndImagePoints(board, markerPos, markerIds, P[i], p[i]);
  }

  Mat K, D; // Matrice intrinseque et parametres de distorsion
  vector<Mat> T, R; // Position et rotation de l'echiquier a chaque image
  double reprojectionError = calibrateCamera(P, p, board, imagesResolution, K, D, R, T, calibrationFlags);
\end{listingETS}

Le code C++ d'étalonnage exécuté alors en arrière-plan par Aruco Unity \autorefp{lst:pinhole_calibration} est simple : (1) on crée l'échiquier décrit, (2) on détecte l'échiquier dans et on extrait les points $P_i$ et $p_i$ pour chaque image de la liste, (3) on calcule la matrice intrinsèque $K$ et les paramètres de distorsion $D$.

Une fois l'étalonnage effectué, un score mesurant sa qualité est alors affiché. Il correspond à la somme de l'erreur de reprojection pour chaque image de la liste : c'est la distance en pixels entre les points $p_i$ observés et la projection simulée des points $P_i$ sur le plan image avec le modèle de sténopé mesuré par l'étalonnage. Ce score doit être le plus proche possible de zéro si l'on souhaite un bon alignement des deux caméras : un décalage trop important dans l'alignement des éléments vrituels sur l'image de la caméra physique briserait l'illusion de la RA.

Outre le score, les résultat de l'étalonnage sont enregistrés dans un fichier XML dans le dossier \code{Assets/ArucoUnity/CameraParameters/} que l'on pourra utiliser à volonté dans n'importe quelle application de RA utilisant cette caméra. L'\hyperref[appendix:ovrvision_camera_parameters]{Annexe~\ref{appendix:ovrvision_camera_parameters}} est le fichier de notre étalonnage de l'Ovrvision Pro.

\subsection{Correction des distorsions d'une caméra}
\label{subsec:pinhole_camera_undistortion}
Une fois étalonner, nous corrigeons les distorsions de la caméra, sur chacune des images capturées.

Corriger les distorsions une caméra consiste à reprojeter ses images selon un sténopé : l'opération inverse de la projection est d'abord faites sur les pixels, 

%  you may retrieve only sensible pixels alpha=0 , keep all the original image pixels if there is valuable information in the corners alpha=1 , or get something in between. When alpha>0 , the undistorted result is likely to have some black pixels corresponding to "virtual" pixels outside of the captured distorted image.

Aruco Unity effectue la rectification et l'alignement automatiquement sans que l'utilisateur n'ai besoin de s'en soucier. Par exemple :
\begin{enumerate}
  \item On ouvre la scène Unity \code{Assets/ArucoUnity/Scenes/Tracking.unity} déjà préparée pour le suivi de marqueurs avec une caméra monoscopique rectilinéaire étalonnée.
\end{enumerate}

Ainsi, lors du démarrage de l'application, le fichier XML d'étalonnage de la caméra est d'abord chargé, puis la caméra virtuelle est configurée avec les paramètres intrinsèques de la caméra physique. En outre, 

Concrètement, 

\begin{listingETS}{cs}{lst:pinhole_calibration}{Code C\# simplifié d'Aruco Unity pour rectifier une caméra monoscopique avec un objectif rectilinéaire.}
  // Demarrage de l'application
  void Start()
  {
    var rotation = new Cv.Mat(); // Mat vide = aucune rotation
    var newK = Cv.GetOptimalCameraMatrix()
    Cv.InitUndistortRectifyMap(K, D, rotation, newK, imageResolution, )
  }

  // Execute a chaque frame
  void Update()
  {
    ... // Suivi de marqueurs

    Cv.Remap(image, image, rectMapX, rectMapY, Cv.InterpolationFlags.Linear);
  }
\end{listingETS}

\subsection{Alignement de la caméra virtuelle}
\label{subsec:virtual_camera_alignement}
Les images de notre caméra physique étant corrigées, nous alignons maintenant les caméras physique et virtuelle.
%- Voir BuJo p.150 + AR-Rift PArt 5 pour les équations de configuration de la caméra virtuelle et du placement du background pour qu'il soit aligné avec le contenu 3D filmé par la caméra virtuelle

% TODO: parler rapidement de comment faire du tracking de marker

\subsection{Alignement avec une caméra avec un objectif \texten{fisheye}}
\label{subsec:fisheye_camera_calibration}
%- determine Knew https://medium.com/@kennethjiang/calibrate-fisheye-lens-using-opencv-part-2-13990f1b157f

Les objectifs utilisés sur l'Ovrvision Pro étant \texten{fisheye}, le modèle du sténopé ne s'y applique pas. En effet, contrairement à la perspective de l'\oe il humain ou des objectifs rectilinéaires, ces objectifs utilisent un autre type de projection perspective avec une distorsion en barillet volontairement importante pour capter un très grand champs de vision jusqu'à \ang{180} \reffigureETSp{JPRoche2004_1.jpg}.

\figureLayoutETS{JPRoche2004}{%
  \subfigureETS[0.15]{JPRoche2004_1.jpg}{Photographie par un objectif \texten{fisheye} : les distorsions sont très importantes dans l'image, les lignes droites sont vues courbes.}%
  \figurehspace%
  \subfigureETS[0.15]{JPRoche2004_2.jpg}{Correction de la photographie en une reprojection en perspective rectilinéaire : les lignes droites le sont à nouveau.}%
}{
  Photographie par un objectif \texten{fisheye} et sa correction.\\
  Adapté de \cite{JPRoche2004}.
}

\cite{Mei2007} proposent un autre modèle, utilisable pour les objectifs \texten{fisheye}. La caméra y est modélisée sur le principe suivant : la scène est d'abord projetée sur une sphère, puis une projection perspective rectilinéaire de la sphère est faite sur le plan image \reffigureETSp{Mei2007.jpg}. Comme pour une caméra avec un objectif rectilinéaire, nous avons besoin de corriger ces distorsions mais aussi de reprojetter les images capturées en perspective rectilinéaire pour pouvoir les aligner avec la caméra virtuelle \reffigureETSp{JPRoche2004_2.jpg}.

\figureETS[0.5]{Mei2007.jpg}{%
  Illustration du modèle de \cite{Mei2007} : un point $X$ de la scène est d'abord projeté en un point $X_s$ sur une sphère puis projeté à nouveau en perspective rectilinéaire en un point $p$ sur le plan image.\\
  Tiré de \cite{Mei2007}%
}

\subsection{Alignement avec une caméra stéréoscopique avec objectifs \texten{fisheye}}
\label{subsec:stereo_camera_calibration}
% - AdrianKaehler2017 -  Chapitre 19 pour l'étalonnage stereo + LeapMotionAlignmentCameraAR2015 pour expliquer pourquoi on applique aux caméras virtuelles l'ICD et non l'IPD

Une caméra stéréoscopique, comme l'Ovrvision Pro, est formée de deux caméra monoscopiques placés côte-à-côte capturant simultanément deux images enregistrées. Sur une caméra stéréoscopique idéale, les caméras gauche et droite auraient les mêmes paramètres intrinsèques et leurs plans images respectifs seraient coplanaires, donc leurs axes optiques parallèles, simplement séparés horizontalement de quelques centimètres \reffigureETSp{StereoProjection.png}. Nous l'avons vu, l'utilisation d'un objectif entraîne des distorsions radiales et tangentielles. Les caméras gauches et droites ont donc des paramètres intrinsèques légèrement différents et leurs plans images ne sont pas coplanaires et ont des orientations différentes.

\figureETS[0.9]{StereoProjection.png}{%
  Modèle d'une caméra stéréoscopique rectilinéaire idéale : les deux plans images sont coplanaires, distants horizontalement, et les deux couples objectif-plan image ont les mêmes paramètres intrinsèques ($f_g = f_d$). Les objectifs pourraient également être placés verticalement.
}

En plus de la correction de la caméra, nous devons la \emph{rectifier} pour rendre à nouveau coplanaires les deux plans images. Le type d'objectif n'a pas d'impact sur la rectification, qui est faite après avoir corrigé la caméra. Comme pour les distorsions, nous n'avons pas besoin de comprendre la procédure de rectification ; pour plus de détails, nous redirigeons le lecteur vers \cite[p. 419]{Bradski2008}.