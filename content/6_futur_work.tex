\chapter{Directions futures}
\label{ch:futur_work}

Mettre les directions futures et les recommendations : évaluer les scénarios du \autoref{ch:concept} au regard des résultats.

Refaire expérience avec technique d'interaction à la hololens et peephole

Refaire l'expérience de wedge avec téléphone seul, wedge, phone touch+ar, mid-air+ar, phone+mid-air+ar

multi-apps + Tâche de commutation d'affichages

bureau en realité augmentée

Dans le cadre Ethereal Planes : rester avec des fenêtres 2D mais à organiser en 3D (data mountain)

Tester les applications : Google Maps, multi-application (comme le Personal Cockpit)

Interaction pour le multi écran / écran étendu : vise avec regard, qui donne l'intention, (ou tête à défaut, voir l'étude pursuit target \cite{Esteves2017}) et agit avec le doigt. Serait bien pour scroller sur une app sur l'écran étendu (car on a vu que difficile de pan avec le doigt). Citer les article utilisation du regard sur les table de travail pour select les outils

Nous n'étudions que Zooming, car c'est le plus courant sur les appareils mobiles. Or on compare les techniques d'interactions de notre solution à un mobile seul. Il aurait été intéressant cependant, dans une deuxième expérimentation, de comparer ces techniques de visualization sur notre solution versus sur un appareil mobile seul.