\documentclass[letterpaper, twoside, 12pt,%
  memoire, francais, creativecommons, hyperref, withAlgo2e%
]{thETS} 

\usepackage{textcomp}

\title{Titre du document}

\author{Erwan NORMAND}
\authorcopyright{Erwan Normand}

\datesoutenance{``Date de soutenance''}
\datedepot{``Date du dépôt au Bureau des cycles supérieurs''}

\directeur{M.}{Michael J. McGuffin}{Département de génie logiciel et des TI
à l’École de technologie supérieure}
\president{M.}{Prénom Nom}{département et institution}
\examinexterne{M.}{Prénom Nom}{département et institution}{}

\listfiles

% Formattage des passages en langue étrangère
\let\oldforeignlanguage\foreignlanguage
\renewcommand{\foreignlanguage}[2]{\oldforeignlanguage{#1}{\emph{#2}}}

% Figure selon les normes de l'ÉTS
\newcommand{\figureETS}[3]{%
  \begin{figure}%
    \centering%
    \fbox{%
      \includegraphics[width=0.75\textwidth]{#1}
    }%
     \\ \parbox{0.75\textwidth}{\caption{#2}\label{figureETS:#3}}
  \end{figure}
}
\newcommand{\reffigureETS}[1]{Figure \ref{figureETS:#1}}
\captionsetup[figure]{labelfont=normalfont,font=singlespacing,justification=centering,labelsep=quad,position=below}

% Tableau selon les normes de l'ÉTS
\newenvironment{tableETS}[2]{%
  \begin{table}%
  \parbox{0.65\textwidth}{\caption{#1}\label{tableETS:#2}}%
  \centering%
}{%	
  \end{table}%
}
\newcommand{\reftableETS}[1]{Tableau \ref{tableETS:#1}}
\captionsetup[table]{labelfont=normalfont,font=singlespacing,justification=centering,labelsep=quad,position=above}

\begin{document}

\pagenumbering{Roman}

\maketitle

\presentjury

\begin{avantpropos}
  \lipsum[1]
\end{avantpropos}

\begin{remerciements}
  \lipsum[1]
\end{remerciements}

\begin{sommaire}{mot-clé1, mot-clé2}
  \lipsum[1]
\end{sommaire}

\begin{abstract}{Titre en anglais}{keyword1, keyword2}
  \lipsum[1]
\end{abstract}

\tableofcontents
\listoftables
\listoffigures

\begin{listofabbr}[3cm]
  \item [ETS] École de Technologie Supérieure
\item [ASC] Agence Spatiale Canadienne
\end{listofabbr}

%\begin{listofsymbols}[3cm]
%  \item [$\textbf{a}$] Première lettre de l'alphabet
\item [$\textbf{A}$] Première lettre de l'alphabet en majuscule
%\end{listofsymbols}

\cleardoublepage
\pagenumbering{arabic}
\reversemarginpar

\begin{introduction}
  \lipsum[1]
\end{introduction}

\begin{revuedelitterature}
  \lipsum[1]
\end{revuedelitterature}

\chapter{Méthodologie}
\section{Matériel}
- Principe : AR video see-through avec un casque VR diffusant des caméras stereo fisheye
  - Avantages/inconvénients par rapport à l'optical see through : « While optical see-through AR is an attractive ideal, in practice it is very difficult to achieve accurate registration with optical see-through AR systems. (Registration is the alignment of the virtual objects shown in the display and their real-world referent). Most of these limitations come from the properties of the display itself, not the calibration procedures. » : \url{https://www.artoolkit.org/documentation/doku.php?id=8_Advanced_Topics:config_optical_see-through}
- Choix techniques
  - Casque de VR (permet tracking)
  - Caméras : même fov (donc fisheye), même résolution + stéréo. Choix ovrvision
  - Leap Motion
  - Téléphone Android



\section{Conception du visiocasque}
\subsection{Logiciels}
- Unity
- OpenCV
  - ArUco
  - Calibration
- Unity networking
- Leap Motion



\section{Réalisation de la bibliothèque de réalité augmentée ArUco Unity}
- Voir 2016-01-25 pour l'archi :
  - Parler du principe pour dialoguer avec un plugin C++ : couche en C, gestion des pointeurs en faisant une API C\# qui encapsule ces appels à la couche C : plugin C++ OpenCV <-> couche en C <-> couche C\# reproduisant la couche C++ 
  - couche Unity avec des gameobjects et components au dessus de la couche C\#
- Aussi parler de la mémoire partagée entre Unity et OpenCv sur les images :
  - calcul de taille : 2 cameras * 950 px * 960 px * 3 bytes (RGB) = 5,47 MB
  - lecture du buffer dans sens différents (voir note 2017-04-11)
  - Threads et ordonnancement + copies des buffers images (c'était plus rapide de faire des copies que de faire attendre l'affichage avant le nouveau detect : voir note 2017-05-10) -> il n'y a pas d'attente/blocages entre les threads hormis sur les copies de buffer
- La doc qui a été faite, la petite PR pour rendre compatible les modues aruco et ccalib, le package Unity, les forks et ajouts sur internet, le package pour ovrvision (preuve que c'est extensible (citer les termes du cours MGL843))
- Utiliser des boards de 2 markers minimum pour la détection : beaucoup plus robuste qu'utiliser des markers seuls



\section{Calibration des caméras}
- Voir AdrianKaehler2017 - Learning OpenCV 3
  - Chapitre 11 + OpenCV3.3.0-Calib3dModule + notes 2016-11-28 pour le modèle pinhole de caméra + calibration
  - Chapitre 19 pour la calibration stereo + LeapMotionAlignmentCameraAR2015 pour expliquer pourquoi on applique aux caméras virtuelles l'ICD et non l'IPD
  - OpenCV3.3.0-CcalibModule (car module fisheye buggé et citer le papier qui dit que le modèle de caméra omnidir s'appliquer aussi aux caméras fisheye) pour la calibration fisheye
- Voir BuJo p.150 + AR-Rift PArt 5 pour les équations de configuration de la caméra virtuelle et du placement du background pour qu'il soit aligné avec le contenu 3D filmé par la caméra virtuelle
- Conseils/notes calibrations :
  - Utiliser une board la plus plate possible (attention à l'humidité de l'air qui est absorbée par le papier)
  - Utiliser une bonne lumière pour que la board soit bien détectée et sans reflets
  - Désactiver l'autofocus de la caméra : une calibration se fait pour une focale fixe (les distorsions restent les même mais pas la camera matrix)
  - La caméra ou la board doit rester fixe pendant la calibration
  - Prendre des dizaines de captures remplissant uniformément l'espace de capture de la caméra en variant les angles de capture
  - L'objectif est d'avoir une erreur de reprojection inférieure à 1 pixel
  - OpenCV est système main droite dans son système de coordonnées (\url{http://homepages.inf.ed.ac.uk/rbf/CVonline/LOCAL_COPIES/OWENS/LECT9/img4.gif}) alors qu'Unity est système main gauche : il suffit d'inverser l'axe des Y (\url{https://answers.unity.com/storage/temp/8053-spaces.jpg}) : faire un petit graphe comme la 2e image
  - OpenCV encode sa rotation dans un vecteur dont les coordonnées normalisées donnent l'axe et sa norme l'angle autour de cet axe, alors qu'Unity utilise des quaternions. Adapté ce calcul (\url{http://www.euclideanspace.com/maths/geometry/rotations/conversions/angleToQuaternion/}) pour passer du premier au second : \url{https://github.com/enormand/aruco-unity/blob/master/src/aruco_unity_package/Assets/ArucoUnity/Scripts/Plugin/Cv/Vec3d.cs}. Voir BuJo 2017-11-07.

AR video see through\\
Goal: Align the 3D content filmed by the virtual camera with the images filmed by the physical camera.\\
How: Match the camera parameters of the two cameras. Intrinsic camera parameters: camera matrix (how the 3D world is projected into a 2D image), distorsion coefficients (), image size. Extrinsic camera parameters: position and rotation in the 3D world.

Pinhole camera model (physical camera)\\
\url{https://www.scratchapixel.com/lessons/3d-basic-rendering/3d-viewing-pinhole-camera/how-pinhole-camera-works-part-1}\\
\url{https://www.scratchapixel.com/lessons/3d-basic-rendering/3d-viewing-pinhole-camera/how-pinhole-camera-works-part-2}

Perspective projection (virtual camera)\\
Perspective projection matrix (virtual camera frustum) = projection transform (camera matrix) + Normalized Device Coordinates (NDC) matrix (orthogonal projection) : \url{http://ksimek.github.io/2013/06/03/calibrated_cameras_in_opengl/}\\
Perspective projection matrix = simpler pinhole camera model : \url{http://ksimek.github.io/2013/08/13/intrinsic/}\\
« They all require a precise understanding of how the pixels in a 2D image relate to the 3D world they represent. In other words, they all hinge on a strong camera model. »\\

Undistortion and rectification of pinhole camera model\\
Voir BuJo

Perspective projection rectification from fisheye image in the unified projection model used in omnidirectional camera calibration (ccalib OpenCV's module)\\
Omnidir camera model : \url{http://rpg.ifi.uzh.ch/docs/omnidirectional_camera.pdf}\\
determine Knew \url{https://medium.com/@kennethjiang/calibrate-fisheye-lens-using-opencv-part-2-13990f1b157f}\\
explications : \url{http://www.bobatkins.com/photography/technical/field_of_view.html}\\
fisheye projections : \url{https://wiki.panotools.org/Fisheye_Projection}\\
various lens projections : \url{http://michel.thoby.free.fr/Fisheye_history_short/Projections/Various_lens_projection.html}\\
ccalib article : \url{https://hal.archives-ouvertes.fr/file/index/docid/767674/filename/omni_calib.pdf}\\

Stereo calibration\\



\section{Communication réseau entre le téléphone et le visiocasque}
- Réalisation de la bibliothèque DevicesSyncUnity basée sur Unity Unet
\chapter{Étude expérimentale}
\label{ch:experiment}

\section{Description de la tâche expérimentale}
\label{sec:experiment_description}

Voir la liste de tâches : \url{https://docs.google.com/document/d/1X5-XW-9GTCz254PXlskjNka-kS-X7MhGXB6Ixv8vAjk/edit#}

Leap in:
User-Defined Gestures for Augmented Reality \url{https://hal.inria.fr/hal-01501749/document}
Pourquoi on teste l'interaction directe et pas mid-air interaction : c'est lent Vulture: a mid-air word-gesture keyboard \url{https://dl.acm.org/citation.cfm?id=2556964}
Grasp-Shell vs Gesture-Speech: A comparison of direct and indirect natural interaction
techniques in Augmented Reality \url{https://ir.canterbury.ac.nz/bitstream/handle/10092/11090/12652683_paper138-cr.pdf?sequence=1}

Touch in:
User-Defined Gestures for Surface Computing \url{https://www.microsoft.com/en-us/research/wp-content/uploads/2009/04/SurfaceGestures_CHI2009.pdf}
Référence industrie (Android) : \url{https://material.io/guidelines/patterns/gestures.html}

justification for why we won’t be doing experimental comparison with ray cast: because Leap Motion and HoloLens require your hand to be visible, not pulled back, so ray cast selection wouldn’t help the user avoid fatigue

Le détail du développement de l'expérience est décrit dans l'annexe \ref{appendix:experiment} \nameref{appendix:experiment}.


\section{Protocole experimental}
\subsection{Procédure}

\subsection{Matériel}
Un ordinateur de bureau avec un processeur Intel Core i5 7400, 8 Go DDR4 de mémoire vive, et une carte graphique NVIDIA GeForce GTX 1060, sous Windows 10 a été utilisé pour faire tourner le visiocasque de RA. Une carte WiFi externe a été ajouté pour la communication avec le téléphone.\\
Le téléphone utilisé était un Xiaomi Redmi Note 4. C'est un téléphone récent, sous une version récente d'Android (version 7), avec une bonne puissance de calcul et un grand écran HD de 5,5 pouces. Il est léger et tient confortablement dans une main.\\
Le visiocasque de RA, l'outil de suivi des mains ainsi que les logiciels utilisés sont décrits dans la \refsection{sec:technical_choices}.

\subsection{Participants}
Nous avons recruté 12 personnes volontaires pour participer à l'expérience, agés entre 18 et 49 ans : six hommes et six femmes. Tous avaient une vision normale ou portaient un dispositif de correction de vision.

\subsection{Hypothèses}

\subsection{Collecte des données}


\section{Résultats}
\chapter{Discussion}
\section{Leçons apprises}



\section{Pistes de conception d'interfaces et d'interactions pour la réalité augmentée}
Quelles interactions sont plus intéressantes et dans quels contextes ? Direct (leap), indirect (hololens), touch
Quelles techniques d'interactions sont à utiliser ?



\section{Pistes de recherches futures}
Refaire expérience avec technique d'interaction à la hololens et peephole
multi-apps + Tâche de commutation d'affichages
bureau en realité augmentée

\begin{conclusion}
  \lipsum[1]
\end{conclusion}

%\appendix
%\multiannexe %% Lorsqu'on a plus qu'une annexe.
%\lipsum[1]

\newpage
\begin{spacing}{1}
  \nocite{*}
  \bibliographystyle{bibETS}
  \addcontentsline{toc}{chapter}{BIBLIOGRAPHIE}
  \bibliography{./bibliography/thesis}
\end{spacing}

\end{document}
